\documentclass[12pt,a4paper,openany,aps,pra]{book}
\usepackage{graphicx}
\usepackage{gensymb}
\usepackage{longtable}
\usepackage{epsfig}% Include figure files
\usepackage{dcolumn}% Align table columns on decimal point
\usepackage{bm}% bold math
\usepackage{anysize}
\usepackage[utf8x]{inputenc}
\usepackage[T1]{fontenc}
%\usepackage[polish]{babel}
\usepackage{amsmath}
\usepackage[urlcolor=blue,colorlinks=true,citecolor=blue,linkcolor=blue,pdfstartview={FitH},bookmarks=false]{hyperref}
\def\mb{\bm}
\usepackage[left=2cm,right=2cm,top=3cm,bottom=3cm]{geometry}
\DeclareMathOperator{\Tr}{Tr}


\begin{document}


\begin{titlepage}
	\centering
	{\Large\scshape Przemys\l{}aw Piekarz\par}
    \vspace{4cm}
	{\scshape\bfseries\Huge Ab initio methods\par}
	\vspace{1cm}
	{\scshape\bfseries\Huge in solid state physics\par}	
	\vfill
	{\scshape\large Department of Computational Materials Science \\ Institute of Nuclear Physics \\ Polish Academy of Sciences\par}
\end{titlepage}

\newpage
\thispagestyle{empty}
%\thispagestyle{plain} % empty
\mbox{}

\tableofcontents


\chapter*{Introduction}
\addcontentsline{toc}{chapter}{Introduction}  

First principles calculation methods, also called {\it ab initio} methods, are based on the laws of quantum mechanics.
The application of the Schr\"{o}dinger equation~\cite{Schrodinger} to atoms and molecules enabled to explain the basis properties of orbitals and chemical bonds~\cite{HL,hartree28,mulliken,JC}. 
Many-electron systems, such as atoms, molecules and solids, are subject to Fermi-Dirac statistics~\cite{fermi26,Dirac26}, which is a direct consequence of the Pauli exclusion principle~\cite{Pauli25}.
Application of quantum statistics to a homogeneous electron gas, which is the simplest model of the metallic state,
made it possible to explain the problems of the classical Drude-Lorentz theory of conductivity.
The band theory of periodic systems, which arose thanks to the works of Bloch~\cite{Bloch}, Peierls~\cite{peierls} and Wilson~\cite{Wilson},
explained the significant differences in the electrical conductivity of materials and created the basis for dividing all crystals into metals, insulators and semiconductors.

Due to many-body interactions occurring in electronic systems, the Schr\"{o}dinger equation
cannot be solved accurately, and therefore approximate methods are needed.
One of the first approaches to quantitative calculations was the Hartree \cite{hartree28} approximation,
where the wave function is given as a product of single-particle functions.
In the single-particle model, the interaction of electrons is replaced by an effective potential,
in which each electron moves. This potential is composed of the attractive interaction generated by atomic nuclei
and from the repulsive part coming from the remaining electrons.
Since the electronic part of the potential depends on the wave functions we want to obtain,
the Schr\"{o}dinger equation must be solved in a self-consistent manner.
Taking into account the Pauli exclusion principle for the multi-electron wave function, which in its simplest form is written
is in the form of the Slater determinant, the Hartree-Fock equation can be derived~\cite{fock30,slater30,slater51}.
In this equation, in addition to the usual Coulomb interaction, there is a non-local
exchange interaction. This is an interaction that occurs only between electrons with parallel spins,
effectively reducing the Coulomb repulsion between them.
The Hartree-Fock approach describes electron systems approximately because it does not properly take into account correlations,
which occur in the many-electron quantum state \cite{wigner34,GB}.
This is also related to improper screening of Coulomb interactions,
which leads to significantly overestimated values of the energy gap in solids.
Methods based on the Hartree-Fock approach, the so-called multi-determinant methods, correctly take into account electronic correlations, but require time-consuming computer calculations. They are mainly used in quantum chemistry to study molecular systems.

The first calculations of the electronic structure of metals used methods in which independent electrons interact
with periodic atomic potential.
In the cell method, proposed by Wigner and Seitz~\cite{wigner33}, the crystal potential is the sum of
spherically symmetric potentials generated by each atom.
Using Bloch's theorem, all we need to do is find solutions to the Schr\"{o}dinger equation
in a single primitive cell, with appropriate boundary conditions at the cell boundary.
In 1937, Slater introduced the augmented plane wave (APW) method. In this approach, the entire crystal is divided
to the areas around the atomic positions (defined by the radius of the atomic sphere) and the remaining interstitial part.
In each of these areas, a different form of the wave function is used. Inside the atomic spheres, the potential changes rapidly and
wave functions, which behave similarly to atomic orbitals, are calculated in the basis of spherical harmonics.
However, in the interstitial area, both the potential and wave functions are slowly changing, and the natural basis are plane waves.
Additionally, the condition of continuity at the border of areas must be met.
In another approach, proposed by Korring~\cite{K47} and Kohn and Rostoker~\cite{KR54},
Green's functions 
and scattering theory are used to determine electronic states in the crystal.
The main drawback of these methods was the approximate nature of the electronic potential.
The most frequently used potential was the {\it muffin-tin} type, which has spherical symmetry in the area of the atomic core,
and outside it takes a constant value.
A step forward was the orthogonalized plane wave (OPW) approach proposed by Herring~\cite{herring40}, which became
the basis of the pseudopotential method~\cite{Antoncik,KP}.
In this approach, the exact atomic potential is replaced by an approximate pseudopotential, having a finite value in the region of the atomic core,
which allows the calculation of an approximate wave function (pseudo wave function) in the basis of plane waves in the entire crystal region.
 
The real breakthrough was the formulation of density functional theory (DFT) by Hohenberg, Kohn and Sham~\cite{kohn64,kohn65}.
The basic theorem of this theory says that the total energy of a given system,
taking into account exchange and correlation interactions, is a functional
of electron density, which can be determined by minimizing the ground state energy.
In practical applications, the electron density is determined from single-particle wave functions, calculated as
self-consistent by solving the Kohn-Sham equation.
The exact values of the exchange-correlation potential are not known and must be determined within known approximations.
The basic approach is the local density approximation (LDA),
in which the exchange and correlation energies are obtained from accurate calculations for a homogeneous electron gas \cite{CeperleyAlder80,PZ,VWN}.
The next step was to account for non-local effects within the generalized gradient approximation
(GGA) \cite{Langreth83,Pardew86,Becke88}.
Since the formulation of DFT, many methods have been proposed to solve the Kohn-Sham equations, such as
the method of associated plane waves \cite{Andersen75}, or approaches based on pseudopotentials \cite{HSC,BHS,Vanderbilt90}.
Thanks to these improvements, density functional theory became the basis for calculating electronic structure~\cite{jones}, determining crystal parameters~\cite{payne}, studying lattice dynamics~\cite{parlinski, baroni} and many other material properties~\cite {martin}.
For its formulation, Walter Kohn received the Nobel Prize in Chemistry in 1998.

% **[REVIEW NEEDED]**: Technical terminology for DFT approximations and strongly correlated systems
However, there are systems in which the approximations used in DFT break down and fail to correctly
describe the electronic structure. According to classical band theory, many transition metal oxides, such as NiO or MnO,
should be in a metallic state, due to the partial filling of $3d$ states.
These materials are actually insulators, because the strong Coulomb interaction in the $3d$ states
prevents the free movement of electrons and leads to their localization \cite{peierls}.
In contrast to band insulators, these materials are called Mott insulators \cite{mott}.
Such local electron interaction, described within the Hubbard model by the parameter $U$ \cite{hubbard},
is much stronger in $d$ and $f$ states than in the more extended (delocalized) $s$ and $p$ states.
The electronic structure of transition metal oxides, determined within LDA or GGA calculations,
either has no gap (CoO) or the gap is significantly underestimated (NiO) \cite{terakura}.
Too small an energy gap is also a problem in calculations for classical semiconductors,
such as silicon or germanium, which means that the exchange-correlation potential is not
well described in these materials either.

% **[REVIEW NEEDED]**: Self-interaction correction terminology and hybrid functional description
The basic approximations used to determine the exchange-correlation energy are a source of errors that effectively lead to electron self-interaction.
This effect is particularly strong for $d$ or $f$ states and leads to an unfavorable increase in energy upon electron localization.
This is the main reason for the incorrect metallic state or too small energy gap obtained
in calculations for transition metal oxides.  
One method for correcting this effect (SIC) \cite{PZ} consists of directly calculating
the self-interaction energy and subtracting it from the total energy functional.
This method enables improvement of the electronic structure, but is not easy to implement in self-consistent calculations.
The electronic state can also be improved by using hybrid functionals, in which the approximate exchange energy is partially
replaced by the exact value obtained within the Hartree-Fock method \cite{becke93}.
This stems from the fact that in the Hartree-Fock approximation there is no self-interaction effect.
  
% **[REVIEW NEEDED]**: LDA+U method description and advanced many-body methods  
The electronic potential for localized states can also be improved by adding to the DFT functional
an additional term proportional to the interaction energy $U$. This can be done in the mean-field
approximation within the LDA+U method \cite{anisimov}. Application of this method to transition metal oxides \cite{anisimov}
or high-temperature superconductors \cite{czyzyk} has significantly improved their electronic structure 
and magnetic properties compared to standard DFT calculations.
Also in the case of other complex materials, in which interactions between spin, orbital,
and lattice degrees of freedom occur, this approach gives qualitatively better results \cite{orbital1,orbital2,orbital3}.
In strongly correlated metals, such as rare earths and actinides,
the LDA+U method allows only partial inclusion of effects related to electron localization.
A good example is plutonium crystal, where the very large discrepancy between the theoretically calculated
volume and the experimental value (about 35\%) can be corrected using
a realistic value of the parameter $U$\cite{Pu-LDAU}.
Unfortunately, dynamic effects such as fluctuation of magnetic moments in the paramagnetic state, or the Kondo effect
cannot be correctly described at the level of mean-field theory.
In the more advanced dynamical mean-field theory (DMFT) method, many-body effects
are included by solving the Anderson impurity model using exact diagonalization
or quantum Monte Carlo \cite{Georges}. The solutions contain information about the self-energy and lifetime
of electronic states, which are not available within DFT calculations.
The quantum Monte Carlo method itself is currently being developed very intensively and is used for calculations
of the electronic structure of molecules and crystalline systems.


\chapter{Electron interactions}

\section{Basic properties}

% **[REVIEW NEEDED]**: Fundamental quantum mechanics terminology - electromagnetic interactions and Hamiltonian description
Among all interactions that occur in nature, the decisive factors for the physical and chemical properties of atomic systems
are electromagnetic interactions. 
Interatomic bonds, which give rise to molecules and solids, 
are the result of mutual electrostatic interactions in the system of electrons and atomic nuclei.
The Hamiltonian, which describes an arbitrary system of atoms, has the form
%
\begin{equation}
H=-\frac{\hbar^2}{2m}\sum_{i}\nabla_i^2-\sum_{i,j} \frac{Z_{j}e^2}{|\bm{r}_i-\bm{R}_j|}+\frac{1}{2}\sum_{i\neq j}\frac{e^2}{|\bm{r}_i-\bm{r}_j|}
-\sum_j \frac{\hbar^2}{2M_j}\nabla_j^2 -\frac{1}{2}\sum_{i\neq j} \frac{Z_iZ_je^2}{|\bm{R}_i-\bm{R}_j|},
\label{hamiltonian}
\end{equation}
%
where $\bm{r}_i$ denote the positions of electrons, $\bm{R_j}$ the positions of atomic nuclei, $Z_j$ the nuclear charges, $M_j$ the nuclear masses, and $m$ is the electron mass.
The first three terms describe respectively the kinetic energy of electrons, the interaction of electrons with nuclei, and the interactions
between electrons. The last two terms are the kinetic energy of atomic nuclei and the interaction between
nuclear charges. The complete wave function of the complex system of electrons and atomic nuclei $\Psi(\bm{r}_i,\bm{R}_j)$ satisfies the Schr\"{o}dinger equation
%
\begin{equation}
H\Psi(\bm{r}_i,\bm{R}_j) = E\Psi(\bm{r}_i,\bm{R}_j),
\label{Sch}
\end{equation}
%
where $E$ is the total energy of the system.

% **[REVIEW NEEDED]**: Born-Oppenheimer approximation description and adiabatic treatment
The quantum-mechanical description of a crystal can be simplified by exploiting the large ratio of the atomic nuclear mass to the electron mass,
which for the lightest nucleus, hydrogen, equals $m_p/m_e=1836$.
This causes the dynamics of electrons to be much faster than that of atomic nuclei, and electronic states adjust very quickly (adiabatically)
to the current positions of atoms. Typical energies associated with electron motion are on the order of 1 eV, whereas average atomic vibration energies
in crystals are at the level of 10 meV. 
Thus, in the first approximation, we can solve the Schr\"{o}dinger equation for the electronic subsystem at fixed atomic positions and
neglect the influence of quantum features of atomic dynamics on electronic wave functions.
This approach is called the adiabatic approximation or the Born-Oppenheimer approximation \cite{BO}.
Introducing the wave function of the entire system as a product of the electronic function $\Phi(\bm{r_i},\bm{R_j})$ and the nuclear function $\chi(\bm{R}_j)$
%
\begin{equation}
\Psi(\bm{r}_i,\bm{R}_j) = \Phi(\bm{r}_i,\bm{R}_j)\chi(\bm{R}_j)
\label{wave}
\end{equation}
%
one can separate (\ref{Sch}) into two equations 
%
\begin{equation}
(-\frac{\hbar^2}{2m}\sum_{i=1}^N \nabla_i^2-\sum_{i,j} \frac{Z_{j}e^2}{|\bm{r}_i-\bm{R}_j|}+\frac{1}{2}\sum_{i\neq j}
\frac{e^2}{|\bm{r}_i-\bm{r}_j|})\Phi(\bm{r}_i,\bm{R}_j)=E_n(\bm{R}_j)\Phi(\bm{r}_i,\bm{R}_j),
\label{Sch-el}
\end{equation}
%
\begin{equation}
(-\sum_j \frac{\hbar^2}{2M_j} \nabla_j^2 -\sum_{i,j} \frac{Z_iZ_je^2}{|\bm{R}_i-\bm{R}_j|}+E_n(\bm{R}_j))\chi_{n\alpha}(\bm{R}_j)=\varepsilon_{n\alpha}\chi_{n\alpha}(\bm{R}_j).
\end{equation}
%
The first equation describes the wave functions and eigenvalues $E_n(\bm{R}_j)$ of the electronic system at fixed positions of atomic nuclei, where $n$ denotes a defined set of quantum numbers of the electronic state. From the second equation we can obtain the wave functions and eigenvalues $\varepsilon_{n\alpha}$ associated with the motion of atomic nuclei,
where $\alpha$ is a quantum number that characterizes these quantities. The potential energy in the equation describing the motion of atomic nuclei depends on the mutual interaction between nuclei and on the energy of the electronic subsystem $E_n(\bm{R}_j)$. Both quantities are functions of the current positions of all atomic nuclei.  

Let us write the wave function of a system of $N$ electrons, at fixed positions of atomic nuclei, in the form $\Phi(\bm{r}_1,\bm{r}_2,...,\bm{r}_N)$. 
Knowledge of the wave function allows us to determine many basic physical quantities for a given system. In particular, the electron density at point $\bm{r}$
is given by the formula
%
\begin{equation}
n(\bm{r})=N\int d\bm{r}_2...d\bm{r}_N \Phi^*(\bm{r},\bm{r}_2,...,\bm{r}_N) \Phi(\bm{r},\bm{r}_2,...,\bm{r}_N).
\end{equation}
%
% **[REVIEW NEEDED]**: Pauli exclusion principle and fermion antisymmetry
The fundamental property of the electronic wave function is its antisymmetry. This means that exchanging the positions of two electrons causes a sign change
%
\begin{equation}
\Phi(\bm{r}_1,\bm{r}_2,...,\bm{r}_i,...,\bm{r}_j,...,\bm{r}_N)=-\Phi(\bm{r}_1,\bm{r}_2,...,\bm{r}_j,...,\bm{r}_i,...,\bm{r}_N).
\label{anty}
\end{equation}
%
This property follows from the Pauli exclusion principle, which states that two fermions (i.e., particles with half-integer spin) cannot occupy the same quantum state - a state described by the same set of quantum numbers.

% **[REVIEW NEEDED]**: Hydrogen molecule example - Heitler-London method terminology
The Schr\"{o}dinger equation (\ref{Sch-el}) contains many-body interactions and has no exact analytical solutions, except for the hydrogen atom or other one-electron systems. Exact numerical solutions of the Schr\"{o}dinger equation can be obtained only for individual atoms and small molecules.
As an example, let us consider a simple system with two electrons, which is the hydrogen molecule H$_2$.
The Hamiltonian of this system, after including the Born-Oppenheimer approximation, takes the form
%
\begin{equation}
H=-\frac{\hbar^2}{2m}\nabla_1^2-\frac{\hbar^2}{2m}\nabla_2^2-\frac{e^2}{r_{1A}}-\frac{e^2}{r_{1B}}-\frac{e^2}{r_{2A}}-\frac{e^2}{r_{2B}}+\frac{e^2}{r_{12}}+\frac{e^2}{r_{AB}},
\end{equation}
%  
where $r_{1A}$, $r_{1B}$, $r_{2A}$, $r_{2B}$ denote the distances of electrons (1 and 2) from the two protons ($A$ and $B$), $r_{12}$ is the distance between electrons,
and $r_{AB}$ is the distance between protons. 
In 1927, Heitler and London proposed a wave function in the form \cite{HL}
%
\begin{equation}
\Phi(\bm{r}_1,\bm{r}_2)=N_{\pm}[\psi_A(\bm{r}_1)\psi_B(\bm{r}_2)\pm \psi_B(\bm{r}_1)\psi_A(\bm{r}_2)]\chi_{\sigma},
\end{equation}
%
where $N_{\pm}$ is the normalization factor, $\chi_{\sigma}$ is the spin part of the wave function, and the four functions $\psi_{\alpha}(\bm{r})$
are $1s$ orbitals for the ground state of the hydrogen atom
%
\begin{equation}
\psi_{\alpha}(\bm{r})=\frac{1}{\sqrt{\pi a_0^3}}e^{-\frac{|\bm{r}-\bm{r}_{\alpha}|}{a_0}},
\end{equation}
where $a_0$ is the Bohr radius, and $\bm{r}_{\alpha}$ is the position of proton $\alpha=A$ or $B$.
Taking into account the antisymmetry condition (\ref{anty}), the wave function takes one of the allowed forms
%
\begin{equation}
\Phi_S(\bm{r}_1,\bm{r}_2)=N_{+}[\psi_A(\bm{r}_1)\psi_B(\bm{r}_2)+\psi_B(\bm{r}_1)\psi_A(\bm{r}_2)]\frac{1}{\sqrt{2}}(|\uparrow\downarrow\rangle-|\downarrow\uparrow\rangle),
\end{equation}
%
for total spin $S=0$ (singlet state), and
%
\begin{equation}
\Phi_T(\bm{r}_1,\bm{r}_2)=\begin{cases}
N_{-}[\psi_A(\bm{r}_1)\psi_B(\bm{r}_2)-\psi_B(\bm{r}_1)\psi_A(\bm{r}_2)]|\uparrow\uparrow\rangle, \\
N_{-}[\psi_A(\bm{r}_1)\psi_B(\bm{r}_2)-\psi_B(\bm{r}_1)\psi_A(\bm{r}_2)]\frac{1}{\sqrt{2}}(|\uparrow\downarrow\rangle+|\downarrow\uparrow\rangle), \\
N_{-}[\psi_A(\bm{r}_1)\phi_B(\bm{r}_2)-\psi_B(\bm{r}_1)\psi_A(\bm{r}_2)]|\downarrow\downarrow\rangle, 
\end{cases}
\end{equation}
%
for spin $S=1$ (triplet state). The ground state of the hydrogen molecule is the singlet state, whose energy
$E_S$ is lower than the energy of the triplet state $E_T$ for any distance between protons $r_{AB}$.
The bound state corresponds to the energy minimum $E_S=\langle\Phi_S|H|\Phi_S\rangle$, which we obtain for $r_{AB}=0.87$ \AA. This distance is larger than the experimental value of 0.74 \AA. Meanwhile, the calculated dissociation energy of the molecule into two hydrogen atoms is $E_d=3.14$ eV and is smaller than the measured energy of 4.75 eV. This is an example of a covalent bond, in which two electrons with opposite spins become shared, which leads
to a lowering of the energy of the entire system compared to the sum of energies of two separate atoms.  
In contrast to the singlet state, the triplet state does not form a bound state of two hydrogen atoms.
More accurate values $r_{AB}=0.74$ \AA\ and $E_d=3.63$ eV are obtained in the Hartree-Fock approximation, which will be the subject of the next chapter.
The best results are obtained by the variational method, in which the wave function is written in the general form of a product of a symmetric spatial function and an antisymmetric spin function \cite{JC}
%
\begin{equation}
\Phi_S(\bm{r}_1,\bm{r}_2)=\Psi(\bm{r}_1,\bm{r}_2)\chi_0.
\end{equation} 
%
The spatial part of the wave function depends on $M$ variational parameters, $p_1$, $p_2$,..., $p_M$.
Minimizing the system energy for $M=13$ yields values $E_d=4.70$ eV and $r_{AB}=4.74$ \AA, in very good agreement with experimental data \cite{JC}.   
For most studied molecular systems and solids, it is not possible to calculate exact solutions of the Schr\"{o}dinger equation. 
Therefore, it is necessary to use approximate methods, mainly numerical, which allow us to determine the best possible wave functions and energies of electronic states
in a reasonable computational time. 

\section{Hartree-Fock equation}
\label{sec:HF}

% **[REVIEW NEEDED]**: Slater determinant and Hartree-Fock approximation terminology
One of the first methods used to describe molecular systems is the Hartree-Fock approximation \cite{hartree28,fock30,slater30}. In this approach, the wave function of a system of $N$ electrons has the form of a Slater determinant
%
\begin{equation}
\Phi = \frac{1}{\sqrt{N!}} 
\begin{vmatrix}
\phi_1(\bm{r}_1,\sigma_1) &  \phi_2(\bm{r}_1,\sigma_1)  & \dots & \phi_N(\bm{r}_1,\sigma_1) \\ 
\phi_1(\bm{r}_2,\sigma_2) &  \phi_2(\bm{r}_2,\sigma_2)  & \dots & \phi_N(\bm{r}_2,\sigma_2) \\
\dots                &  \dots                 &       &\dots \\
\phi_1(\bm{r}_N,\sigma_N) &  \phi_2(\bm{r_N},\sigma_N)  & \dots & \phi_N(\bm{r}_N,\sigma_N) \\
\end{vmatrix},
\label{slater}
\end{equation}
%
where the single-electron functions are products of the part depending on position $\bm{r}_i$
and on spin $\sigma_i$
%
\begin{equation}
\phi_i(\bm{r}_i,\sigma_i)=\psi_i^{\sigma_i}(\bm{r}_i)\xi_i(\sigma_i).
\end{equation}
%
The wave function written in the form of a Slater determinant satisfies the basic
properties of a system of indistinguishable particles.
Exchanging two electrons corresponds to exchanging two columns in the determinant, which causes
a sign change of the wave function in accordance with the antisymmetry condition.
If two electrons are in the same quantum state, then two columns
are identical and the entire determinant vanishes, which is consistent with the Pauli exclusion principle. 

The solution of the Schr\"{o}dinger equation corresponds to a wave function for which the average
of the Hamiltonian takes a minimum value. Assuming orthonormality of the function $\Phi$,
this corresponds to the variational principle in the form
%
\begin{equation}
\delta(\langle\Phi|H|\Phi\rangle - E\langle\Phi|\Phi\rangle)=0,
\label{var}
\end{equation}
%
where the symbol $\delta$ denotes the variational derivative.
Calculating the average of the Hamiltonian for the electronic subsystem (\ref{Sch-el}) in the state described by the wave function (\ref{slater}), we obtain
%
\begin{equation}
\langle\Phi|H|\Phi\rangle=\sum_{i,\sigma}\int d\mb{r} \psi_i^{\sigma*}(\mb{r})[-\frac{\hbar^2}{2m}\nabla_i^2+V_Z(\mb{r})]\psi_i^{\sigma}(\mb{r}) 
+E_H+E_x,
\label{E_HF}
\end{equation}
%
% **[REVIEW NEEDED]**: Hartree energy and exchange energy terminology
where $V_Z$ is the electrostatic potential generated by atomic nuclei, and $E_H$ is called the Hartree energy 
%
\begin{equation}
E_H=\frac{e^2}{2}\sum_{i,j,\sigma,\sigma'}
\int d\mb{r} \int d\mb{r}' \frac{\psi_i^{\sigma*}(\mb{r})\psi_j^{\sigma'*}(\mb{r}')\psi_i^{\sigma}(\mb{r})\psi_j^{\sigma'}(\mb{r}')}{|\mb{r}-\mb{r}'|}.
\end{equation}
%
Introducing the charge density at point $\mb{r}$
%
\begin{equation}
n(\mb{r})=e\sum_{i,\sigma} |\psi_i^{\sigma}(\mb{r})|^2,
\label{dens}
\end{equation}
%
we can write the Hartree energy in a simpler form
%
\begin{equation}
E_H=\frac{1}{2} \int d\mb{r} \int d\mb{r}' \frac{n(\mb{r})n(\mb{r}')}{|\mb{r}-\mb{r}'|}, 
\label{Hartree}
\end{equation}
%
which corresponds to the classical Coulomb interaction energy for a given distribution of electric charge density. 
%
$E_x$ is called the exchange energy and equals
%
\begin{equation}
E_x=-\frac{e^2}{2}\sum_{i,j,\sigma}
\int d\mb{r} \int d\mb{r}' \frac{\psi_i^{\sigma*}(\mb{r})\psi_j^{\sigma*}(\mb{r}')\psi_i^{\sigma}(\mb{r}')\psi_j^{\sigma}(\mb{r})}{|\mb{r}-\mb{r}'|}.
\label{exc}
\end{equation}
%
% **[REVIEW NEEDED]**: Exchange interaction and exchange hole concept
The exchange energy is a quantum quantity that has no counterpart in classical physics. 
The exchange interaction is characterized by two important features that are closely related to each other. 
First of all, this interaction concerns only electrons with parallel spins. 
According to the Pauli principle, two electrons with parallel spins cannot have the same
remaining quantum numbers, which would be associated with occupying the same quantum state. 
Effectively, this leads to an increase in the distance between such electrons and
a reduction of the Coulomb interaction energy. From this follows the second feature
of the exchange interaction: the negative value of this interaction energy. One can interpret the exchange interaction
as the Coulomb interaction of an electron with a positive charge called the {\it exchange hole},
which causes a reduction of the negative charge density around each electron. 

After applying the variational method to (\ref{E_HF}), we obtain the Hartree-Fock equation
%
\begin{equation}
[-\frac{\hbar^2}{2m}\nabla_i^2+V_Z(\mb{r})+V_H(\mb{r})]\psi_i^{\sigma}(\mb{r})
-\frac{e^2}{2}\sum_{j,\sigma}\int d\mb{r}' \frac{\psi_j^{\sigma*}(\mb{r}')\psi_i^{\sigma}(\mb{r}')}{|\mb{r}-\mb{r}'|}\psi_j^{\sigma}(\mb{r})=E_i\psi_i^{\sigma}(\mb{r}),
\label{HF}
\end{equation} 
%
where $V_H$ denotes the Hartree potential
%
\begin{equation}
V_H(\mb{r})=\int d\mb{r}' \frac{n(\mb{r}')}{|\mb{r}-\mb{r}'|}.
\end{equation}
%
The Hartree-Fock equation is a nonlinear equation because the last term of the Hamiltonian contains the wave function we are looking for.
Slater proposed a different form of equations (\ref{HF}), which
allows better understanding of the nature of the exchange interaction \cite{slater51}.
Multiplying and dividing the exchange term by $\psi_i^{\sigma}(r)$, we obtain
%
\begin{equation}
[-\frac{\hbar^2}{2m}\nabla_i^2+V_Z(\mb{r})+V_H(\mb{r})
-\frac{e}{2}\int d\mb{r}' \frac{n(\mb{r},\mb{r}')}{|\mb{r}-\mb{r}'|}]\psi_i^{\sigma}(\mb{r})=E_i\psi_i^{\sigma}(\mb{r}),
\label{HFS}
\end{equation}
%
where $n(\mb{r},\mb{r}')$ can be interpreted as the exchange charge density 
%
\begin{equation}
n(\mb{r},\mb{r}')= e\sum_{j,\sigma}\frac{\psi_j^{\sigma*}(\mb{r}')\psi_i^{\sigma}(\mb{r}')\psi_j^{\sigma}(\mb{r})\psi_i^{\sigma*}(\mb{r})}{\psi_i^{\sigma*}(\mb{r})\psi_i^{\sigma}(\mb{r})},
\label{EC}
\end{equation}
%
which is a function of two positions $\mb{r}$ and $\mb{r}'$ and depends on the quantum state $i$. The total exchange charge
equals the charge of a single electron, which can be easily shown by integrating over $\mb{r}'$ and using the orthogonality 
of wave functions
%
\begin{equation}
q=e\sum_{j,\sigma}[\int d\mb{r}'\psi_j^{\sigma*}(\mb{r}')\psi_i^{\sigma}(\mb{r}')]\frac{\psi_j^{\sigma}(\mb{r})}{\psi_i^{\sigma}(\mb{r})}=
e\sum_{j,\sigma}\delta_{ij}\frac{\psi_j^{\sigma}(\mb{r})}{\psi_i^{\sigma}(\mb{r})}=e.
\end{equation}
%
% **[REVIEW NEEDED]**: Self-interaction cancellation and Koopman's theorem
Taking into account the case when both positions are the same $\bm{r}=\bm{r'}$, we obtain the exchange charge density consistent with formula (\ref{dens}).
This causes the term describing the interaction of an electron with its own exchange field to have the same form as the Hartree potential, but with the opposite sign.
Due to this property, both terms cancel each other out, and in the Hartree-Fock approximation there is no problem of an electron interacting with its own field ({\it self-interaction}). 
In the form (\ref{HFS}), the Hartree-Fock equation has the form of a one-electron Schr\"{o}dinger equation with an exchange potential generated
at the location of the electron by the exchange charge.
According to Koopman's theorem, the eigenvalues $E_i$ correspond to the energies required to remove an electron from the orbital 
$\psi_i^{\sigma}(\mb{r})$~\cite{Koopman}.


\section{Self-consistent field method}

The Hartree-Fock equations are most often solved numerically using the self-consistent field method, also called the mean-field approximation.
For fixed atomic positions, we choose initial wave functions $\psi_{i0}^{\sigma}$, 
which can be, for example, orbitals of isolated atoms.
For these wave functions, we calculate the electron density $n(\mb{r})$ and potential $V(\mb{r})$ 
at each point in space of the studied system. 
For the potential calculated in this way, we solve equation (\ref{HF}) determining the set of one-electron wave functions and energies 
of electronic states. We use the determined wave functions to recalculate the electron density
and effective potential. We repeat the procedure until the energies and wave functions obtained in successive steps
are the same or differ by a small specified amount. Most often, the parameter used to test convergence
of the calculation is the total energy of the system. The self-consistent field method was first applied
to solve the Hartree equation, which we can obtain from the Hartree-Fock equation by removing
the exchange interaction. The Hartree equation corresponds to a wave function in the form of
a product of independent single-particle functions, i.e., it describes a set of independent electrons
in an effective electrostatic field. The self-consistent field method is currently used in most 
computational methods for the electronic structure of solids. 

\section{Electronic correlations}

% **[REVIEW NEEDED]**: Correlation energy definition and strongly correlated systems
The Hartree-Fock approach is an approximate method, so the calculated energy
and wave functions differ from exact quantities, which correspond to solutions of the Schr\"{o}dinger equation
for the many-electron wave function.
Taking into account all effects of many-body interactions in the electronic system enables
an additional lowering of the total energy compared to the Hartree-Fock energy.
The difference between the exact energy value ($E_{\rm{exact}}$) and the energy calculated in the 
Hartree-Fock approximation ($E_{\rm{HF}}$) is called the correlation energy
%
\begin{equation}
E_c=E_{\rm{\text{exact}}}-E_{\rm{\text{HF}}}.
\end{equation}
%
The source of electronic correlations are Coulomb interactions, which tend to spatially separate 
electrons. An electron located at position $\bm{r}$ causes other electrons to avoid this position.
That is, the probability of finding an electron at a given point depends on the positions of all the remaining $N-1$
electrons. Materials in which local Coulomb interactions have a decisive influence on the electronic structure
and transport properties are called strongly correlated systems. 

% **[REVIEW NEEDED]**: Configuration interaction and multi-determinant methods
The Hartree-Fock method is the starting point for more advanced methods used mainly in quantum chemistry.
One can generally write the wave function of a system of $N$ electrons as an expansion in a finite number of Slater determinants
%
\begin{equation}
\Phi(\bm{r}_1,...,\bm{r}_N)=c_0\Phi_0(\bm{r}_1,...,\bm{r}_N)+\sum_{i=1}^{N_\text{det}} c_i\Phi_i(\bm{r}_1,...,\bm{r}_N),
\end{equation}
%
where $\Phi_0$ is the ground state wave function in the Hartree-Fock approximation,
and the determinants $\Phi_i$ correspond to excited states. The larger the number of these determinants $N_{\text{det}}$, the more accurately
the electronic correlations are described. The functions $\Phi_i$ are constructed using single-electron orbitals that are not occupied
in the ground state and can be occupied by excited electrons. The determinants can describe single transitions of electrons to unoccupied orbitals ($\Phi_i^{\text{S}}$),
double electron excitations ($\Phi_i^{\text{D}}$), triple excitations ($\Phi_i^{\text{T}}$), and so on.
The total number of all possible excited configurations grows very rapidly with the number of electrons $N$ and orbitals $M$ (occupied and empty)
according to the formula 
%
\begin{equation}
N_{\text{det}}=\frac{(M+1)!}{N!(M+1-N)!}
\end{equation}
%
In the simplest approach called the configuration interaction (CI) method \cite{GTO},
the determinants are constructed from occupied and empty single-electron orbitals of the Hartree-Fock Hamiltonian.
The wave function takes the form of a series of determinants with an increasing number of excited electrons
%
\begin{equation}
\Phi_{\text{CI}}=c_0\Phi_0+\sum_{i=1}^{N_\text{S}} c^\text{S}_i \Phi_i^{\text{S}}+\sum_{i=1}^{N_\text{D}} c^\text{D}_i \Phi_i^{\text{D}}+\sum_{i=1}^{N_\text{T}} c^\text{T}_i \Phi_i^{\text{T}}+...,
\label{phiCT}
\end{equation}
% 
where $N_\text{S}$, $N_\text{D}$, $N_\text{T}$ determine the number of determinants for a given number of excited orbitals.
The ground state is found by determining the coefficients $c_i$ that minimize the energy
%
\begin{equation}
E_\text{CI}=\int \Phi^*(\bm{r}_1,...,\bm{r}_N)H\Phi(\bm{r}_1,...,\bm{r}_N)d\bm{r}_1d\bm{r}_2...d\bm{r}_N,
\end{equation}
%
where $H$ is the Hamiltonian of the electronic system (\ref{Sch-el}).
The minimization is carried out under the condition of wave function normalization, which translates into a condition for all expansion coefficients
$\sum_ic_i^2=1$.
In the limit of an infinite number of determinants, the wave function obtained in this way corresponds to the exact many-electron wave function.
In practice, expansions are used up to a few largest terms of the series (\ref{phiCT}), which
include double, triple, or quadruple excitations.  
The next step is to enable optimization of single-electron orbitals as well, in combination with
optimization of expansion coefficients within the multi-configuration self-consistent field (MCSCF) method. 
Multi-determinant methods allow very accurate determination of wave functions and energies of electronic states, 
however, the computational time scales exponentially with the system size,
so it is used only for small molecular systems.

\section{Electron gas}

% **[REVIEW NEEDED]**: Free electron gas model and Fermi energy
A homogeneous electron gas is a good approximation of the actual electronic structure
of simple metals and is often used in many fundamental
computational methods (e.g., in the local density approximation).
The simplest model of an electron gas consists of non-interacting particles
confined in a cube of volume $V=L^3$. The length of the electronic de Broglie wave
propagating in each direction must satisfy the periodicity condition ($n\lambda=L$),
which leads to quantization of the wave vector in the $x$, $y$, and $z$ directions
%
\begin{equation}
\mb{k}=(\frac{2\pi n_x}{L},\frac{2\pi n_y}{L},\frac{2\pi n_z}{L}),
\end{equation}
%
where $n_x$, $n_y$, $n_z$ are integers.
The energies and wave functions of discrete electronic states are given by the formulas
%
\begin{equation}
E_k=\frac{\hbar^2 k^2}{2m},
\end{equation}
%
\begin{equation}
\psi_k(\mb{r})=\frac{1}{\sqrt{V}}e^{i\mb{kr}},
\label{pw}
\end{equation}
%
We assume that the system is not magnetically polarized and the number of electrons with spins pointing 
up and down is equal, $N_{\uparrow}=N_{\downarrow}=\frac{1}{2}N$.
According to the Pauli exclusion principle, in a quantum state with a given wave vector $\mb{k}$, there can be
a maximum of two electrons with opposite spins.
At temperature $T=0$ K, electrons occupy consecutive states from the lowest to
the maximum value $E_F$ called the Fermi energy. 
The total energy of a system of $N$ electrons is
%
\begin{equation}
E=2\sum_{k<k_F} E_k = \frac{2V}{(2\pi)^3}\int d\bm{k} \frac{\hbar^2 k^2}{2m} = \frac{V}{(2\pi)^3}\frac{4\pi\hbar^2}{5m} k_F^5, 
\label{E}
\end{equation}
%
where $(2\pi)^3/V$ is the volume in reciprocal space per one electronic state, 
and $k_F$ is the Fermi wave vector, which is the radius of a sphere in reciprocal space
containing the occupied electronic states. In the general case, the surface encompassing occupied electronic states,
called the Fermi surface, can have any shape and consist of many separate parts.
The number of electrons in volume $V$ is expressed by the formula
%
\begin{equation}
N=\frac{2V}{(2\pi)^3}\int d\bm{k}=\frac{2V}{(2\pi)^3}\frac{4\pi}{3} k_F^3.  
\label{N}
\end{equation}
%
Dividing (\ref{E}) by (\ref{N}), we obtain the average energy per
single electron, expressed through the Fermi energy
%
\begin{equation}
\frac{E}{N}=\frac{3}{5} \frac{\hbar^2 k_F^2}{2m} = \frac{3}{5} E_F.
\label{EN}
\end{equation}  
%
In the case of a homogeneous gas, the electron density is given by the expression
%
\begin{equation}
n = \frac{N}{V} = \frac{k_F^3}{3\pi^2}.
\label{nkf}
\end{equation}
%
A commonly used quantity is the volume of a sphere in real space per one electron 
%
\begin{equation}
V_s=\frac{4\pi}{3}r_{s}^3=\frac{1}{n},
\end{equation}
%
where $r_s$ is its radius
%
\begin{equation}
r_s=(\frac{3}{4\pi n})^{1/3}.
\end{equation}
%
% **[REVIEW NEEDED]**: Hartree-Fock treatment of electron gas and exchange energy derivation
We will now calculate the energy of the electron gas in the Hartree-Fock approximation.
We assume that the positive charge is distributed uniformly throughout the entire space $V$
and has the same density as the electron gas. The interaction energy
of electrons with such a field cancels with the Hartree energy, and the total energy
of an electron consists only of the kinetic part and the exchange interaction.
The exchange energy can be easily calculated by representing the Coulomb potential
in the form of a Fourier transform
%
\begin{equation}
\frac{1}{|\mb{r}-\mb{r}'|}=4\pi\int \frac{d\mb{q}}{(2\pi)^3}\frac{e^{i\mb{q}(\mb{r}-\mb{r}')}}{q^2},
\end{equation}
%
Inserting the wave function in the form (\ref{pw}) into the expression for the exchange energy 
and performing integration, we obtain the energies of single-electron states
%
\begin{equation}
E_k=\frac{\hbar^2 k^2}{2m} - \int_{\mb{k'}<\mb{k_F}} \frac{d\bm{k'}}{(2\pi)^3} \frac{4\pi}{|\mb{k}-\mb{k'}|^2}
=\frac{\hbar^2 k^2}{2m} - \frac{k_F}{\pi}(1+\frac{k_F^2-k^2}{2kk_F}ln|\frac{k_F+k}{k_F-k}|).
\end{equation}
%
The total energy of the electron system is obtained by summing this expression over the wave vector    
%
\begin{equation}
E=\sum_{k<k_F}[\frac{\hbar^2 k^2}{2m} - \frac{k_F}{\pi}(1+\frac{k_F^2-k^2}{2kk_F}ln|\frac{k_F+k}{k_F-k}|)].
\end{equation}
%
Using (\ref{EN}) and converting the summation in the second term to an integral, we obtain
%
\begin{equation}
E=N(\frac{3}{5}E_F - \frac{3k_F}{4\pi})=N(\frac{3}{5}E_F - \frac{3(3\pi^2n)^{\frac{1}{3}}}{4\pi}),
\end{equation}
where we used the relation (\ref{nkf}). The exchange energy per electron is 
%
\begin{equation}
\varepsilon_x=\frac{E_x}{N}=-\frac{3}{4}\Big(\frac{3}{\pi}\Big)^{\frac{1}{3}}n^\frac{1}{3}.
\label{exchange}
\end{equation}
%
% **[REVIEW NEEDED]**: Exchange energy formula and local density approximation
This formula determining the dependence of exchange energy on electron density
is used in density functional theory within the local density approximation.

For an electron gas, one can derive an expression describing the distribution of exchange hole charge (\ref{EC}),
which depends only on distance $r$
%
\begin{equation}
g^{\sigma}_x(r)=1 - [3\frac{sin(rk_F)-rk_Fcos(rk_F)}{(rk_F)^3}]^2.
\end{equation}
%
% **[REVIEW NEEDED]**: Correlation energy approximations - Wigner, Gell-Mann-Brueckner, and QMC
It is not possible to express the correlation energy for a free electron gas in the form of 
an analytical function of density $n$. The first approximate formula was proposed by Wigner \cite{wigner34}
%
\begin{equation}
\varepsilon(r_s)=-\frac{0.44}{r_s+7.8}.
\end{equation}
%
In the limit of low densities ($r_s\rightarrow \infty$), electrons form the so-called Wigner crystal, whose correlation energy 
coincides with the electrostatic energy of localized point charges.
In the limit of high density ($r_s\rightarrow 0$), Gell-Mann and Brueckner
derived a formula for the correlation energy density of a magnetically unpolarized
electron gas \cite{GB}
%
\begin{equation}
\varepsilon_c (r_s)=0.311\ln(r_s)+r_s(A\ln(r_s)+C)-0.048+... .
\end{equation}
%
The correlation energy of the electron gas was most accurately determined numerically
by the quantum Monte Carlo (QMC) method \cite{CeperleyAlder80}. 


\chapter{Stany elektronowe w krysztale}

\section{Sieć krystaliczna}

Kryształy zbudowane są z atomów lub molekuł ułożonych w periodyczną sieć.
Trójwymiarową sieć krystaliczną definiujemy jako zbiór punktów, których
położenia określone są wektorem translacji
%
\begin{equation}
\bm{R}_n=n_1\bm{a}_1+n_2\bm{a}_2+n_3\bm{a}_3,
\label{trans}
\end{equation}
%
gdzie wektory $\bm{a}_i$ nazywamy podstawowymi lub prymitywnymi wektorami translacji, a $n_i$ są liczbami całkowitymi.
Zbiór punktów zdefiniowanych tym wzorem nazywamy siecią Bravais. Podobnie można zdefiniować sieć Bravais
dla dowolnego wymiaru $d$ wybierając odpowiednią ilość wektorów bazowych: $\bm{a}_1,\bm{a}_2,...,\bm{a}_d$. 
Z każdym punktem sieci Bravais można związać zbiór atomów zwanych bazą. Najmniejsza przestrzeń kryształu,
która po przesunieciu o wszystkie translacje sieciowe wypełni całą przestrzeń nazywamy komórką prymitywną. 
Komórka prymitywna zawiera dokładnie jeden węzeł sieci Bravais i zwykle zbudowana jest na podstawowych wektorach translacji sieci.
Istnieje specjalny rodzaj komórki prymitywnej nazywany komórką Wignera-Seitza, która jest przestrzenią zawierającą punkty przestrzeni, 
które znajdują się bliżej danego punktu sieci niż pozostałych. 
Często zamiast komórki prymitywnej używa się komórki elementarnej, która zbudowana jest na innych wektorach bazowych
i jej symetria jest taka sama jak symetria całego kryształu.
Objętość komórki elementarnej jest całkowitą wielokrotnością objętości komórki prymitywnej.
W strukturach prostych (bez centrowania powierzchniowego lub przestrzennego), komórka elementarna pokrywa się
z komórką prymitywną. Różnice między poszczególnymi rodzajami komórek można przedstawić na przykładzie
sieci dwuwymiarowej. Na rysunku \ref{fig:bravais} pokazana jest struktura centrowana prostokątna z zaznaczonymi dwoma wektorami podstawowaymi $\bm{a}_p$
i $\bm{b}_p$, które definiują komórkę prymitywną, pokazaną na rysunku w dwóch wariantach. Komórka Wignera-Seitza pokazana
jest po prawej stronie. Komórka elementarna wyznaczona jest przez dwa wektory bazowe $\bm{a}_e$ i $\bm{b}_e$.
W dwóch wymiarach mamy w sumie pięć rodzajów sieci Bravais: skośna, prostokątna, prostokatna centrowana, heksagonalna i kwadratowa.

\begin{figure}[h]
\centering
\includegraphics[scale=0.4]{lattice.pdf}
\caption{Dwuwymiarowa sieć periodyczna.}\label{fig:bravais}
\end{figure}

W przypadku sieci trójwymiarowych, wszystkie możliwe struktury można podzielić na siedem układów krystalograficznych: 
regularny, tetragonalny, rombowy, heksagonalny, trygonalny, jednoskośny i trójskośny, którym odpowiada czternaście możliwych sieci Bravais,
które podzielone są na proste (P), centrowane powierzchniowo (F), centrowane na podstawach (C) i centrowane przestrzennie (I). 
Wszystkie układy krystalograficzne z odpowiadającymi im sieciami Bravais przedstawione są w tabeli \ref{lattice}.
Dla każdego układu pokazana jest zalezność między stałymi sieci ($a$, $b$, $c$) oraz kątami między kierunkami ($\alpha$, $\beta$, $\gamma$),
które definiują krawędzie komórek elementarnych. 

\begin{table}[h!]
\caption{Układy krystalograficzne.}\label{lattice}
\begin{center}
\begin{tabular}{|c|c|c|c|}
\hline
Układ & Stałe sieci & Kąty & Sieci Bravais\\
\hline
regularny & $a=b=c$ & $\alpha=\beta=\gamma=90^{\degree}$ &  P, F, I \\ 
tetragonaly & $a=b\ne c$ & $\alpha=\beta=\gamma=90^{\degree}$ &  P, I \\  
rombowy & $a\ne b\ne c$ & $\alpha=\beta=\gamma=90^{\degree}$ &  P, F, C, I \\ 
heksagonalny &  $a=b\ne c$ & $\alpha=\beta=90^{\degree}$, $\gamma=120^{\degree}$ &  P  \\
trygonalny & $a=b\ne c$ & $\alpha=\beta=90^{\degree}$, $\gamma=120^{\degree}$ &  P \\
jednoskośny & $a=b\ne c$ & $\alpha=\gamma=90^{\degree}$, $\beta\ne 90^{\degree}$ & P, C \\ 
trójskośny & $a\ne b\ne c$ & $\alpha\ne \beta\ne \gamma\ne 90^{\degree}$ & P \\ \hline
\end{tabular}
\end{center}
\end{table}

Zbiór operacji, który pozostawia dany kryształ niezmienionym tworzy grupą przestrzenną.
Składa się ona z translacji sieci krystalicznej opisanych wzorem (\ref{trans}) oraz symetrii punktowych: inwersji, obrotów i odbić.
Dodatkowo występują operacje niesymorficzne, które są połączeniem operacji punktowych i ułamkowych translacji.
Wszystkich grup przestrzennych jest 230, z których 73 to symorficzne i 157 niesymorficzne.

\section{Przestrzeń odwrotna}

Strukturę elektronową materiału analizuje się najczęściej wykorzystując przestrzeń wektorów falowych $\bm{k}$,
którą nazywamy przestrzenią odwrotną. Każdej sieci Bravais w przestrzeni rzeczywistej odpowiada sieć punktów
w przestrzeni odwrotnej zdefiniowana wektorami
%
\begin{equation}
\bm{G}_m=m_1\bm{b}_1+m_2\bm{b}_2+m_3\bm{b}_3,
\end{equation}
%
gdzie $m_j$ są liczbami całkowitymi, a $\bm{b}_j$ są wektorami odwrotnymi do wektorów bazowych $\bm{a}_i$,
czyli spełniają warunek
%
\begin{equation}
\bm{a}_i \cdot \bm{b}_j = 2\pi \delta_{ij}.
\label{orto}
\end{equation}
%
Wektory sieci odwrotnej można wyznaczyć z następujących zależności
%
\begin{eqnarray}
\bm{b}_1=\frac{2\pi}{\Omega}\bm{a}_2\times \bm{a}_3, \\
\bm{b}_2=\frac{2\pi}{\Omega}\bm{a}_3\times \bm{a}_1, \\
\bm{b}_3=\frac{2\pi}{\Omega}\bm{a}_1\times \bm{a}_2, 
\end{eqnarray}
%
gdzie $\Omega=\bm{a}_1\cdot(\bm{a}_2\times \bm{a}_3)$ jest objętością komórki prymitywnej.

W przestrzeni odwrotnej również definiujemy komórkę prymitywną, która najczęściej ma kształt
komórki Wignera-Seitza i nosi nazwę strefy Brillouina ({ang. Brillouin zone} - BZ).  
Komórka, która obejmuje punkt $\bm{K}=0$, nazywany punktem $\Gamma$, nosi nazwę pierwszej strefy Brillouina.
Na rysunku \ref{fig:lattice} pokazane są przykładowe strefy Brillouina dla trzech sieci regularnych. 

\begin{figure}[h]
\centering
\includegraphics[scale=1.2]{brillouin.pdf}
\caption{Strefy Brillouina dla sieci regularnych: prostej (P), centrowanej objetościowo (I) i centrowanej przestrzennie (F).}
\label{fig:lattice}
\end{figure}


\section{Twierdzenie Blocha}

W najprostszym opisie elektronów w periodycznym potencjale kryształu $V(\bm{r})$, stany elektronowe opisywane są przez niezależne jednocząstkowe funkcje falowe, bedące rozwiązaniami równania Schr\"{o}dingera w postaci
%
\begin{equation}
[-\frac{\hbar^2\nabla^2}{2m}+V(\bm{r})]\psi^{\sigma}_{\bm{k}j}(\bm{r})=\varepsilon_{j\sigma}(\bm{k})\psi^{\sigma}_{\bm{k}j}(\bm{r}),
\label{RS}
\end{equation}
%
gdzie $\bm{k}$ jest wektorem falowym, $\sigma$ określa spin elektronu i indeks $j$ numeruje stany własne hamiltonianu odpowiadajace energiom $\varepsilon_{j\sigma}(\bm{k})$.
Zgodnie z twierdzeniem Blocha, funkcja falowa elektronów w periodycznym potencjale ma postać
%
\begin{equation}
\psi^{\sigma}_{\bm{k}j}(\bm{r})=e^{i\bm{k}\bm{r}}u^{\sigma}_{\bm{k}j}(\bm{r}),
\end{equation}
%
gdzie $u^{\sigma}_{\bm{k}j}(\bm{r})$ jest funkcją periodyczną spełniającą warunek $u^{\sigma}_{\bm{k}j}(\bm{r})=u^{\sigma}_{\bm{k}j}(\bm{r}+\bm{R}_n)$ dla
każdego wektora traslacji $\bm{R}_n$. Można pokazać, że funkcje Blocha są wektorami własnymi
operatora translacji
%
\begin{equation}
\hat{T}_n\psi^{\sigma}_{\bm{k}j}(\bm{r})=\psi^{\sigma}_{\bm{k}j}(\bm{r}+\bm{R}_n)=e^{i\bm{k}(\bm{r}+\bm{R}_n)}u^{\sigma}_{\bm{k}j}(\bm{r}+\bm{R}_n)=e^{i\bm{k}\bm{R}_n}\psi^{\sigma}_{\bm{k}j}(\bm{r}).
\end{equation}
%
Hamiltonian jest niezmienniczy ze względu na działanie operatora translacji,
co oznacza, że obydwa operatory ze sobą komutują. Funkcje Blocha są zatem również stanami własnymi hamiltonianu,
co dowodzi prawdziwości twierdzenia Blocha. 

Funkcja Blocha określona jest dla wektora falowego $\bm{k}$, który w układach periodycznych jest dobrze zdefiniowaną liczbą kwantową.
Wektor falowy związany jest z kwazipędem elektronu w stanie $\psi_{\bm{k}j}$
%
\begin{equation}
\bm{p}=\hbar\bm{k}.
\end{equation}
%
Ponieważ wszystkie własności struktury elektronowej są translacyjnie niezmiennicze 
ze względu na przesunięcie o dowolny wektor sieci odwrotnej, również kwazipęd nie zależy od takiego przesunięcia.

\section{Fale płaskie}

Metody obliczeniowe struktury pasmowej różnią sie między sobą stosowanymi reprezentacjami funkcji falowej.
Ogólnie możemy zapisać funkcję falową w formie
%
\begin{equation}
\psi^{\sigma}_{\bm{k}j}(\bm{r})=\sum_mc^{\sigma}_{jm}(\bm{k})\phi_{\bm{k}m}(\bm{r}),
\end{equation}
% 
gdzie $\phi_{\bm{k}m}(\bm{r})$ są funkcjami bazowymi, a $c^{\sigma}_{jm}(\bm{k})$ to współczynniki rozwinięcia.
W krysztale, ze względu na periodyczny potencjał, naturalną bazę stanowią fale płaskie. Jeżeli cześć periodyczną funkcji
Blocha rozwiniemy w tej bazie, funkcja falowa przyjmuje postać
%
\begin{equation}
\psi^{\sigma}_{\bm{k}j}(\bm{r})=e^{i\bm{k}\bm{r}}\sum_{m} c^{\sigma}_{jm}(\bm{k})e^{i\bm{G_m}\bm{r}}=\sum_{m} c^{\sigma}_{jm}(\bm{k})e^{i(\bm{k}+\bm{G_m})\bm{r}},
\end{equation}
% 
gdzie $\bm{G_m}$ są wektorami sieci odwrotnej. Równanie Schr\"{o}dingera (\ref{RS}) przetransponowane do przestrzeni odwrotnej w bazie
fal płaskich przyjmuje postać
%
\begin{equation}
\sum_{n}H_{mn} c^{\sigma}_{jn}(\bm{k})=\varepsilon_{j\sigma}(\bm{k}) c^{\sigma}_{jm},
\label{ham_rec}
\end{equation}
%
gdzie elementy macierzowe Hamiltonianu dane są wyrażeniem
%
\begin{equation}
H_{mn}=\frac{\hbar^2}{2m}|\bm{k}+\bm{G_m}|^2\delta_{mm'}+V(\bm{G_m}-\bm{G_n}).
\end{equation}
%
Drugi wyraz jest transformatą Fouriera potencjału elektronowego
%
\begin{equation}
V(\bm{G})=\frac{1}{\Omega}\int_{\Omega}\bm{dr}V(\bm{r})e^{-i\bm{Gr}},
\end{equation}
%
gdzie całkowanie przebiega po komórce prymitywnej.

\section{Periodyczne warunki brzegowe}

Podobnie jak dla gazu elektronowego zamkniętego w pojemniku, również dla skończonego kryształu
możemy wprowadzić periodyczne warunki brzegowe (warunki Borna-Karmana). Załóżmy, że rozmiary kryształu
określone są przez liczbę komórek prymitywnych w każdym kierunku kryształu $(N_1,N_2,N_3)$.
Zatem liczby całkowite, które definiują wektory translacyjne sieci (\ref{trans})
zmieniają się w zakresie $n_i=0,1,2,...,N_i$, gdzie $i=1,2,3$.
Zakładając, że funkcja falowa na dwóch przeciwległych brzegach układu (w każdym z trzech kierunków) przyjmuje
takie same wartości i korzystając z twierdzenia Blocha otrzymujemy warunek
%
\begin{equation}
\psi^{\sigma}_{\bm{k}j}(\bm{r})=\psi^{\sigma}_{\bm{k}j}(\bm{r}+N_i\bm{a}_i)=e^{i\bm{k}N_i\bm{a}_i}\psi^{\sigma}_{\bm{k}j}(\bm{r}),
\label{pwb}
\end{equation} 
%
który spełniony jest gdy 
%
\begin{equation}
\bm{k}N_i\bm{a}_i=2\pi n_i.
\end{equation}
%
Wykorzystując bazę wektorów prymitywnych sieci odwrotnej ($\bm{b}_1,\bm{b}_2,\bm{b}_3$) oraz biorąc pod uwagę zależność (\ref{orto}), dostajemy zbiór dozwolonych wektorów falowych
%
\begin{equation}
\bm{k}=\frac{n_1}{N_1}\bm{b}_1+\frac{n_2}{N_2}\bm{b}_2+\frac{n_3}{N_3}\bm{b}_3.
\label{vectk}
\end{equation}
%
Wektory te określają funkcje falowe Blocha $\psi_{i,\bm{k}}$ i energie stanów $\varepsilon_i(\bm{k})$.
Skończona ilość dostępnych stanów wynika zatem z ograniczonych rozmiarów kryształu i jest równa ilości komórek prymitywnych w układzie: $N=N_1N_2N_3$.  
Ze wzoru (\ref{vectk}) wynika również, że dozwolone wektory falowe stanów elektronowych należą do pojedynczej komórki prymitywnej w przestrzeni odwrotnej.
Najczęściej do analizy stanów elektronowych wybiera się wektory należące do pierwszej strefy Brillouina.

\section{Sumowanie w przestrzeni odwrotnej}

Wiele wielkości fizycznych wyznaczanych jest jako sumy po punktach w przestrzeni odwrotnej. 
Przedstawiona powyżej analiza pokazuje, że wystarczy ograniczyć
sie do wektorów falowych należących do pierwszej strefy Brillouina.
Dodatkowo, można wykorzystać symetrię kryształu do zmniejszenia ilości punktów, które konieczne
są do wyznaczenia danej wielkości fizycznej. 
Oznaczmy operacje punktowe (symorficzne) przez $O_i$, a wektory translacji ułamkowej związane z daną symetrią $O_i$
przez $\bm{t}_i$. W przestrzeni odwrotnej rozpatrujemy jedynie symorficzne operacje symetrii ponieważ translacje
ułamkowe nie mają wpływu na przestrzeń odwrotną. Hamiltoniana opisujący kryształ jest niezmienniczy
względem operacji punktowych: $\bm{r}\rightarrow O_i\bm{r}+\bm{t}_i$ oraz $\bm{k}\rightarrow O_i\bm{k}$.
Oznacza to, że funkcje falowe otrzymane po tych transformacjach 
%
\begin{equation}
\psi^{\sigma}_{O_i\bm{k}j}(O_i\bm{r}+\bm{t}_i)=\psi^{\sigma}_{\bm{k}j}(\bm{r}),
\end{equation}  
%
są również funkcjami własnymi hamiltonianu z tą samą energią własną $\varepsilon_{j\sigma}(\bm{k})$.
Oznacza to, że można ograniczyć sumowanie danej wielkości fizycznej $f(\bm{k})$ do wektorów $\bm{k}$ należących do mniejszego obszaru,
który nazywany jest nieredukowalną częścią strefy Brillouina ({\it ang. irreducible Brillouin zone} - IBZ). 
Obszar ten tworzą wektory $\bm{k}$, które są nierównoważne, czyli nie można ich wzajemnie połączyć operacją symetrii punktowej. 
Wartości w pozostałych punktach dostajemy przez zastosowanie odpowiednich operacji symetrii $f(O_i\bm{k})=f(\bm{k})$.
Do sumowania wykorzystujemy wagi $w_{\bm{k}}$, które zdefiniowane są jako ilości wektorów $\bm{k}$, powiązanych symetrią punktową 
z wektorem należącym do IBZ (z uwzglednieniem tego wektora), podzielone przez całkowitą ilość wektorów $N_{\bm{k}}$.
Przykładowo, wartość średnią wielkości $f$, możemy wyliczyć sumując po punktach należących do IBZ
%
\begin{equation}
\bar{f}=\frac{1}{N_{\bm{k}}}\sum_{\bm{k}}^{BZ}f(\bm{k})=\sum_{\bm{k}}^{IBZ}w_{\bm{k}}f(\bm{k}).
\end{equation}
%

Przy sumowaniu po przestrzeni odwrotnej można w zasadzie wybrać dowolne punkty należące do IBZ.
Jednak wykorzystując własności funkcji periodycznych, można tak wybrać wektory $\bm{k}$,
aby zminimalizować błędy przy obliczaniu sum lub całek.
Dowolną funkcję periodyczną można rozwinąć przy pomocy transformaty Fouriera
%
\begin{equation}
f(\bm{k})=\sum_n f(\bm{R}_n)e^{i\bm{k}\bm{R}_n},
\end{equation} 
%
gdzie $\bm{R}_n$ są wektorami translacji sieci krystalicznej.
Zbiór punktów, który jest optymalny do wyliczania sum w przestrzeni odwrotnej, 
nazywany jest siecią Monkhorsta-Packa \cite{MP}. Wyznacza się go ze  wzoru
%
\begin{equation}
\bm{k}(n_1,n_2,n_3)=\sum_i^3 \frac{2n_i-N_i-1}{2N_i}\bm{b}_i,
\end{equation}
%
gdzie liczby $N_1$, $N_2$ i $N_3$ określają ilość punktów $\bm{k}$ w każdym kierunku: $n_i=1,2,...,N_i$.
Tak zdefiniowane punkty tworzą jednorodną sieć w przestrzeni odwrotnej i charakteryzują się
tym, że suma wartości periodycznej funkcji, która posiada komponenty Fouriera ograniczone w każdym kierunku do $\bm{R}_n=N_i\bm{a}_i$,
równa jest dokładnej całce z tej funkcji.  

\section{Energia Fermiego}

Wielkości fizyczne, które dotyczą stanu podstawowego, wyliczane są zwykle dla obsadzonych stanów elektronowych.
Czyli sumowanie w przestrzeni odwrotnej uwzglednia tylko zajęte stany.
Energia najwyższego obsadzonego stanu określa energię Fermiego ($E_F$).
W metalach pasma elektronowe przecinają $E_F$ i przy sumowaniu po dyskretnych punktach $\bm{k}$ pojawiają się nieciągłości
w obsadzeniach stanów. Powoduje to wolniejszą zbieżność ze względu na ilość punktów $\bm{k}$. 
Można łatwo rozwiązać ten problem przez wprowadznie ciagłej funkcji obsadzania stanów w pobliżu $E_F$.
Naturalnym wyborem jest zastosowanie funkcji rozkładu Fermiego-Diraca dla skończonej temperatury
%
\begin{equation}
F(\varepsilon,E_F,T)=[\exp(\frac{\varepsilon-E_F}{kT})+1]^{-1}.
\end{equation}   
%
Dla $T=0$, dostajemy funkcję schodkową, która odpowiada rozkładowi w stanie podstawowym.
Aby wyznaczyć $E_F$ dla dowolnej temperatury, korzystamy z warunku
%
\begin{equation}
\sum_{\bm{k},j} w_{\bm{k}}F(\varepsilon,E_F,T)=n_{el},
\end{equation}   
%
gdzie $n_{el}$ jest liczbą elektronów, a sumowanie wykonywane jest po wszystkich obsadzonych stanach
uwzględniając polaryzację spinową.



\section{Struktura pasmowa}

Zbiór wszystkich stanów elektronowych $\varepsilon_{\bm{k}j}^{\sigma}$ otrzymanych w wyniku diagonalizacji hamiltonianiu 
dla wektorów falowych $\bm{k}$ tworzy strukturę pasmową danego materiału.
Dostępne energie stanów elektronowych, które odpowiadają funkcji własnej z indeksem $j$, grupują się w przedziały nazywane pasmami energetycznymi. 
Rozwiązania równania (\ref{ham_rec}) w funkcji wektora falowego $\bm{k}$ są periodyczne ze względu na przesunięcie o 
wektor sieci odwrotnej
%
\begin{equation}
\varepsilon_{\bm{k}j}^{\sigma}=\varepsilon_{\bm{k}+\bm{G_m}j}^{\sigma}.
\end{equation}   
%
Zatem przy prezentacji struktury pasmowej można ograniczyć się do pierwszej strefy Brillouina. Zwykle prezentuje się tylko stany elektronowe dla wektorów $\bm{k}$ należących do wybranych kierunków i punktów wysokiej symetrii w przestrzeni odwrotnej. Jeżeli energie pasm przyjmują różne wartości dla dwóch kierunków spinu,
$\varepsilon_{\bm{k}j}^{\uparrow}\ne \varepsilon_{\bm{k}j}^{\downarrow}$, mamy doczynienia z rozszczepieniem spinowym (wymiennym), które związane jest z uporządkowaniem magnetycznym.

Ważną wielkością, która charakteryzuje strukturą pasmową jest gęstość stanów elektronowych dla danego kierunku spinu $\rho_{\sigma}(E)$,
która wyliczana jest jako suma po pierwszej strefie Brillouina
%
\begin{equation}
\rho_{\sigma}(E)=\frac{1}{N_k}\sum_{j,\bm{k}}\delta[\varepsilon_{\bm{k}j}^{\sigma}-E],
\end{equation}
%
gdzie $N_k$ jest ilością wektorów $\bm{k}$ w pierwszej strefie Brillouina.
Całkowita gęstość stanów równa jest sumie obydwóch składowych spinowych, $\rho(E)=\rho_{\uparrow}(E)+\rho_{\downarrow}(E)$.
Przy wyliczaniu gestości stanów często zamiast sieci Monkhorsta-Packa stosuje się metodę tetraedrów.
Polega ona na podzieleniu całej strefy Brillouina na tetraedry i wyliczeniu energii stanów w punktach $\bm{k}$, które
odpowiadają wierzchołkom tetraedrów. Energie pomiędzy tymi punktami wyznacza się stosując interpolację liniową.

Struktura pasmowa decyduje o podstawowych własnościach danego materiału i jest jego wyróżnikiem, pozwalającym zakwalifikować go do określonej grupy materiałów. 
Trzy podstawowe typy materiałów krystalicznych to metale, półprzewodniki i izolatory.
Struktura pasmowa metali charakteryzuję się występowaniem powierzchni Fermiego, która odgranicza zajęte stany elektronowe od stanów pustych.
Zatem w typowych metalach gęstość stanów na poziomie Fermiego jest niezerowa, $\rho(E_F)>0$.
Tuż powyżej energii Fermiego ($E_F$) znajdują się stany, do których przechodzą elektrony wzbudzone termicznie lub w wyniku oddziaływania z innymi
elektronami. W półprzewodnikach i izolatorach, powyżej ostatniego zajętego stanu elektronowego, znajduje się przerwa obejmująca
zakres zabronionych energetycznie stanów.
Pasmo, które znajduje się tuż poniżej przerwy energetycznej nazywa się pasmem walencyjnym, a to które jest powyżej
pasmem przewodnictwa.
Półprzewodniki zwykle charakteryzują się mniejszą przerwa energetyczną ($\sim$1-2 eV) niż izolatory.
Występują też materiały nazywane półmetalami, w których przerwa energetyczna i $\rho(E_F)$ są równe zeru (np. grafen).
Odrębną grupę materiałów stanowią izolatory Motta, w których przerwa energetyczna jest wynikiem silnych oddziaływań
elektronowych.  



\chapter{Teoria funkcjonału gęstości}

\section{Twierdzenia Hohenberga-Kohna}

Większość nowoczesnych metod obliczeniowych stosowanych do badania własności materiałów opiera się na teorii funkcjonału gestości ({\it ang. density functional theory} - DFT), która została sformułowana w pracach Hohenberga i Kohna \cite{kohn64}
oraz Kohna i Shama \cite{kohn65}. Obecnie, DFT stosowana jest powszechnie do wyliczania struktury elektronowej,
optymalizacji sieci krystalicznej, badania własności elastycznych i dynamiki sieci oraz wielu innych własności materiałowych.
Jest praktycznie jedyną metodą umożliwiającą obliczenia kwantowo-mechaniczne dla układów zawierających setki,
a nawet tysiące atomów.  
Podobnie jak w opisie oddziaływania wymiennego zaproponowanym przez Slatera~\cite{slater51}
i w teorii Thomasa-Fermiego~\cite{thomas,fermi1927,dirac} podstawową wielkością jest tutaj gęstość elektronowa zdefiniowana w każdym punkcie materiału $n(\bm{r})$.
Główną ideą tego podejścia jest możliwość zastąpienie dokładnej funkcji falowej, układem stanów jednocząstkowych w efektywnym potencjale elektronowym, który
daje taką samą gęstość elektronową jak opis wieloelektronowy.
Takie podejście umożliwia wykorzystania dokładnej informacji o oddziaływaniach wymiennych i korelacyjnych
otrzymanych dla jednorodnego gazu elektronowego do badania niejednorodnych układów atomowych.

Praca Hohenberga i Kohna \cite{kohn64} zawiera dwa fundamentalne twierdzenia, które
dotyczą relacji między gęstością elektronową $n(\bm{r})$, zewnętrznym potencjałem $V_{ext}(\bm{r})$ 
oraz energią całkowitą $E[n]$, która jest funkcjonałem gęstości elektronowej. Zależność funkcjonalna oznacza, że dla dowolnego rozkładu 
elektronów w całej przestrzeni $n(\bm{r})$ można jednoznacznie wyznaczyć energię całkowitą 
układu $E[n]$. Zapiszmy hamiltonian układu oddziałujących elektronów
w ogólnej formie

\begin{equation}
H=-\frac{\hbar^2}{2m}\sum_{i}\nabla_i^2+V_\text{ext}(\bm{r})+\frac{1}{2}\sum_{i\neq j}\frac{e^2}{|\bm{r}_i-\bm{r}_j|}.
\label{hamil}
\end{equation}
%
Twierdzenia Hohenberga-Kohna można sformułować następująco: 

{\bf Twierdzenie I:} 
Zewnętrzny potencjał układu oddziałujących elektronów $V_\text{ext}$ jest jednoznacznie określony (z dokładnością do stałej wartości) 
przez gęstość elektronową w stanie podstawowym $n_0(\mb{r})$.

{\bf Twierdzenie II:}
Dla ustalonego potencjału zewnętrznego $V_\text{ext}(\bm{r})$, funkcjonał energii $E[n]$ osiąga
minimalną wartość $E_0$ dla gęstości elektronowej w stanie podstawowym $n_0(\bm{r})$.

Z twierdzeń tych wynikają następujace wnioski. Ponieważ hamiltonian jest jednoznacznie określony (z dokładnościa do stałej)
przez $n_0(\bm{r})$, funkcje falowe wszystkich stanów elektronowych,
jak również wszystkie własności układu są całkowicie zdeterminowane przez gęstość elektronową w stanie podstawowym.
Znajomość funkcjonału energii $E[n]$ jest wystarczająca do wyznaczenia energii i gęstości elektronowej
stanu podstawowego. 

\section{Równanie Kohna-Shama}

Twierdzenia Kohna-Shama pozwalają w sposób ścisły powiązać gestość elektronową w stanie podstawowym
z równaniem Schr\"{o}dingera dla układu oddziałujacych elektronów, opisanego hamiltonianem (\ref{hamil}).
W praktyce nie jest możliwe bezpośrednie rozwiązanie równania Schr\"{o}dingera i wyznaczenie wielociałowych funkcji falowych.
Teoria funkcjonału gęstości pozwala zastąpić równanie wielociałowe nowym równaniem, nazywanym równaniem Kohna-Shama, które ma
taką postać jak dla nieoddziałujących cząstek znajdujacych się w efektywnym polu $V_{eff}$.
Znając jednoelektronowe rozwiązania równania Kohna-Shama $\psi_i^{\sigma}(\bm{r})$, zależne od położenia i spinu, 
możemy wyznaczyć gęstość elektronową w każdym punkcie
%
\begin{equation}
n(\bm{r})=\sum_{i,\sigma} f_{i\sigma}|\psi_i^{\sigma}(\bm{r})|^2,
\label{density}
\end{equation}
%
gdzie dla uproszczenia indeks $i$ określa zarówno numer stanu $j$, jak i wektor falowy $\bm{k}$, a $f_{i\sigma}$ są liczbami obsadzeń
stanów, które w ogólnym przypadku mogą przyjmować ułamkowe wartości, np. zgodnie z rozkładem Fermiego-Diraca.
Przyjmujemy również, że ładunek elementarny $e=1$, co oznacza, że gęstość ładunku jest tożsama z gęstością elektronową. 
Twierdzenia Kohna-Shama zapewniają nam, że efektywny potencjał $V_{eff}$ jest jednoznacznie
określony przez gęstość elektronową, jak również, że funkcjonał energii całkowitej osiąga
minimum dla gęstości elektronowej w stanie podstawowym.
Zatem stan podstawowy układu możemy wyznaczyć jeżeli znamy funkcjonał energii i potrafimy
znaleźć jego minimum. Ogólnie możemy zapisać ten funkcjał w postaci 
%
\begin{equation}
E[n]=T[n] + E_\text{ext}[n] + E_\text{H}[n] + E_\text{xc}[n],
\label{EKS}
\end{equation}
%
gdzie $T[n]$ jest energią kinetyczną nieoddziałujących elektronów
%
\begin{equation}
T[n]=-\frac{\hbar^2}{2m}\sum_{i,\sigma} \int d\bm{r} \psi^{\sigma *}_i(\bm{r})\nabla_i^2 \psi^{\sigma}_i(\bm{r}),
\end{equation}
%
$E_\text{ext}[n]$ jest energią oddziaływania z potencjałem zewnętrznym
%
\begin{equation}
E_\text{ext}[n]=\int d\bm{r} V_\text{ext}(\bm{r}) n(\bm{r}),
\end{equation}
%
$E_\text{H}[n]$ jest energią Hartree dana wzorem (\ref{Hartree}) i $E_\text{xc}[n]$ zawiera wszystkie pozostałe oddziaływania,
czyli oddziaływanie wymienne i korelacje elektronowe, jak również różnicę między energią kinetyczną układu
oddziałujących i nieoddziałujących elektronów. W skrócie ten wyraz nazywany jest funkcjonałem wymienno-korelacyjnym.

Możemy teraz zastosować podejście wariacyjne uwzględniając warunek ortonormalności
orbitali Kohna-Shama: $\langle\psi_i^{\sigma *}|\psi_j^{\sigma'}\rangle = \delta_{ij}\delta_{\sigma\sigma'}$
i wprowadzając mnożniki Lagrange'a
%
\begin{equation}
\frac{\delta}{\delta\psi_i^{\sigma *}}(E[n]-\sum_{i,\sigma}\varepsilon_{i\sigma}[\int d\bm{r} \psi_i^{\sigma *}(\bm {r})\psi_i^{\sigma}(\bm{r})-1]) = 0.
\label{delta}
\end{equation}
%
Wstawiając funkcjonał energii w postaci (\ref{EKS}) do (\ref{delta}) otrzymujemy 
%
\begin{equation}
\frac{\delta T[n]}{\delta \psi_i^{\sigma *}}+\frac{\delta}{\delta n}(\int d\bm{r} V_{ext}(\bm{r}) n(\bm{r}) + E_H[n] + E_{xc}[n])\frac{\delta n}{\delta \psi_i^{\sigma *}}-\varepsilon_{i\sigma} \frac{\delta}{\delta\psi_i^{\sigma *}}\sum_{j,\sigma'}\int d\bm{r} \psi_j^{\sigma' *}\psi_j^{\sigma}=0,
\end{equation}
%
gdzie zastosowalismy wzór na pochodna funkcji złozonej:
% 
\begin{equation}
\frac{\delta}{\delta \psi_i^{\sigma *}}=\frac{\delta}{\delta n}\frac{\delta n}{\delta \psi_i^{\sigma *}}. 
\end{equation}
%
Wykonując pochodne wariacyjne dostajemy równanie Kohna-Shama
%
\begin{equation}
[-\frac{\hbar^2\nabla_i^2}{2m}+V_{KS}(\bm{r})]\psi_i^{\sigma}(\bm{r})=\varepsilon_{i\sigma}\psi_i^{\sigma}(\bm{r}),
\end{equation}
%
które ma postać jednocząstkowego równania Schr\"{o}dingera z potencjałem Kohna-Shama złożonym z trzech wyrazów
%
\begin{equation}
V_{KS}(\bm{r})=V_{ext}(\bm{r})+V_H(\bm{r})+V_{xc}(\bm{r})=V_{ext}(\bm{r})+\int d\bm{r'} \frac{n(\mb{r}')}{|\bm{r}-\bm{r'}|}+\frac{\delta E_{xc}[n]}{\delta n}.
\label{VKS}
\end{equation}
%
W potencjale tym nieznana jest dokładnie energia wymienno-korelacyjna, która musi być przybliżana
odpowiednimi metodami, które będą omówione w kolejnych rozdziałach.
Znając spektrum energetyczne rozwiązań równania Kohna-Shama, energię stanu podstawowego można wyrazić w postaci
%
\begin{equation}
E[n,f_i]=\sum_{i\sigma}f_{i\sigma} \varepsilon_{i\sigma} - \int d\bm{r} n(\bm{r})V_{KS}(\bm{r}) + \int d\bm{r} V_{ext}(\bm{r}) n(\bm{r}) + E_H[n] + E_{xc}[n].
\end{equation} 
%
Funkcje jednoelektronowe i energie stanów Kohna-Shama $\varepsilon_{i\sigma}$ nie mają jednoznacznej interpretacji fizycznej.
Można je jednak powiązać ze zmianą energii całkowitej przy zmianie obsadzenia stanów
%
\begin{equation}
\frac{\partial E}{\partial f_{i\sigma}}=\Big(\frac{\partial E}{\partial f_{i\sigma}}\Big)_n+\int d\bm{r} \frac{\delta E}{\delta n}\frac{\partial n}{\partial f_{i\sigma}}.
\label{janak}
\end{equation} 
%
Drugi wyraz po prawej stronie równania zeruje się dla stanu podstawowego ($\delta E/\delta n=0$),
co prowadzi do wzoru
%
\begin{equation}
\varepsilon_{i\sigma}=\frac{\partial E}{\partial f_{i\sigma}},
\end{equation}
%
który nazywany jest twierdzeniem Janaka \cite{janak}.
Zastosowanie tego wzoru do najwyższego obsadzonego stanu elektronowego daje energię jonizacji, czyli energię potrzebną do usunięcia pojedynczego elektronu 
z danego układu atomowego. 



\section{Funkcjonał wymienno-korelacyjny}

\subsection{Przybliżenie lokalnej gęstości (LDA)}

Funkcjonał wymienno-korelacyjny $E_{xc}$ i odpowiadający mu potencjał $V_{xc}$ nie są znane dokładnie i w ramach teorii
funkcjonału gęstości muszą być opisywane w przybliżony sposób.
Najprostszym podejściem jest przybliżenie lokalnej gęstości ({\it ang. local density approximation} - LDA).
W przybliżeniu LDA przyjmujemy, że energia wymienno-korelacyjna w każdym punkcie przestrzeni, gdzie gęstość elektronowa wynosi $n(\mb{r})$,
równa jest energii wymienno-korelacyjnej jednorodnego gazu elektronowego o tej samej gęstości, $n=n(\mb{r})$.
Funkcjonał wymienno-korelacyjny może być wtedy zapisany w postaci
%
\begin{equation}
E_{xc}[n]=\int d\mb{r} n(\mb{r}) \varepsilon_{xc}(n),
\end{equation}
%
gdzie $\varepsilon_{xc}(n)$ jest energią wymienno-korelacyjną przypadającą na pojedynczy elektron w jednorodnym gazie elektronowym o gestości $n$.
Można ją zapisać jako sumę części wymiennej i korelacyjnej, $\varepsilon_{xc}(n)=\varepsilon_x(n)+\varepsilon_c(n)$.   
Przybliżenie LDA jest dokładne w granicy wolno zmieniającej się gęstości, co odpowiada warunkowi
%
\begin{equation}
\frac{q}{k_F}\ll 1,
\end{equation} 
%
gdzie $q$ jest miarą niejednorodności układu
%
\begin{equation}
q=\frac{|\nabla k_F|}{2k_F},
\end{equation}
%
a $k_F$ odpowiada wektorowi falowemu Fermiego dla gazu jednorodnego, który w punkcie o lokalnej gęstości $n(\bm{r})$ dany jest wzorem
%
\begin{equation}
k_F=[3\pi^2n(\bm{r})]^{1/3}.
\end{equation} 
%

Można łatwo uogólnić to przybliżenie do układów z polaryzacją spinową. Wtedy energia wymienno-korelacyjna jest funkcjonałem
gęstości spinów skierowanych do góry $n_{\uparrow}$ i w dół $n_{\downarrow}$
%
\begin{equation}
E_{xc}[n^{\uparrow},n^{\downarrow}]=\int d\bm{r} n(\mb{r}) \varepsilon_{xc}(n_{\uparrow},n_{\downarrow}).
\label{xclda}
\end{equation}
%
To uogólnienie nazywane jest czasem przybliżeniem lokalnej gestości spinowej ({\it ang. local spin density} - LSD).
Potencjał wymienno-korelacyjny zależny od spinu $\sigma$ w tym przybliżeniu określony jest przez zależność
%
\begin{equation}
V^{\sigma}_{xc}=\frac{\delta E_{xc}}{\delta n_{\sigma}}=\varepsilon_{xc}+n_{\sigma}\frac{\partial \varepsilon_{xc}}{\partial n_{\sigma}}.
\end{equation}
%
Zgodnie ze wzorem (\ref{exchange}), część wymienna dana jest wzorem
%
\begin{equation}
\varepsilon_x(n_{\sigma})=-\frac{3}{4}\Big(\frac{3}{\pi}\Big)^{\frac{1}{3}}n_{\sigma}^\frac{1}{3},
\end{equation}
%
co prowadzi do potencjału wymiennego w formie 
%
\begin{equation}
V^{\sigma}_x=-\Big(\frac{3}{\pi}\Big)^{\frac{1}{3}}n_{\sigma}^\frac{1}{3}. 
\label{Vex}
\end{equation}
%
Część korelacyjna może być wyznaczona bardzo dokładnie numerycznie metodą kwantowego Monte Carlo \cite{CeperleyAlder80}. 
W praktyce stosuje się odpowiednie wyrażenia analityczne dofitowane do wyników numerycznych, które określają zależność energii korelacji 
od gęstości elektronowej \cite{PZ,VWN}. 
Przykładowo energia korelacji w parametryzacji Pardew-Zungera \cite{PZ} ma postać
%
\begin{eqnarray}
\varepsilon_c(r_s)&=&-0.048+0.031 \ln(r_s) -0.0116 r_s+0.002r_s \ln(r_s), \quad r_s<1  \\
             &=&-0.1423/(1+1.0529\sqrt{r_s}+0.3334 r_s), \quad r_s>1.
\end{eqnarray}
%

\subsection{Uogólnione przybliżenie gradientowe (GGA)}
\label{sec:GGA}

Uwzględnienie lokalnych zmian gęstości poprzez rozwinięcie energii wymienno-korelacyjnej w szereg gradientów gęstości
zaproponowano już w pracy Kohna i Shama z 1965 roku \cite{kohn65}. Podejście to jednak nie spełnia reguł sum i załamuje się dla większości układów atomowych, 
w których zmiany gęstości elektronowej sa przeważnie bardzo duże.
Zaproponowane zostało nowe podejście zwane uogólnionym przybliżeniem gradientowym ({\it ang. generalized gradient approximation} - GGA),
gdzie energia wymienno-korelacyjna jest funkcjonałem gęstości elektronowej i jej gradientów \cite{Langreth83,Pardew86,Becke88}. 
W ogólnej formie dla układu spolaryzowanego spinowo może być zapisana w formie
%
\begin{equation}
E_{xc}[n_{\uparrow},n_{\downarrow}]=\int d\mb{r} f(n_{\uparrow},n_{\downarrow},\nabla n_{\uparrow},\nabla n_{\downarrow}) 
\label{xcgga}
\end{equation}
%
Część wymienną tego funkcjonału dla układu bez polaryzacji spinowej można zapisać w postaci
%
\begin{equation}
E_x[n]=\int d\mb{r} n \varepsilon_x(n) F_x(s),
\end{equation}
%
gdzie $s=|\nabla n|/2k_Fn$ jest przeskalowanym (bezwymiarowym) gradientem gęstości elektronowej. Rozszerzenie na układy z polaryzacja spinową uzyskuje się stosując następujący wzór,
który spełniony jest dla dokładnej energii wymiennej
%
\begin{equation}
E_x[n^{\uparrow},n^{\downarrow}]=\frac{1}{2}(E_x[2n^{\uparrow}]+E_x[2n^{\downarrow}]).
\end{equation}
%
Wiele form funkcji $F_x(s)$ zostało zaproponawych, w tym te najczęściej stosowane, które opisane są w pracach: A. D. Becke (Becke88) \cite{Becke88}, 
J. P. Perdew i Y. Wang (PW91) \cite{PW91} i  J. P. Perdew, K. Burke, and M. Ernzerhof (PBE) \cite{PBE}.
Jako przykład omówię funkcjonał PBE, który ma prostą postać i dobrze zdefiniowane warunki graniczne  
które musi spełniać funkcja $F_x(s)$:

\begin{enumerate}

\item{W granicy małych wartości gradientu ($s\rightarrow0$)
spełniony jest warunek
%
\begin{equation}
F_x(s)\rightarrow 1+\mu s^2,
\end{equation}
%
gdzie $\mu =0.219$. Warunek ten zapewnia właściwe zachowanie się odpowiedzi liniowej jednorodnego gazu
elektronów (liniowe przyczynki od energii wymiany i korelacji kasują się).}

\item{Dla dużych wartości gradientu ($s\rightarrow\infty$) funkcja ograniczona jest od góry $F_x(s)\leq 1.804$.}

Funkcja, która spełnia te warunki ma postać
%
\begin{equation}
F_x(s)=1+\kappa-\frac{\kappa}{1+\frac{\mu s^2}{\kappa}},
\end{equation} 
%
gdzie $\kappa=0.804$.
\end{enumerate}

Część korelacyjną można zapisać w ogólnej formie
%
\begin{equation}
E_c[n^{\uparrow},n^{\downarrow}]=\int d\mb{r} n [\epsilon_c(r_s,\zeta)+H(r_s,\zeta,t)],
\end{equation}
%
gdzie $\zeta=(n^{\uparrow}-n^{\downarrow})/n$ jest względną polaryzacją spinową, $t=|\nabla n|/2\phi k_s n$ jest bezwymiarowym gradientem,
$\phi=\sqrt{(1+\zeta)^{2/3}+(1-\zeta)^{2/3})}/2$ jest spinową funkcją skalującą i $k_s=\sqrt{4k_F/\pi a_0}$ wektorem ekranowania Thomasa-Fermiego.
Funkcja $H$ spełnia następujące warunki:

\begin{enumerate}
\item{Dla małych gradientów ($t\rightarrow 0$) funkcja dana jest wyrazem rozwinięcia drugiego rzędu
%
\begin{equation}
H\rightarrow \frac{e^2}{a_0}\beta\phi^3 t^2,
\end{equation}
%
gdzie $\beta=0.067$.}
\item{Przy szybko zmieniających się gęstościach ($t\rightarrow \infty$) energia korelacji znika
co spełnione jest przy warunku
%
\begin{equation}
H\rightarrow -\epsilon_c.
\end{equation}}
%
\item{W granicy dużych gęstości ($r_s\rightarrow 0$) energia korelacji dąży do stałej wartości.
Funkcja $H$ musi kasować logarytmiczną osobliwość w $\epsilon_c$
%
\begin{equation}
H\rightarrow \frac{e^2}{a_0}\gamma\phi^3 \ln t^2,
\end{equation}
%
gdzie $\gamma=0.031$.}
\end{enumerate}

Warunki te spełnia funkcja w postaci
%
\begin{equation}
H=\frac{e^2}{a_0}\gamma\phi^3\ln[1+\frac{\beta}{\gamma}t^2\frac{1+At^2}{1+At^2+A^2t^4}],
\end{equation}
%
gdzie 
%
\begin{equation}
A=\frac{\beta}{\gamma}[\exp\{-\epsilon_c/(\gamma\phi^3e^2/a_0)\}-1]^{-1}.
\end{equation}


W tabeli \ref{EXC} porównane są wartości energii wymiany i korelacji w przybliżeniu LDA i PBE
z wartościami dokładnymi wyznaczonymi dla kilku wybranych atomów. Wartości bezwzględne energii wymiany są większe
o około rząd wielkości od energii korelacji. W LDA, energia wymiany jest zaniżona średnio o $10\%$, a energia korelacji jest znacznie zawyżona,
nawet powyżej $100\%$, w porównaniu do wartości dokładnych. Ze względu na przeciwne znaki odchyleń od wartości dokładnych, błędy energii wymiany 
i korelacji w przybliżeniu LDA częściowo kasują się. 

\begin{table}[h!]
\caption{Porównanie energii wymiennej i korelacyjnej uzyskanych w kilku przybliżeniach z wartościami dokładnymi dla kilku wybranych atomów.}
\label{EXC}
\begin{center}
\begin{tabular}{|c|c|c|c|c|c|c|}
\hline
 & \multicolumn{2}{c|}{LDA} & \multicolumn{2}{c|}{PBE} & \multicolumn{2}{c|}{Dokładna wartość} \\ \hline
Atom& $E_x$ & $E_c$ & $E_x$ & $E_c$  & $E_x$  & $E_c$ \\
\hline
  H & -0.2680 & -0.0222 &  -0.3059 & -0.0060 & -0.3125 & 0.0000\\
  He & -0.8840 & -0.1125 &  -1.0136 & -0.0420 & -1.0258 & -0.0420\\
  Be & -2.3124 & -0.2240 &  -2.6358 & -0.0856 & -2.6658 & -0.0950\\
  N &  -5.9080 & -0.4268 &  -6.5521 & -0.1799 & -6.6044 & -0.1858\\
  Ne & -11.0335 & -0.7428 &  -12.0667 & -0.3513 & -12.1050 & -0.3939\\ \hline
\end{tabular}
\end{center}
\end{table}

Te duże różnice w energii wymiany i korelacji wynikają z nieuwzględnienia wpływu lokalnych zmian 
gęstości elektronowej, które są charakterystyczne dla atomów i molekuł \cite{Pardew92}.  
W przybliżeniu GGA, wartości obydwóch energii poprawiają się. Energia wymiany jest w dalszym ciągu zaniżona, ale tylko
na poziomie $\sim 1\%$. Energia korelacji w atomie wodoru jest znacznie zredukowana w porównaniu do LDA,
a dla atomu helu jest bardzo dobrze odtworzona. W pozostałych przypadkach pokazanych w tabeli \ref{EXC} jest mniejsza
od wartości dokładnej o kilka procent.

Funkcjonał GGA w granicy zerowych gradientów, czyli dla jednorodnego gazu elektronowego, jest równoważny przybliżeniu LDA.
Oznacza to, że dla niejednorodnych układów powinień dawać zawsze wyniki lepsze niż LDA. Tak się jednak nie dzieje, a wynika
to z faktu, że dla niektórych układów częściowe redukowanie się błedów energii wymiany i korelacji jest większe w LDA niż w GGA.
Przykładem są metale szlachetne (Ag, Au, Pt), dla których stałe sieci wyznaczone w LDA bardzo dobrze zgadzają się
z eksperymentem, a w GGA są zawyżone. Większe wartości stałej sieci w przybliżeniu GGA wynikają z zaniżonej energii kohezji kryształów.
LDA zwykle daje zawyżone wartości energii kohezji i zaniżone wartości stałych sieci. 
Dla układów magnetycznych, czyli z otwartymi powłokami $d$ lub $f$, przybliżenie GGA daje zwykle lepsze wyniki niż LDA.
Klasycznym przykładem jest żelazo, dla którego stanem podstawowym w przybliżeniu LDA jest struktura centrowana powierzchniowo ($fcc$) 
z uporządkowaniem antyferromagnetycznym, a GGA daje poprawny wynik, czyli strukturę centrowaną objętościowo ($bcc$) z uporządkowaniem ferromagnetycznym.

\section{Procedura iteracyjna}

Równania Kohna-Shama (KS) rozwiązywane są metodą iteracyjną, której kolejne kroki przedstawione są na rysunku \ref{fig:diagram}.
Ogólnie stosowane procedury pozwalają zminimalizować funkcjonał energii całkowitej przy zadanych położeniach atomów,
jak również wyznaczyć położenia atomów i parametry sieci krystalicznej, które odpowiadają minimum energii całkowitej
układu. Najpierw ustalamy początkowe położenia atomów, które można przyjąć na przykład na podstawie
danych z eksperymentu dyfrakcyjnego. Przy ustalonych położeniach atomowych przyjmuje się początkowy
rozkład gęstości elektronowej $n_0(\bm{r})$, dla którego wylicza się potencjał ze wzoru (\ref{VKS}).
Dla tego potencjału rozwiązuje się równania KS i wyznacza jednoelektronowe funkcje falowe
w odpowiedni wybranej bazie funkcyjnej. 

\begin{figure}[h]
\centering
\includegraphics[scale=0.45]{diagram.pdf}
\caption{Schemat interacyjnej procedury rozwiązywania równań Kohna-Sham.}
\label{fig:diagram}
\end{figure}

Rozwiązywanie równań KS dla atomów w ciele stałym jest złożonym problemem i będzie dyskutowane w kolejnych rozdziałach.
Znając funkcje falowe i energie własne równania KS wyznacza się gęstość elektronową w każdym punkcie przestrzeni 
z zależności (\ref{density}) i całkowitą energię układu. 
Nowy rozkład gęstości pozwala ponownie wyznaczyć potencjał KS i rozwiązać równania KS. Dostajemy nowy rozkład gęstości i zmienioną
energię całkowitą. Jeżeli ta energia pokrywa się z energią w poprzednim
kroku lub te energie różnią sie o małą wielkość (zdefiniowaną na początku obliczeń) to oznacza, że 
otrzymaliśmy prawidłowo zbiegnięty rozkład gęstości. Zakończona jest w ten sposób pętla elektronowa, 
która daje rozwiązanie równań KS i minimalną energię całkowitą przy zadanych położeniach atomowych.

Procedurę minimalizacji można przyspieszyć wykorzystując odpowiednio gęstości elektronowe
w każdym kroku iteracyjnym. W najprostszym podejściu początkową wartość gęstości elektronowej $n^p_{i+1}$ w kroku $i+1$ 
można wyznaczyć jako liniową poprawkę do gęstości z poprzedniego kroku
%
\begin{equation}
n^p_{i+1}=n^p_i+\alpha(n^k_i-n^p_i),
\label{mixing}
\end{equation}
% 
gdzie $\alpha$ jest współczynnikiem liniowym, $n^p_i$ i $n^k_i$ są początkową i końcową wartością gęstości w kroku $i$.
Stała wartość współczynnika $\alpha$ nie daje optymalnej szybkości zbieżności.
W najczęściej stosowanej metodzie Broydena, współczynnik liniowy wyznaczany jest z jakobianu układu $J_i$,
który optymalizowany jest w każdym kroku iteracji, $\alpha=-J^{-1}_i$.

Aby zoptymalizować układ ze względu na położenia atomowe należy odpowiednio przesuwać
atomy w nowe położenia, tak aby całkowita energia obniżała się w kolejnych krokach. 
Wykorzystuje się do tego bardzo efektywną metodę sprzężonego gradientu ({\it ang. conjugate gradient}),
która jest również metodą iteracyjną (pętla jonowa).
W każdym kroku tej procedury wykonuje się pełną optymalizację elektronową, aby wyznaczyć rozkład gęstości elektronowej 
i energię układu dla aktualnego położenia atomów. Wyznaczona gęstość końcowa jest używana do wyznaczenia
gęstości startowej w kolejnym kroku pętli jonowej według formuły (\ref{mixing}).
Warunkiem zbieżności jest osiągnięcie stanu o minimalnej energii ze względu na położenia atomów
i parametry sieci krystalicznej. Według najczęściej przyjmowanego kryterium zbieżność jest osiągnięta jeżeli różnice energii 
całkowitej w dwóch kolejnych krokach jonowych jest mniejsza od zadanej na początku wartości.

Stan podstawowy otrzymany w wyniku procedury minimalizacyjnej powinien spełniać warunki stanu równowagi badanego materiału.
Warunki te są następujące: 

\begin{enumerate}
{\item
Całkowita siła działająca na każdy atom zeruje się.
Siły działające na atomy można wyznaczyć stosując twierdzenie Hellmanna-Feynmana \cite{Hellmann,Feynman}, 
które mówi, że siła działająca na atom $i$ równa jest pochodnej energii całkowitej po położeniu
tego atomu wziętej ze znakiem przeciwnym
%
\begin{equation}
\bm{F_i}=-\frac{\partial E_{tot}}{\partial\bm{R_i}}.
\end{equation}
%
Zgodnie z tym wzorem, w stanie podstawowym, który odpowiada minimum energii, wszystkie siły $\bm{F_i}$ są równe zeru.
Wartości sił stanowią często dodatkowe kryterium zbieżności układu w procedurze optymalizacyjnej:
zbieżność jest osiągnięta jeżeli największa siła działająca na atomy jest mniejsza od zadanej wartości.}

{\item Makroskopowe naprężenia w układzie równe jest naprężeniu wywołanym ciśnieniem zewnętrznym.
Średni tensor naprężeń określony jest przez pochodną energii całkowitej po tensorze deformacji 
%
\begin{equation}
\sigma_{\alpha\beta}=-\frac{1}{V}\frac{\partial E}{\partial u_{\alpha\beta}}.
\end{equation}
gdzie $V$ jest objętością układu. Tensor deformacji jest symetrycznym tensorem pochodnych wektora przesunięcia
$\bm{u}=\bm{r}-\bm{r'}$ po położeniu $\bm{r}$.
Przy kompresji hydrostatycznej, ciśnienie wiąże się z tensorem naprężeń zależnością $P=-\frac{1}{2}\sum_{\alpha}\sigma_{\alpha\alpha}$.}
\end{enumerate}



\chapter{Metody wyznaczania struktury elektronowej}

\section{Ogólna charakterystyka}

Najważniejsze metody wyliczania struktury pasmowej oparte są na teorii DFT i polegają na
wyznaczaniu jednocząstkowych funkcji falowych poprzez rozwiązanie równań Kohna-Shama
lub podobnych równań zawierających efektywny potencjał elektronowy.  
Podstawowym elementem odróżniającym poszczególne metody jest wybór funkcji bazowych, 
które służą do rozwinięcia funkcji falowych w całym obszarze kryształu lub w poszczególnych jego częściach. 
Najbardziej naturalną bazą w periodycznym krysztale są fale płaskie i obliczenia w tej bazie są bardzo efektywne oraz łatwe do implementacji. 
Niestety, nie nadają sie do opisu elektronów w całym obszarze kryształu i wszystkich stanów elektronowych.
Elektrony, które znajdują się najbliżej jąder atomowych i obsadzają najniższe stany energetyczne 
wchodzą w skład rdzenia atomowego ({\it ang. atomic core}).
W obszarze rdzeni atomowych, potencjał elektryczny jest silnie przyciągający i zbliżony jest do potencjału atomowego.
Przez to stany elektronowe znajdujące się blisko jądra atomowego, zachowują sie podobnie do orbitali atomowych, czyli 
silnie oscylują i zmieniają swój znak. Oznacza to również, że energia kinetyczna elektronów w tym obszarze jest duża.
Funkcje falowe stanów rdzenia sa silnie zlokalizowane i szybko zanikają z odległością. 
W tym przypadku znacznie lepszą zbieżność uzyskuje się stosując bazę zlokalizowanych funkcji, np. harmoniki sferyczne.
Funkcja falowa elektronów walencyjnych zachowuje się odmiennie w obszarze rdzeni atomowych i w obszarze międzywęzłowym.
Schematycznie potencjał periodyczny i funkcja falowa elektronów walencyjnych pokazana jest na rysunku \ref{fig:bloch}.
W obszarze rdzeni, zaznaczonych kółkami o promieniu $r_c$, funkcja falowa, podobnie jak w przypadku stanów rdzenia,
zmienia się bardzo szybko.
W obszarze miedzywęzłowym ($r>r_c$), potencjał i gęstość elektronowa zmieniają się wolno w funkcji położenia. 
Również funkcja falowa elektronów walencyjnych w tym obszarze jest wolnozmienna i 
w tym przypadku można uzyskać szybką zbieżność w rozwinięciu na fale płaskie. 


\begin{figure}[h!]
\centering
\includegraphics[scale=0.5]{crystal.pdf}
\caption{Potencjał periodyczny i funkcja falowa elektronów walencyjnych.}
\label{fig:bloch}
\end{figure}

\newpage

Najważniejsze metody wyznaczania struktury pasmowej można podzielić na trzy główne grupy:


\begin{enumerate}
\item{{\bf Pseudopotencjały.} Jest to grupa metod, która ogranicza ilość elektronów i rozwiązania równania Kohna-Shama tylko do stanów walencyjnych. 
Ma to uzasadnienie wynikające z charakteru stanów rdzenia, które są silnie zlokalizowane blisko jądra atomowego i nie biorą udziału w wiązaniach atomowych.
Przez to najważniejsze własności materiałów zdeterminowane są przez zachowanie elektronów walencyjnych. 
W metodzie pseudopotencjału stosuje się rozwinięcie funkcji falowej w bazie fal płaskich. Aby zachować jednolity opis funkcji falowej w całym obszarze kryształu, stosuje się w tych metodach przybliżony potencjał działający na elektrony walencyjne w obszarze rdzenia atomowego, który nazywany jest pseudopotencjałem. 
Dla elektronów walencyjnych wyznacza się pseudofunkcję falową, która jest rozwiązaniem jednocząstkowego równania Schr\"{o}dingera z odpowiednim pseudopotencjałem. W obszarze międzywęzłowym jest ona równa dokładnej funcji falowej, a w obszarze rdzenia jest wolnozmienną funkcją bez oscylacji i miejsc
zerowych. Najdokładniejsze obliczenia w tej grupie metod stosują pseudopotencjały ultramiękkie i potencjały typu PAW.}
\item{{\bf Zlokalizowane orbitale.} W tym podejściu funkcję falową elektronów zapisujemy w bazie orbitali zlokalizowanych na poszczególnych
atomach. W najprostszym przybliżeniu ciasnego wiązania jedynymi istotnymi parametrami są elementy macierzowe Hamiltonianu, które
opisują przekrywanie się lokalnych orbitali lub funkcji Wanniera. W dokładniejszych obliczeniach jako bazę stosuje się funkcje Gaussa
lub orbitale Slatera.}
\item{{\bf Stowarzyszone fale i atomowe sfery}. Ta grupa obejmuje metody obliczeniowe, które bazują na ogólnej zasadzie podziału
kryształu na dwa obszary. Pierwszy obszar zawiera rdzenie atomowe, w których funkcje falowe zachowują cechy orbitali atomowych i drugi obszar między atomami,
gdzie elektrony walencyjne opisane są wolnozmienną funkcją falową. W każdym z dwóch charakterystycznych obszarów stosuje się
rozwinięcia funkcji falowej w różnych bazach funkcyjnych. Obszar rdzenia atomowego definiuje wartość promienia $r_c$. Funkcje falowe
otrzymane dla odległości mniejszych i wiekszych od $r_c$ muszą spełniać odpowiednie warunki ciągłości na granicy rdzenia atomowego.}

\end{enumerate}

W kolejnych rozdziałach opisane zostaną metody obliczeniowe należące do tych trzech grup.
  
\section{Pseudopotencjały}

Główną ideę pseudopotencjału ilustruje rysunek \ref{fig:pseudopot}. W obszarze rdzenia atomowego ($r<r_c$) dokładny potencjał $V$ przybliżany
jest pseudopotencjałem $\tilde{V}$, który ma skończoną wartość dla $r=0$. Odpowiadająca mu pseudofunkcja falowa $\tilde{\psi}$
ma gładkie zachowanie (bez oscylacji i miejsc zerowych) w tym obszarze. Dla $r>r_c$ pseudopotencjał pokrywa się z dokładnym potencjałem,
a pseudofunkcja odpowiada dokładnej funkcji falowej $\psi$. Dzięki takiemu przybliżeniu, pseudofunkcja falowa może być rozwinięta
na fale płaskie w całym obszarze kryształu.   

  
\begin{figure}[h!]
\centering
\includegraphics[scale=0.1]{pseudopots-new.pdf}
\caption{Psudopotencjał i pseudofunkcja falowa.}
\label{fig:pseudopot}
\end{figure}  
  
Metoda pseudopotencjału rozwinęła się z podejścia zortogonalizowanych fal płaskich ({\it ang. ortogonalized plane waves} - OPW)
zaproponowanego przez Herringa w 1940 \cite{herring40}. 
Można zdefiniować transformację, która prowadzi od dokładnego potencjału do pseudopotencjału, wprowadzając jednocześnie pseudofunkcję falową do opisu elektronów walencyjnych \cite{Antoncik,KP}.
Wprowadzamy osobne oznaczenia dla stanów walencyjnych $|\psi_v\rangle$ i stanów rdzenia $|\psi_c\rangle$, które są stanami własnymi
hamiltonianu $H$ z odpowiednimi energiami własnymi $\varepsilon_v$ i $\varepsilon_c$.
Zakładamy, że dokładna funkcja falowa elektronów walencyjnych może być wyrażona przez gładką funkcję $\tilde{\psi}_v$ (pseudofunkcję),
która jest ortogonalna do stanów rdzenia
%
\begin{equation}
|\psi_v\rangle =|\tilde{\psi}_v\rangle +\sum_{\alpha,c} a_{\alpha} |\psi_{\alpha c}\rangle,
\label{OPW}
\end{equation}
%
gdzie sumowanie przebiega po atomach $\alpha$ i stanach rdzenia $c$. 
Dla uproszczenia zapisu wskaźniki $v$ i $c$ określają zarówno numer stanu, jak i kierunek spinu elektronu.
Współczynniki rozwinięcia dostajemy z warunku ortogonalności  
%
\begin{equation}
a_{\alpha}=-\langle \psi_{\alpha c}|\tilde{\psi}_v\rangle.
\label{a_alpha}
\end{equation}
% 
W obszarze miedzywęzłowym dokładna funkcja falowa równa jest pseudofunkcji, która może być wyrażona w bazie fal płaskich.
Wstawiając (\ref{OPW}) do równania $H|\psi_v\rangle=\varepsilon_v|\psi_v\rangle$ otrzymujemy
%
\begin{equation}
H|\tilde{\psi}_v\rangle +\sum_{\alpha,c} a_{\alpha} \varepsilon_{\alpha c} |\psi_{\alpha c}\rangle =
\varepsilon_v|\tilde{\psi}_v\rangle + \sum_{\alpha c} a_{\alpha} \varepsilon_v|\psi_{\alpha c}\rangle.
\end{equation}
%
Przenosząc drugie równanie na prawą stronę i wykorzystując (\ref{a_alpha}) dostajemy
%
\begin{equation}
(H +\sum_{\alpha,c}(\varepsilon_v - \varepsilon_{\alpha c})|\psi_{\alpha c}\rangle \langle \psi_{\alpha c}|)|\tilde{\psi}_v\rangle =
\varepsilon_v|\tilde{\psi}_v\rangle.
\end{equation}
%
Dostaliśmy równanie typu Schr\"{o}dingera z dodatkowym potencjałem 
%
\begin{equation}
V^{R}=\sum_{\alpha,c}(\varepsilon_v - \varepsilon_{\alpha c})|\psi_{\alpha c}\rangle \langle \psi_{\alpha c}|.
\end{equation}
%
Jeżeli dodamy to wyrażenie do oryginalnego potencjału $V$ dostaniemy wielkość nazywaną pseudopotencjałem $\tilde{V}=V+V^{R}$.
W obszarze walencyjnym, pseudopotencjał $\tilde{V}$ pokrywa się z potencjałem $V$.  
Silnie przyciagający potencjał atomowy w obszarze rdzenia jest znacznie osłabiony przez dodatnią wartość $V^R$.
To umożliwia zbieżność pseudofunkcji falowej w bazie fal płaskich. 


\subsection{Pseudopotencjały zachowujące normę}

Dobre pseudopotencjały mają charakter uniwersalny. Wygenerowane w określonej configuracji atomowej
powinny zachowywać się jednakowo dobrze w każdym innym ukladzie atomowym. 
W roku 1979, Hamann, Schl\"{u}ter i Chiang \cite{HSC} zaproponowali pseudopotencjały zachowujące normę, 
które spełniały  nastepujące warunki:

\begin{enumerate}
{\item Prawdziwa funkcja i pseudofuncja falowa są równe, $\psi_i(r)=\tilde{\psi}_i(r)$, dla $r\geq r_c$.}
{\item Ich energie własne dla elektronów walencyjnych powinny być równe.}
{\item Ładunek prawdziwy i pseudoładunek zawarty w obszarze o promieniu $r\geq r_c$ powinny się pokrywać (zachowanie normy)
%
\begin{equation}
\int_0^r dr r^2 |\psi_i(r)|^2=\int_0^r dr r^2 |\tilde{\psi}_i(r)|^2.
\end{equation}}
%
{\item Pochodna logarytmiczna prawdziwej funkcji i pseudofunkcji są równe dla $r\geq r_c$ 
%
\begin{equation}
\frac{1}{\psi_i(r)}\frac{d\psi_i(r)}{dr}=\frac{1}{\tilde{\psi}_i(r)}\frac{d\tilde{\psi}_i(r)}{dr}.
\end{equation}}

\end{enumerate} 
%
Z tych własności wynika również spełnienie warunku równych pochodnych po energii funkcji dokładnej i pseudofunkcji dla $r\geq r_c$,
co dodatkowo wzmacnia przenośny charakter tych pseudopotencjałów.
Metody generowania pseudopotencjałów zachowujących normę zostały opisane szczegółowo w wielu pracach \cite{HSC,BHS}.

Ponieważ pseudopotencjały w obszarze rdzenia zależą od orbitalnej liczby kwantowej $l$, ich charakter nie jest lokalny.
Wartość charakterystycznego promienia $r_c$ również zależy od $l$.
Ogólnie można podzielić cały pseudopotencjał na część lokalną i nielokalną
%
\begin{equation}
V_l(r)=V_{lok}(r)+\delta V_l(r).
\end{equation}
%
Nielokalny charakter dotyczy tylko obszaru rdzenia, więc $\delta V_l(r)=0$ dla $r>r_c$
i wszystkie dalekozasięgowe efekty zależą tylko od $V_{lok}(r)$.
Możliwość wyboru promienia $r_c$ daje pewną swobodę przy konstrukcji pseudopotencjałów.
Dokładniejsze i bardziej uniwersalne pseudopotencjały charakteryzują sie mniejszymi wartościami
$r_c$. Jednak większa dokładnosć wymaga uwzględnienia większej ilości fal płaskich
w rozwinięciu pseudofunkcji falowej. Takie pseudopotencjały nazywane są twardymi. Miękkie pseudopotencjały odznaczają się
większymi wartościami $r_c$, mniejszą ilość fal płaskich i bardziej gładkim charakterem pseudofunkcji falowej. 

\subsection{Pseudopotencjały ultramiękkie}

W roku 1990, Vanderbilt zaproponował nowy rodzaj pseupotencjałów, nazywane ultramiękkimi ({\it ang. ultrasoft pseudopotentials} - US-PP), które nie zachowują normy \cite{Vanderbilt90}. W tym podejściu, pseudofunkcje falowe spełniają ugólnione równanie własne w postaci
%
\begin{equation}
H|\tilde{\psi}_i\rangle=\varepsilon_iS|\tilde{\psi}_i\rangle,
\end{equation}
%
gdzie Hamiltonian ma ogólną postać 
%
\begin{equation}
H=-\frac{\hbar^2}{2m}\nabla^2+V_{lok}+\delta V_{NL}, 
\end{equation}
a $S$ jest operatorem przekrywania
%
\begin{equation}
S=1+\sum_{i,j} Q_{ij}|\beta_i\rangle\langle\beta_j|,
\end{equation}
%
który określa uogólniony warunek normalizacji pseudofunkcji falowych $\langle\tilde{\psi}_i|S|\tilde{\psi}_j\rangle=\delta_{ij}$.
Macierz $Q_{ij}$ opisuje różnicę w normalizacji funkcji i pseudofunkcji falowych
%
\begin{equation}
Q_{ij}=\int_0^{r_c} dr Q_{ij}(\bm{r})=\int_0^{r_c} dr r^2[\psi_i^*(\bm{r})\psi_j(\bm{r})-\tilde{\psi}_i^*(\bm{r})\tilde{\psi}_j(\bm{r})].
\end{equation}
%
Zbiór lokalnych funkcji falowych
%
\begin{equation}
|\beta_i\rangle=\sum_j (B^{-1})_{ij}|\chi_j\rangle,
\end{equation}
%
gdzie funkcje 
%
\begin{equation}
|\chi_j\rangle=(\varepsilon_i+\frac{\hbar^2}{2m}\nabla^2-V_{loc})|\tilde{\psi}_j\rangle
\end{equation}
%
zerują się w obszarze dla $r>r_c$. 
Nielokalną część speudopotencjału otrzymuje się ze wzoru
%
\begin{equation}
\delta V_{NL}=\sum_{ij}D_{ij}|\beta_i\rangle\langle\beta_j|,
\end{equation}
%
gdzie $D_{ij}=B_{ij}+\varepsilon_j Q_{ij}$. Całkowitą gęstość elektonów walencyjnych
w danym punkcie przestrzeni otrzymujemy ze wzoru 
%
\begin{equation}
n_v(\bm{r})=\sum_i \tilde{\psi}_i^*(\bm{r})\tilde{\psi}_i(\bm{r})+\sum_{i,j}\sum_k\langle\beta_i|\tilde{\psi}_k\rangle\langle\tilde{\psi}_k|\beta_j\rangle Q_{ij}(\bm{r}),
\end{equation}
% 
gdzie wyraz jest poprawką wynikającą z niezachowania normy przez pseudofunkcje falowe.


\begin{figure}[h]
\centering
\includegraphics[scale=1]{usp.pdf}
\caption{Porównanie dokładnej funkcji falowej $2p$ dla tlenu (linia ciągła) z pseudofunkcją zachowującą normę (linia kropkowana)
i pseudofunkcją otrzymaną dla pseudopotancjału ultramiękkiego (linia przerywana). Rysunek z pracy \cite{Vanderbilt90}.}
\label{fig:ups}
\end{figure}

Na rysunku \ref{fig:ups} pokazana jest dokładna funkcja falowa dla orbitalu $2p$ w tlenie oraz dwie pseudofunkcje otrzymane dla pseudopotencjału zachowujacego normę i pseudopotencjału ultramiękkiego. Promień obcięcia jest wyraźnie większy dla pseudopotencjału ultramiękkiego ($r_c\approx1.5$ a.u.) niż dla zachowującego normę ($r_c\approx0.5$ a.u.).

\subsection{Metoda PAW}

Omówione metody pseudopotencjałów pozwalaja efektywnie wyliczać wiele wielkości fizycznych bazując na pseudofukcjach falowych.
W wielu przypadkach dokładniejsze obliczenia wymagają znajomości funkcji falowych elektronów walencyjnych
również w obszarze rdzenia atomowego.
W roku 1994, Bl\"{o}chl zaproponował nową metodę PAW ({\it ang. projector augmented-wave}),
która łączy efektywność pseudopotencjałów i dokładność obliczeń porównywalną z metodami pełnego potencjału \cite{Blochl}. 
Do opisu elektronów walencyjnych podejście to wykorzystuje również pseudofunkcje falowe $\tilde{\psi}_v$, które rozwijane są na fale płaskie i 
w obszarze międzywęzłowym pokrywają sie z funkcjami dokładnymi $\psi_v$.
Główną zaletą tej metody jest możliwość wyznaczenia dokładnej funkcji falowej poprzez transformację liniową pseudofukcji falowej
%
\begin{equation}
|\psi_v\rangle=\mathcal{T}|\tilde{\psi}_v\rangle.
\end{equation}
% 
Operator liniowy $\mathcal{T}$ działa w obszarach otoczonych sferą o promieniu $r_c$ wokół atomów w położeniach $\bm{R}_m$
%
\begin{equation}
\mathcal{T}=1+\sum_m \mathcal{T}_m.
\label{operator}
\end{equation}
% 
Dokładna funkcja falowa i pseudofukcja rozwinięte są w obszarze atomowym na funkcje parcjalne $|\phi_m\rangle$ i $|\tilde{\phi}_m\rangle$
%
\begin{eqnarray}
|\psi_v\rangle=\sum_m c_m |\phi_m\rangle, \label{eq1}\\
|\tilde{\psi_v}\rangle=\sum_m c_m |\tilde{\phi}_m\rangle,
\label{eq2}
\end{eqnarray}
%
gdzie w obydwóch sumach występują te same współczynniki rozwinięcia $c_m$. Zatem te lokalne funkcje powiązane są tą samą
transformacją
%
\begin{equation}
|\phi_m\rangle=(1+\sum_m \mathcal{T}_m)|\tilde{\phi}_m\rangle.
\end{equation}
%
Funkcje $|\phi_m\rangle$ są rozwiązaniami równania Schr\"{o}dingera dla dokładnego potencjału atomowego, które odpowiadają energiom $\varepsilon_m$
i są ortogonalne do stanów rdzenia. Wskaźnik $m$ określa jednocześnie położenia atomów $\bm{R}$ i zbiór liczb kwantowych orbitali atomowych.
Każdej parcjalnej funkcji dokładnej odpowiada jedna pseudofunkcja $|\tilde{\phi}_m\rangle$, z którą pokrywa się poza sferą o promieniu $r_c$.
Oba rodzaje funkcji parcjalnych są funkcjami radialnymi, zdefiniowanymi na logarytmicznej siatce radialnej, przemnożonymi przez
harmoniki sferyczne. 

Odejmując stronami (\ref{eq1}) i (\ref{eq2}) dostajemy równanie
%
\begin{equation}
|\psi_v\rangle=|\tilde{\psi}_v\rangle-\sum_m c_m |\tilde{\phi}_m\rangle + \sum_m c_m |\phi_m\rangle.
\label{psiv}
\end{equation}
% 
Aby operator $\mathcal{T}$ był liniowy współczynniki $c_m$ muszą być liniowymi funkcjonałami pseudofunkcji falowej $|\tilde{\psi}_v\rangle$
%
\begin{equation}
c_m=\langle\tilde{p}_m|\tilde{\psi}_v\rangle,
\label{coef}
\end{equation}
%
gdzie $\langle\tilde{p}_m|$ są funkcjami rzutowymi dualnymi względem funkcji parcjalnych 
$\langle\tilde{p}_m|\tilde{\phi}_n\rangle=\delta_{m,n}$.
Dla każdej funkcji $|\tilde{\phi}_m\rangle$ mamy jedną funkcja rzutową $\langle\tilde{p}_m|$
i dla nich spełniony jest warunek $\sum_m |\tilde{\phi}_m\rangle\langle\tilde{p}_m|=1$.
Funkcje rzutowe są funkcjami radialnymi pomnożonymi przez harmoniki sferyczny, a następnie przetransformowane do bazy fal płaskich.
Są przypisane do położeń atomów, ale nie zależą od potencjału atomowego.
%
Wstawiając (\ref{coef}) do (\ref{psiv}) otrzymujemy
%
\begin{equation}
|\psi_v\rangle=|\tilde{\psi}_v\rangle+\sum_m (|\phi_m\rangle-|\tilde{\phi}_m\rangle)\langle\tilde{p}_m|)\tilde{\psi}_v\rangle
=[1+\sum_m (|\phi_m\rangle-|\tilde{\phi}_m\rangle)\langle\tilde{p}_m|]|\tilde{\psi}_v\rangle,
\label{psiv2}
\end{equation}
%
gdzie wyrażenie w nawiasie kwadratowym jest wprowadzonym wcześniej operatorem liniowym
%
\begin{equation}
\mathcal{T}=1+\sum_m \mathcal{T}_m=1+\sum_m (|\phi_m\rangle-|\tilde{\phi}_m\rangle)\langle\tilde{p}_m|.
\end{equation}
%

Funkcje falowe rdzenia $|\psi_c\rangle$, podobnie do funkcji walencyjnych, są rozłożone na trzy składowe
%
\begin{equation}
|\psi_c\rangle=|\tilde{\psi}_c\rangle+|\phi_c\rangle-|\tilde{\phi}_c\rangle,
\end{equation}
%
gdzie kolejne wyrazy odpowiadają pseudofunkcji elektronów rdzenia, która pokrywa się z dokładną funkcją dla $r>r_c$,
lokalnej (parcjalnej) funkcji rdzenia i lokalnej pseudofukcji rdzenia. Obydwie lokalne funcje wyrażone są jako funkcje radialne
pomnożone przez harmoniki sferyczne.

Średnia z operatora $A$ może być wyznaczona przy pomocy dokładnych funkcji lub pseudofunkcji
%
\begin{equation}
\langle A \rangle = \sum_n f_n \langle \psi_n|A|\psi_n\rangle = \sum_n f_n \langle \tilde{\psi}_n|\tilde{A}|\tilde{\psi}_n\rangle,
\end{equation}
% 
gdzie $f_n$ określa obsadzenie stanu $n$, a $\tilde{A}$ jest pseudooperatorem, który można otrzymać transformując operator $A$
%
\begin{equation}
\tilde{A}=\mathcal{T}^{\dagger} A \mathcal{T}= A +\sum_{i,j} |\tilde{p}_i\rangle (\langle \phi_i |A|\phi_j\rangle - \langle \tilde{\phi}_i |A|\tilde{\phi}_j\rangle ) \langle\tilde{p}_j|.
\end{equation}
%
Przykładowo, stosując te wyrażenia do operatora gęstości $n=|\bm{r}\rangle\langle\bm{r}|$,
możemy wyrazić gęstość elektronową następującym wyrażeniem
%
\begin{equation}
n(\bm{r})=\tilde{n}(\bm{r})+n^1(\bm{r})-\tilde{n}^1(\bm{r}),
\end{equation}
%
gdzie
%
\begin{equation}
\tilde{n}(\bm{r})=\sum_n f_n \langle \tilde{\psi}_n|\bm{r}\rangle\langle \bm{r}| \tilde{\psi}_n\rangle=\sum_n f_n |\tilde{\psi_n}(\bm{r})|^2,
\end{equation}
%
\begin{equation}
n^1(\bm{r})=\sum_{n,i,j} f_n \langle \tilde{\psi}_n|\tilde{p}_i\rangle\langle\phi_i|\bm{r}\rangle\langle \bm{r}|\phi_j\rangle\langle\tilde{p}_j |\tilde{\psi}_n\rangle,
\end{equation}
%
\begin{equation}
\tilde{n}^1(\bm{r})=\sum_{n,i,j} f_n \langle \tilde{\psi}_n|\tilde{p}_i\rangle\langle\tilde{\phi}_i|\bm{r}\rangle\langle \bm{r}|\tilde{\phi}_j\rangle\langle\tilde{p}_j |\tilde{\psi}_n\rangle.
\end{equation}
%
W powyższych wzorach sumowanie obejmuje zarówno stany walencyjne, jak i stany rdzenia. 

Metoda PAW należy obecnie do najdokładniejszych i najbardziej efektywnych metod. Została zaimplementowana
w takich programach, jak Vienna Ab Initio Simulation Package (VASP) \cite{Vasp,PawVasp} i Quantum Espresso \cite{QE}.


\section{Zlokalizowane orbitale}

\subsection{Metoda ciasnego wiązania}

W odróżnieniu od metod pseudopotencjału, które wykorzystują bazę fal płaskich, metody opisane w tym rozdziale
stosują lokalne orbitale, które powiązane są z danymi atomami lub centrowane są w położeniach atomów.
Opis przy pomocy lokalnych orbitali nadaje się dobrze do układów, gdzie występują zlokalizowane stany
elektronowe, ale możliwy jest ruch elektronów poprzez przeskoki do sąsiednich atomów.
Najprostszym opisem takich układów jest model ciasnego wiązania.
Lokalne orbitale atomowe możemy zapisać w postaci $\phi_{\alpha}(\bm{r}-\bm{R}_{\alpha})$, 
gdzie wektor $\bm{R}_{\alpha}$ oznacza położenie danego atomu. Wskaźnik $\alpha$ oznacza tutaj wszystkie liczby kwantowe, 
które charakteryzują dany orbital ($\alpha=n,l,m$). Zakładamy dla uproszczenia, że z danym atomem związny jest tylko
jeden orbital, ale model ciasnego wiązania można łatwo uogólnić na dowolną liczbę orbitali.
Hamiltonian możemy zapisać jako
%
\begin{equation}
H=-\frac{\hbar^2}{2m}\nabla^2+\sum_{\alpha}V_{\alpha}(\bm{r}-\bm{R}_{\alpha}),
\end{equation}
%
gdzie $V_{\alpha}$ jest potencjałem atomowym wokół położenia $\bm{R}_{\alpha}$.
Elementy macierzowe hamiltonianu możemy wyznaczyć ze wzoru
%
\begin{equation}
H_{\alpha\beta}=\int d\bm{r} \phi_{\alpha}^*(\bm{r}-\bm{R}_{\alpha})H\phi_{\beta}(\bm{r}-\bm{R}_{\beta}).
\label{matrix}
\end{equation}
% 
Ponieważ orbitale atomowe należące do różnych atomów nie są wzajemnie ortogonalne
wprowadza się również macierz przekrywania
%
\begin{equation}
S_{\alpha\beta}=\int d\bm{r} \phi_{\alpha}^*(\bm{r}-\bm{R}_{\alpha})\phi_{\beta}(\bm{r}-\bm{R}_{\beta}).
\end{equation}
%
Elementy diagonalne hamiltonianu $\varepsilon_{\alpha}=H_{\alpha\alpha}$ określają lokalną energię elektronu w danym orbitalu $\alpha$.
Natomiast elementy pozadiagonalne $t_{\alpha\beta}=H_{\alpha\beta}$ nazywane są całkami przeskoku i określają prawdopodobieństwo
przeskoku elektronu między dwoma atomami.

Aby przetransformować elementy macierzowe hamiltonianu i macierzy $S$ do przestrzeni odwrotnej
wprowadzamy bazę, która odpowiada wektorom falowym $\bm{k}$
%
\begin{equation}
\phi_{\bm{k}\alpha}(\bm{r})=A_{\alpha\bm{k}}\sum_n e^{i\bm{k}\bm{T}_n}\phi_{\alpha}[\bm{r}-(\bm{R}_{\alpha}+\bm{T}_n)],
\end{equation}
%
gdzie $A_{\bm{k}\alpha}$ są czynnikami normalizacji, a $\bm{T}_n$ są wektorami translacji sieci krystalicznej.
W tej bazie możemy zapisać funkcję falową, która spełnia twierdzenie Blocha
%
\begin{equation}
\psi_{\bm{k}i}=\sum_{\alpha} c_{i\alpha} \phi_{\bm{k}\alpha}(\bm{r}),
\end{equation}
%
oraz elementy macierzowe hamiltonianu i macierzy $S$
%
\begin{equation}
H_{\alpha\beta}(\bm{k})=\int d\bm{r} \phi_{\alpha\bm{k}}^*(\bm{r})H\phi_{\beta\bm{k}}(\bm{r})=\sum_n e^{i\bm{k}\bm{T}_n} H_{\alpha\beta},
\end{equation}
%
\begin{equation}
S_{\alpha\beta}(\bm{k})=\int d\bm{r} \phi_{\alpha\bm{k}}^*(\bm{r})\phi_{\beta\bm{k}}(\bm{r})=\sum_n e^{i\bm{k}\bm{T}_n} S_{\alpha\beta}.
\end{equation}
%
Równanie na energie własne $\varepsilon_i(\bm{k})$ i współczynniki rozwinięcia funkcji falowej $c_{i,\alpha}(\bm{k})$ przyjmuje postać
%
\begin{equation}
\sum_{\beta}[H_{\alpha\beta}(\bm{k})-\varepsilon_i(\bm{k})S_{\alpha\beta}(\bm{k})]c_{i\beta}(\bm{k})=0.
\label{ham_loc} 
\end{equation}
%
Dla orbitali wzajemnie ortogonalnych, mamy $S_{\alpha\beta}=\delta_{\alpha\beta}$ i równanie (\ref{ham_loc})
przyjmuje postać taką samą jak równanie w bazie fal płaskich (\ref{ham_rec}).

\subsection{Funkcje Wanniera}
\label{sec:wannier}

W metodzie ciasnego wiązania często wykorzystuje się bazę funkcji Waniera, które są zlokalizowanymi funkcjami, 
centrowanymi na położeniach atomowych $\bm{R}_n$. 
Funkcja Wanniera zdefiniowana jest jako transformata Fouriera funkcji Blocha powiązanej z danym pasmem $j$
%
\begin{equation}
w_j(\bm{r}-\bm{R}_n)=\frac{1}{\sqrt{N}}\sum_{\bm{k}}e^{-i\bm{k}\bm{R}_n}\psi_{\bm{k}j}(\bm{r}). 
\end{equation}
%
Spełniona jest również transformata odwrotna, która pozwala wyrazić funkcje Blocha przez funkcje
Wanniera
%
\begin{equation}
\psi_{\bm{k}j}(\bm{r})=\frac{1}{\sqrt{N}}\sum_n e^{i\bm{k}\bm{R}_n} w_j(\bm{r}-\bm{R}_n).
\end{equation}
%
Funkcje Wanniera dla różnych położeń atomowych są ortogonalne
%
\begin{multline}
\int d\bm{r} w_j^*(\bm{r}-\bm{R}_n)w_j(\bm{r}-\bm{R}_m)=\frac{1}{N}\sum_{\bm{k},\bm{k}'}\int d\bm{r}e^{i(\bm{k}\bm{R}_n-\bm{k}'\bm{R}_m)}\psi_{\bm{k}j}(\bm{r})\psi_{\bm{k}'j}(\bm{r}) \\
=\frac{1}{N}\sum_{\bm{k}}e^{i\bm{k}(\bm{R}_n-\bm{R}_m)}=\delta_{mn},
\end{multline}
%
gdzie wykorzystaliśmy ortogonalność funkcji Blocha. Spełniony jest również warunek zupełności
%
\begin{equation}
\sum_n w_j^*(\bm{r}-\bm{R}_n)w_j(\bm{r}'-\bm{R}_n)=\delta^3(\bm{r}-\bm{r}').
\end{equation}
Zatem funkcje Wanniera tworzą zbiór zlokalizowanych funkcji bazowych.
Energie pasma $j$ dla funkcji Blocha w tej bazie może być wyznaczona ze wzoru
%
\begin{equation}
\varepsilon_{\bm{k}j}=\int d\bm{r} \psi_{\bm{k}j}^*(\bm{r})H\psi_{\bm{k}j}(\bm{r})
=\frac{1}{N}\int d\bm{r}\sum_n w_j^*(\bm{r}-\bm{R}_n)e^{-i\bm{k}\bm{R}_n}H\sum_m w_j(\bm{r}-\bm{R}_m)e^{i\bm{k}\bm{R}_m}.
\end{equation}
%
Sumowanie po wskaźnikach $n$ i $m$ możemy rozdzielić na dwa wyrazy dla $n=m$ i $n\neq m$
%
\begin{multline}
\varepsilon_{\bm{k}j}=\frac{1}{N}\sum_n\int d\bm{r} w_j^*(\bm{r}-\bm{R}_n)Hw_j(\bm{r}-\bm{R}_n)
\\ +\frac{1}{N}\sum_{n\neq m}e^{i\bm{k}(\bm{R}_m-\bm{R}_n)}\int d\bm{r} w_j^*(\bm{r}-\bm{R}_n)Hw_j(\bm{r}-\bm{R}_m).
\end{multline}
%
Wykorzystując oznaczenia, które wprowadziliśmy dla elementów macierzowych hamiltonianu (\ref{matrix})
dostajemy
%
\begin{equation}
\varepsilon_{\bm{k}j}=\frac{1}{N}\sum_n \varepsilon_n+\frac{1}{N}\sum_{n\neq m}e^{i\bm{k}(\bm{R}_m-\bm{R}_n)}t_{nm}.
\end{equation}
%
Wzór ten pozwala wyznaczyć strukturę pasmową w ramach metody ciasnego wiązania.
Jeżeli ograniczymy się do modelu jednopasmowego i do przeskoków tylko między najbliższymi sąsiadami
możemy dostać uproszczony model z jednym parametrem przeskoku $t$.
Dla prostej struktury kubicznej dostajemy
%
\begin{equation}
\varepsilon_{\bm{k}j}=\varepsilon_0+\frac{t}{N}\sum_{n}e^{i\bm{k}(\bm{R}_{n+1}-\bm{R}_n)}=\varepsilon_0+2t(cosk_xa+cosk_ya+cosk_za).
\label{band}
\end{equation}
%
gdzie $a$ jest stałą sieci, a $\varepsilon_0$ jest energią elektronu zlokalizowanego w stanie atomowym.
Szerokość pasma w tym przybliżeniu dana jest wzorem $w=2z|t|$, gdzie $z$ jest liczbą najbliższych sąsiadów.  

Metoda ciasnego wiązania, która wykorzystuje bazę orbitali atomowych lub funkcji Wanniera,
często wykorzystywana jest w obliczeniach modelowych. W ramach tego podejścia można łatwo uwzględnić lokalne oddziaływania
kulombowskie lub oddziaływania parujące w modelach nadprzewodników wysokotemperaturowych.
Lokalne energie i całki przeskoku mogą być wyznaczone przez dofitowanie struktury elektronowej
z modelu ciasnego wiązania do wyników eksperymentalnych lub do struktury pasmowej wyznaczonej z obliczeń DFT.
W tym drugim przypadku stosuje się najczęściej bazę maksymalnie zlokalizowanych funkcji Wanniera.
Wykorzystuje się swobodę wyboru fazy dla funkcji Blocha, którą możemy przemnożyć przez czynnik $e^{i\theta(\bm{k})}$,
gdzie $\theta(\bm{k})$ jest dowolną funkcją rzeczywistą, bez wpływu na energie stanów elektronowych. 
Taki czynnik ma jednak wpływ na kształt funkcji Wanniera, w szczególności na ich zasięg przestrzenny. 
Można tak dobrać czynnik fazowy, aby funkcja Wanniera $w_i(\bm{r}-\bm{R}_n)$ zlokalizowana była
wokół punktu $\bm{R}_n$ i szybko zanikała z odległością. 

\begin{figure}[h]
\centering
\includegraphics[scale=1]{FeSe.pdf}
\caption{Porównanie struktury pasmowej otrzymanej w ramach modelu ciasnego wiązania i obliczeń DFT dla {\bf a.} CeCoIn$_5$ i {\bf b.} FeSe.
Rysunek z pracy \cite{FeSe}.}
\label{fig:fese}
\end{figure}

Na rysunku \ref{fig:fese} pokazano przykłady struktur pasmowych dla dwóch związków, CeCoIn$_5$ i FeSe, wyliczonych w modelu ciasnego wiązania z bazą
maksymalnie zlokalizowanych funkcji Wanniera. 
Parametry modelu wyznaczono przez dofitowanie energii stanów elektronowych, dla każdego wektora falowego $\bm{k}$, 
do struktur pasmowych otrzymanych w ramach obliczeń DFT \cite{FeSe}.
Obliczenia wykonano programem Quantum Espresso \cite{QE}, stosując funkcjonał GGA \cite{PW91} w ramach metody PAW \cite{Blochl}.  
Wyznaczone modele ciasnego wiązania zostały następnie wykorzystane do wyliczenia podatności tworzenia par Coopera
i zbadania własności nadprzewodzących obydwóch materiałów \cite{FeSe}.
 

\subsection{Funkcje Gaussa}

Lokalne orbitale można również wykorzystać jako bazę do rozwiniecia funkcji falowej w ramach samozgodnej procedury rozwiązywania równań Kohna-Shama.
Jeżeli funkcjami bazowymi są orbitale atomowe mówimy o metodzie liniowych kombinacji orbitali atomowych ({\it ang. linear combination
of atomic orbitals} - LCAO). Zwykle jednak stosuje się atomo-podobne orbitale, których postać funkcyjna ułatwia implementację
i przyspiesza obliczenia. Do najczęściej stosowanych należą orbitale typu Slatera \cite{STO1,STO2} i funkcje typu Gaussa (gaussiany) \cite{GTO}.
Te drugie są szczególnie wygodne ponieważ całki z funkcji Gaussa można wyliczyć analitycznie.
Naturalną reprezentacją dla gaussianów jest układ współrzędnych biegunowych, ale wygodniejszy do wyznaczania elementów macierzowych hamiltonianu jest układ kartezjański, w którym te funkcje mają postać
%
\begin{equation}
G_{ijk}(\bm{r},\alpha,\bm{R}_n)=N(x-x_n)^i(y-y_n)^j(z-z_n)^ke^{-\alpha (\bm{r}-\bm{R}_n)^2},
\end{equation}
%
gdzie $\alpha$ jest parametrem wariacyjnym, który umożliwia optymalizację bazy dla konkretnych atomów, $\bm{R}_n=(x_n,y_n,z_n)$ jest wektorem określającym centrowanie danej funkcji, a $N$ jest czynnikiem normalizacyjnym.
Spełniona jest ponadto zależność $l=i+j+k$, gdzie $l$ jest orbitalną liczbą kwantową.
W wyniku iloczynu dwóch fukcji typu $s$ ($l=0$) centrowanych w punktach $\bm{R}_n$ i $\bm{R}_m$ dostajemy
%
\begin{equation}
G_s(\bm{r},\alpha,\bm{R}_n)G_s(\bm{r},\beta,\bm{R}_m)=e^{-\gamma(\bm{R}_n-\bm{R}_m)^2}G_s(\bm{r},\kappa,\bm{R}_p),
\end{equation}
%
gdzie 
%
\begin{equation}
\kappa=\alpha+\beta, \quad \gamma=\frac{\alpha\beta}{\alpha+\beta}, \quad \bm{R}_p=\frac{\alpha\bm{R}_n+\beta\bm{R}_m}{\alpha+\beta},
\end{equation}
%
gdzie $\bm{R}_p$ jest wektorem centrowania wynikowej funkcji Gaussa.
W ogólnym przypadku iloczyn dwóch gaussianów można zapisać w formie
%
\begin{multline}
G_{i_1j_1k_1}(\bm{r},\alpha,\bm{R}_n)G_{i_2j_2k_2}(\bm{r},\beta,\bm{R}_m) = Ne^{-\gamma(\bm{R}_n-\bm{R}_m)^2}e^{-\kappa(\bm{r}-\bm{R}_p)^2} \\
\times (x-x_n)^{i_1}(x-x_m)^{i_2}(y-y_n)^{j_1}(y-y_m)^{j_2}(z-z_n)^{k_1}(z-z_m)^{k_2}.
\end{multline}
%
Te wzory pokazują, ze iloczyn funkcji Gaussa jest również funkcją Gaussa. Jest to główna zaleta tej bazy umożliwiająca
analityczne wyliczenie złożonych całek zawierajacych kilka funkcji bazowych i w ten sposób znaczne przyspieszenie rachunków.
W odróżnieniu od metod wykorzystujących fale płaskie, metoda funkcji Gaussa wymaga odpowiedniego wybrania funkcji bazowych
dla danego układu atomowego. Metoda ta stosowana jest głównie przez chemików, a najbardziej popularnym programem jest GAUSSIAN.
Głównym twórca tego programu jest John Pople, który w roku 1998 otrzymał wspólnie z Walterem Kohnem Nagrodę Nobla w dziedzinie chemii.

\section{Stowarzyszone fale i atomowe sfery}

Ostatnia grupa metod obliczeniowych łączy główne cechy dwóch poprzednich: pseudopotencjałów i lokalnych orbitali. 
Do rozwinięcia funkcji falowych stosowane są dwie różne bazy w zależności od położenia w krysztale.
Wokół rdzeni atomowych naturalnym wyborem sa lokalne funkcje, które można otrzymać z rozwiązania
równania Schr\"{o}dingera w potencjale sferyczno-symetrycznym. W obszarze miedzywezłowym
stosuje się rozwinięcie w bazie funkcji, które umożliwiają szybką zbieżność rachunków (np. fale płaskie).
Centralnym problemem stają sie odpowiednie warunki brzegowe, które zapewniają ciągłość funkcji falowej i jej pochodnej 
na granicy między tymi obszarami.
Wszystkie metody omawiane w tym rozdziale stosowały na początku uproszczony potencjał typu {\it muffin-tin}, który w obszarze rdzenia
ma charakter potencjału atomowego, a w obszarze międzywęzłowym jego wartość jest stała.
Obecnie podejścia te zostały zaimplementowane w ramach procedury samozgodnej i zaliczane są do metod pełnego potencjału. 

\subsection{Metoda KKR}

W dwóch pracach, Korringa \cite{K47} oraz Kohn i Rostoker \cite{KR54} (KKR) zaproponowali metodę, która nazywana jest również metodą funcji Greena
lub metodą wielokrotnego rozpraszania.
Pierwsze obliczenia stałych sieci i modułów sztywności wykonano właśnie z wykorzystaniem tej metody \cite{Moruzzi77}. Była to przełomowa praca, która 
pokazała możliwość zastosowania DFT do badania własności materiałowych.  

Podstawową wielkością stosowaną tutaj jest funkcja Greena $G(E,\bm{r},\bm{r}')$, która opisuje propagację elektronu o energii $E$ między punktami $\bm{r}$ i $\bm{r}'$. Proces przemieszczania się elektronu można podzielić na swobodny ruch w obszarze międzywęzłowym i rozpraszanie na sferach atomowych
w wyniku oddziaływania elektronu z potencjałem rdzenia atomowego.
Dla elektronu opisanego jednocząstkowym równaniem Schr\"{o}dingera
%
\begin{equation}
[-\frac{\hbar^2}{2m}\nabla^2+V(\bm{r})]\psi_i(\bm{r})=\varepsilon_i\psi_i(\bm{r}),
\end{equation}
%
funkcja Greena spełnia równanie
%
\begin{equation}
[-\frac{\hbar^2}{2m}\nabla^2+V(\bm{r})-E]G(E,\bm{r}-\bm{r}')=\delta(\bm{r}-\bm{r}'),
\end{equation}
%
które posiada rozwiązanie w formie
%
\begin{equation}
G(E,\bm{r},\bm{r}')=\sum_i\frac{\psi_i(\bm{r})\psi_i^*(\bm{r}')}{E-\varepsilon_i}.
\end{equation}
%
Dla nieoddziałujących elektronów mamy $V(\bm{r})=0$ i funkcja Greena ma postać
%
\begin{equation}
G_0(E,|\bm{r}-\bm{r}'|)= \frac{m}{2\pi\hbar^2}\frac{e^{ik_0|\bm{r}-\bm{r}'|}}{|\bm{r}-\bm{r}'|},
\end{equation}
%
gdzie $k_0=\frac{\sqrt{2mE}}{\hbar}$.

Funkcja Greena dla elektronu oddziałującego z siecią atomową może być zapisana w postaci rozwinięcia
%
\begin{equation}
G=G_0+G_0tG_0+G_0tG_0tG_0+...=G_0+G_0tG,
\end{equation}
%
które prowadzi do zależności
%
\begin{equation}
G=(G_0^{-1}-t)^{-1},
\end{equation}
%
gdzie $t$ jest macierzą rozpraszania elektronów na pojedynczych atomach sieci krystalicznej.
Można wprowadzić również macierz całkowitego, wielokrotnego rozpraszania elektronów $T$, która zdefiniowana jest wzorem
%
\begin{equation}
G=G_0+G_0TG_0=G_0+G_0(t+tG_0T)G_0,
\end{equation}
%
z którego dostajemy
%
\begin{equation}
T=(t^{-1}-G_0)^{-1}.
\end{equation}
%
Stany stacjonarne, które odpowiadają rozwiązaniom równania Schr\'{o}dingera, są biegunami macierzy $T$
i mogą być wyznaczone jako miejsca zerowe wyznacznika
%
\begin{equation}
\det(t^{-1}-G_0)=0.
\end{equation}
%
Równanie to dostarcza rozwiązania problemu wielokrotnego rozpraszania elektronów w przybliżeniu potencjału {\it muffin-tin}.
Pełna informacja o rozpraszaniu elektronu na pojedynczej sferze atomowej zawarta jest w macierzy rozpraszania $t_l(E)$, 
która zależy tylko od energii stanu elektronowego $E$ i orbitalnego krętu $l$.
Korzystając z teorii rozpraszania, można wyrazić macierz rozpraszania jako funkcję przesunięcia fazy funkcji falowej $\eta_l(E)$
%
\begin{equation}
t_l(E)=-\frac{1}{k_0}e^{i\eta_l(E)}\sin(\eta_l(E)).
\label{fshift}
\end{equation} 
%
Funkcja Greena $G_0$ zależy tylko od geometrii sieci krystalicznej i energii $E$.
Wykorzystując rozwinięcie fal płaskich w bazie harmonik sferycznych, funcję Greena
możemy zapisać w formie
%
\begin{equation}
G_0(E,|\bm{r}-\bm{r}'|)=\sum_{L,L'} i^lj_l(k_0r)Y_L(\hat{\bm{r}})B_{LL'}(E,\bm{R}-\bm{R}')(-i)^{l'}j_{l'}(k_0r')Y^*_{L'}(\hat{\bm{r}}'),
\end{equation} 
%
gdzie $j_l(k_0r)$ są funkcjami Bessela, $B_{LL'}$ są stałymi struktury KKR i $L=\{l,m\}$.
Wektory położenia $\bm{r}$ i $\bm{r}'$ odnoszą się do dwóch sfer atomowych, których środki
określają wektory $\bm{R}$ i $\bm{R}'$. Przy założeniu, że rozpraszanie w każdym węźle atomowym jest takie samo,
co odpowiada sieci złożonej z jednego rodzaju atomów, możemy wprowadzić zależność stałych struktury od wektora falowego
% 
\begin{equation}
B_{LL'}(E_{\bm{k}},\bm{k})=\sum_{m} B_{LL'}(E_{\bm{k}},\bm{T}_m)e^{-i\bm{k}\bm{T}_m},
\end{equation}
%
gdzie $\bm{T}_m$ są wektorami translacji sieci krystalicznej, a energie $E_{\bm{k}}$
są rozwiązaniami równania
%
\begin{equation}
\det[t_l^{-1}(E_{\bm{k}})\delta_{LL'}-B_{LL'}(E_{\bm{k}},\bm{k})]=0.
\end{equation}
%
Korzystając z wyrażenia (\ref{fshift}) dostajemy podstawowe równania na strukturę pasmową w metodzie KKR
%
\begin{equation}
\sum_{L'}[B_{LL'}(E_{\bm{k}},\bm{k})+k_0\cot(\eta_l(E_{\bm{k}}))\delta_{LL'}]a_{L'}(\bm{k})=0.
\label{KKR}
\end{equation}
%
Rozwiązania tego równania pozwalają nam wyznaczyć funkcje falowe wewnątrz sfer atomowych
%
\begin{equation}
\psi_{\bm{k}}(\bm{r})=\sum_{L'}a_{L'}(\bm{k})u_l(r,E_{\bm{k}})Y_L(\hat{\bm{r}}),
\end{equation}
%
gdzie $u_l(r,E_{\bm{k}})$ są rozwiązaniami równania radialnego wewnątrz sfery, umówionego dokładnie w następnym rozdziale.
Funkcje falowe w obszarze międzywęzłowym są rozwiązaniami następującego równania całkowego
%
\begin{equation}
\psi_{\bm{k}}(\bm{r})=-\int d\bm{r}'G_0(E_{\bm{k}},|\bm{r}-\bm{r}'|)V(\bm{r}')\psi_{\bm{k}}(\bm{r}').
\end{equation} 
%
Funkcje falowe w obydwóch obszarach muszą spełniać warunki brzegowe na każdej sferze atomowej. 

W ogólnym przypadku macierz rozpraszania zależy od położenia atomu $t_l(E,\bm{R})$
i równanie na funkcje Greena przyjmuje postać
%
\begin{equation}
[G_{LL'}(E,\bm{R},\bm{R}')]^{-1}=\Big[[B_{LL'}(E,\bm{R}-\bm{R}')]^{-1}-t_l(E,\bm{R})\delta_{\bm{R}\bm{R}'}\delta_{LL'}]\Big],
\end{equation}
%
a stany stacjonarne otrzymujemy z warunku
%
\begin{equation}
\det\Big[t_l^{-1}(E,\bm{R})\delta_{\bm{R}\bm{R}'}\delta_{LL'}-B_{LL'}(E,\bm{R}-\bm{R}')\Big]=0.
\end{equation}

Funkcja Greena pozwala wyznaczyć różne wielkości fizycznych. Na przykład, część urojona funkcji Greena zwiazana jest z lokalną gęstością stanów 
%
\begin{equation}
n_L(E,\bm{R})=-\frac{1}{\pi}\operatorname{Im}G_{LL}(E+i\delta,\bm{R}).
\end{equation}
%
Podobnie część urojona transformaty Fouriera funkcji Green daje nam gęstość spektralną,
czyli gęstość stanów elektronowych dla danej energii i wektora falowego
%
\begin{equation}
n_L(E,\bm{k})=-\frac{1}{\pi}\operatorname{Im}G_{LL}(E+i\delta,\bm{k}).
\end{equation}
%

Metoda funkcji Greena umożliwia wprowadzenie efektywnego (koherentnego) potencjału, który łączy własności dwóch różnych atomów.
Jest to przybliżenie koherentnego potencjału ({\it ang. coherent potential approximation} - CPA), które stosowane jest do badania
stopów metalicznych lub układów domieszkowanych \cite{CPA1,CPA2}.
Metoda polega na uśrednieniu własności rozpraszania dwóch atomów umieszczonych w pobliżu efektywnego potencjału.
Warunek równoważności średniej ważonej po obydwóch atomach i efektywnego potencjału prowadzi do zespolonego,
zależnego od energii potencjału CPA. 


\subsection{Metoda LAPW}

Metoda zlinearyzowanych stowarzyszonych fal płaskich ({\it ang. linearized augmented plane wave} - LAPW)
jest modyfikacją oryginalnego podejścia rozszerzonych fal płaskich (APW) zaproponowanego przez Slatera w roku 1937 \cite{slater37}.
Podobnie jak w metodzie pseudopotencjału, cały kryształ podzielony jest na obszary wewnątrz sfer otaczających atomy ($r<r_c$)
i obszar międzywęzłowy ($r>r_c$). Funkcje falowe w otoczeniu jądra atomowego charakteryzują sie dużą zmiennością i
w przybliżeniu maja kształt sferyczny. Z tego względu naturalnym wyborem bazy sa funkcje radialne, będące rozwiązaniami równania Schr\"{o}dingera z
potencjałem sferycznie symetrycznym. W obszarze więdzywęzłowym, gdzie potencjał zmienia sie bardzo wolno, rozwiązania
funkcji falowej rozwijane są w bazie fal płaskich. Funkcję falową w metodzie APW można
zapisać w postaci
%
\begin{equation}
\psi_{\bm{k}j}(\bm{r})=\sum_{\bm{G}}c_{\bm{G}j}\phi_{\bm{k}+\bm{G}}(\bm{r}),
\end{equation}
%
gdzie $\bm{G}$ oznaczają wektory sieci odwrotnej. Funkcje bazowe mogą być zapisane w formie 
%
\begin{equation} 
\phi_{\bm{k}+\bm{G}}(\bm{r})=\begin{cases}
        e^{i(\bm{k}+\bm{G})\bm{r}} & r>r_c \\
       \sum_{lm}A_{lm}(\bm{k}+\bm{G})u_l(r,E_l)Y_{lm}(\theta,\varphi) &  r<r_c,
             \end{cases}    
\label{APWbase}                      
\end{equation}
%
gdzie $A_{lm}$ są współczynnikami rozwinięcia, $Y_{lm}(\theta,\varphi)$ są harmonikami sferycznymi i $u_l(r,E_l)$ są rozwiązaniami równania
%
\begin{equation}
[-\frac{d^2}{dx^2}+\frac{l(l+1)}{r^2}+V(r)-E_l]ru_l(r,E_l)=0,
\label{radial}
\end{equation}
%
gdzie $V(r)$ jest potencjałem wewnątrz sfery, a $E_l$ pełni rolę parametru i niekoniecznie jest to energia własna. 
W przybliżeniu potencjału {\it muffin-tin} (MT), w obszarze atomowym jest on sferycznie
symetryczny ($V(\bm{r})=V(r)$), a w obszarze międzywezłowym ma wartość stałą ($V(\bm{r})=V_0$) lub zero. Wtedy funkcje falowe $\psi(\bm{r})$ mogą być wyrażone przez rozwiazania równania Schr\"{o}dingera w poszczególnych obszarach kryształu, odpowiednio przez harmoniki sferyczne wokół atomów i fale płaskie między atomami. 
Współczynniki $A_{lm}$ wyznacza się tak, aby spełniony był warunek ciągłości funkcji falowej na granicy sfery (warunek ciągłości pierwszych pochodnych nie jest spełniony). w ten sposób współczynniki $A_{lm}$ stają się zależne od vectora falowego.  
Współczynniki $c_{\bm{G}}$ i liczby $E_l$ są parametrami wariacyjnymi metody APW. Rozwiązania, które numerowane są wektorami $\bm{G}$, składające sie z pojedynczych fal płaskich w obszarze miedzywęzłowym i dopasowane do nich rozwiązania radialne wewnątrz sfery nazywane są stowarzyszonymi falami płaskimi.

Funkcje bazowe wewnątrz sfery zależą od energii $E_l$, co powoduje, że odpowiednie równanie własne na funkcje falowe jest nieliniowe. Jest to główny problem
podejścia APW, uniemożliwiający łatwe rozszerzenie stosowalności tej metody na dowolne potencjały. W szczególności potencjały wyznaczane metodą samozgodną w ramach teorii funkcjonału gęstość.
Rozwiązaniem tego problemu była zaproponowana przez Andersena w roku 1975 linearyzacja równań APW, która prowadzi do metody LAPW \cite{Andersen75}.
W tym podejściu, modyfikuje się funkcje bazowe wewnątrz sfery do postaci
%
\begin{equation}
\phi_{\bm{k}+\bm{G}}(\bm{r})=\sum_{lm}[A_{lm}(\bm{k}+\bm{G})u_l(r,E_l)Y_{lm}(\theta,\varphi)+B_{lm}(\bm{k}+\bm{G})\dot{u}_l(r,E_l)Y_{lm}(\theta,\varphi)],
\end{equation}    
%
gdzie współczynniki rozwinięcia $A_{lm}$ i $B_{lm}$ wyznacza się z warunku ciągłości funkcji falowej i jej pochodnej na sferze, a pochodna $\dot{u}_l(r,E_l)=\partial u_l/\partial E$ spełnia równanie
%
\begin{equation}
[-\frac{d^2}{dx^2}+\frac{l(l+1)}{r^2}+V(r)-E_l]r\dot{u}_l(r,E_l)=ru_l(r,E_l).
\end{equation}
Dodatkowo narzuca sie warunek normalizacji funkcji $u_l$
%
\begin{equation}
\int_0^{r_c}dr[ru_l(r,E_l)]^2=1,
\end{equation}
%
oraz ortogonalności funkcji $u_l$ i $\dot{u}_l$
%
\begin{equation}
\int_0^{r_c}drr^2u_l(r,E_l)\dot{u}_l(r.E_l)=0.
\end{equation}

Znajomość gradientu $\dot{u}_l$ umożliwia wyznaczenie funkcji radialnej dla danej energii pasma $\varepsilon$, 
uwzlędniając poprawkę liniową
%
\begin{equation}
u_l(r,\varepsilon)=u_l(r,E_l)+(\varepsilon-E_l)\dot{u}_l(r,E_l)+O((\varepsilon-E_l)^2),
\end{equation}
%
gdzie $O((\varepsilon-E_l)^2)$ oznacza błąd, który jest rzędu kwadratu różnicy tych dwóch energii.

W ramach procedury samozgodnej, pełny potencjał wyliczany jest w obydwóch obszarach kryształu,
stosując rozwinięcia w odpowiednich bazach
%
\begin{equation} 
V(\bm{r})=\begin{cases}
        \sum_{\rm{G}}V_{\bm{G}} e^{i\bm{G}\bm{r}} & r>r_c \\
       \sum_{lm}V_{lm}Y_{lm}(\theta,\phi) &  r<r_c.
             \end{cases}   
\label{fullpot}                       
\end{equation}
%
Podobnie wylicza się gęstość ładunku wewnątrz sfery korzystając ze współczynników rozwinięcia funkcji falowych
w bazie harmonic sferycznych oraz w obszarze międzywęzłowym w bazie fal płaskich.

W odróżnieniu od metody pseudopetcjału, gdzie uwzględniane są tylko elektrony walencyjne, w metodzie LAPW wyznaczane są
zarówno stany walencyjne, jak i stany rdzenia.
Ponieważ funkcje falowe stanów rdzenia sa zlokalizowane blisko jądra atomowego, wylicza się je stosując tylko 
potencjał centrosymetryczny. Jeżeli zasięg funkcji falowej stanu rdzenia przekracza granicę strefy atomowej,
stosuje się gładkie przedłużenie potencjału centrosymetrycznego poza tę granice. 
Do wyznaczenia stanów rdzenia stosuje sie podejście w pełni relatywistyczne,
czyli rozwiązuje się równanie Diraca, zamiast równania Schr\"{o}dingera (\ref{radial}). 
Stany walencyjne wyznaczone są dla pełnego potencjału w całym obszarze kryształu.

W niektórych materiałach występują również stany elektronowe, mające wspólne cechy stanów rdzenia i walencyjnych, które nazywane są
pseudordzeniowymi. Przykładem są wysoko położone i wzlędnie rozległe stany $5d$ w metalach ziem rzadkich.
Występowanie tych stanów wymaga modyfikacji metody LAPW. W standardowej wersji, baza konstruowana
jest tak, aby jak najdokładniej opisać stany elektronowe bliskie energii $E_l$
i zwykle wybiera się tę wartość blisko środka pasma walencyjnego.
Jeżeli występują dodatkowe stany w innym zakresie energii, wybór wartości $E_l$ nie jest oczywisty. 
Najlepszym podejściem do opisu stanów pseudordzeniowych jest rozszerzenie bazy przez wprowadzenie dodatkowych 
lokalnych orbitali (LO), które mieszczą się wewnątrz atomowej sfery. Ta modyfikacja nazywa się metodą APW+LO \cite{singh91}. 
Ponieważ istnieje swoboda wyboru bazy, pochodną funkcji radialnej $\dot{u}_l$ usuwa sie z głównej bazy i dołacza do LO.
Zatem całkowita baza w tym podejściu składa się z oryginalnej bazy APW (\ref{APWbase})  
i dodatkowych orbitali w formie
%
\begin{equation}
\chi^{lo}_L(\bm{r})=[a^{lo}_{lm}u_l(r,E_l)+b^{lo}_{lm}\dot{u}_l(r,E_l)]Y(\theta,\varphi),
\end{equation}
%
gdzie $r<r_c$, a współczynniki $a^{lo}_{lm}$ i $b^{lo}_{lm}$ są tak dobrane, aby LO
znikały na sferze. Liczba orbitalna ograniczona jest w tym przypadku do wartości, które odpowiadają 
fizycznym stanom kwantowym ($l\le 3$). W odróżnieniu od LAPW, w metodzie APW+LO nie ma ograniczenia na
wartości pochodnych funkcji bazowych na sferze atomowej. 
  

\subsection{Metoda LMTO}

W metodzie zlinearyzowanych orbitali {\it muffin-tin} ({\it ang. linearized muffin-tin orbitals} - LMTO)
również wprowadza się podział kryształu na obszary wewnątrz sfer atomowych i obszar miedzywęzłowy.
Funkcja falowa odpowiadająca energii $\varepsilon$ może być zapisana w ogólnej formie 
%
\begin{equation}
\psi_{\bm{k}}(\varepsilon,\bm{r})=\sum_{lm} a_{lm}(\bm{k}) \phi_{lm}(\varepsilon,\kappa,\bm{r}),
\end{equation}
%
gdzie funkcje bazowe w poszczególnych obszarach mają postać
%
\begin{equation}
\phi_{lm}(\varepsilon,\kappa,\bm{r})=i^lY_{lm}(\theta,\phi)\begin{cases}
        u_l(\varepsilon,r) +\kappa cot(\eta_l(\varepsilon))J_l(\kappa,r) & r<r_c \\
       \kappa N_l(\kappa,r) &  r>r_c.
             \end{cases}    
\end{equation}
%
Funkcja $J_l(\kappa,r)$ jest tak skonstruowana, aby funkcje bazowe wewnątrz sfery nie zależały od energii, 
czyli spełnione było równanie
%
\begin{equation}
\frac{d}{d\varepsilon}\phi_{lm}(\varepsilon,\kappa,\bm{r})=i^lY_{lm}[\dot{u}_l(\varepsilon,r) +\kappa\frac{d}{d\varepsilon} cot(\eta_l(\varepsilon))J_l(\kappa,r)]=0
\end{equation}
%
dla energii $\varepsilon=E_{\nu}$ odpowiadającej średniej energii danego orbitalu LMTO.
Ten warunek prowadzi do wzoru
%
\begin{equation}
J_l(\kappa,r)=-\frac{\dot{u}_l(E_{\nu},r)}{\kappa\frac{d}{d\varepsilon}cot(\eta_l(E_{\nu}))}.
\end{equation}
%
W obszarze międzywęzłowym funkcje bazowe centrowane w położeniu danej sfery $\bm{R}$ otrzymuje się przez rozwinięcie na funkcji $J_l$
powiązane z sąsiednimi sferami w położeniach $\bm{R}'$
%
\begin{equation}
N_l(\kappa,\bm{r}-\bm{R})=4\pi\sum_{l',l''}C_{l',l'',l'''}n^*_{l''}(\kappa,\bm{R}-\bm{R'})J_{l'}(\kappa,\bm{r}-\bm{R}').
\end{equation}
%
Taka konstrukcja powoduje, że funkcje LMTO są kombinacją liniową funkcji $u_l$ i $\dot{u}_l$ wewnątrz danej sfery
i są gładko przedłużane do obszaru międzywęzłowego, łącząc się w sposób ciagły z funkcjami $\dot{u}_l$ z każdej
sąsiedniej sfery.

Metodę LMTO można uprościć poprzez pominięcie obszaru międzywęzłowego i ograniczenie się do optymalizacji funkcji falowych tylko wewnątrz sfer.
W podejściu nazywanym przybliżeniem atomowych sfer ({\it ang. atomic sphere approximation} - ASA)
stosuje się sfery, które częściowo na siebie nachodzą i ich sumaryczna objętość równa się całkowitej objetości kryształu. 
Zastoswanie takiego podejścia ma uzasadnienie jedynie dla kryształów z gęstym upakowaniem atomów.

\chapter{Funkcjonały i potencjały zależne od orbitali}

\section{Problemy lokalnych funkcjonałów}
\label{sec:mgga}

Teoria funkcjonału gęstości odniosła wiele spektakularnych sukcesów w badaniach własności ciał stałych.
Posiada jednak braki, które nie pozwalają opisać poprawnie struktury elektronowej wielu materiałów.
Te problemy wynikają zasadniczo z przybliżeń stosowanych do wyliczenia energii wymienno-korelacyjnej, 
której dokładna postać funkcjonalna nie jest znana.  
Główną zaletą przybliżeń stosowanych w DFT, pozwalająca na obliczenia nawet dla dużych układów, jest lokalny lub kwazilokalny 
charakter potencjału wymienno-korelacyjnego. Dokładny potencjał jest zwykle nielokalny i zależny jest od funkcji falowych (orbitali), a nie tylko gęstości elektronowej. Przykładem jest dokładna energia oddziaływania wymiennego, która wyliczana jest z orbitali jednocząstkowych w przybliżeniu Hartree-Focka. 
Standardowe funkcjonały wymienno-korelacyjnych stosowane w DFT nie zależą od rodzaju obsadzonych orbitali. 
Mówiąc ściślej, nie odróżniają stanów o różnej liczbie kwantowej $m$.
Wszystkie orbitale traktowane są jednakowo i nie ma możliwości wyróżnienia cech, które decydują o nierównomiernym ich obsadzeniu.
Taka możliwość często decyduje o charakterze stanu podstawowego i jest szczególnie istotna w układach silnie skorelowanych. 

W przybliżeniach LDA i GGA występują dwa główne problemy, które są ze sobą ściśle powiązane.
Pierwszym z nich jest brak nieciągłości pochodnej funkcjonalnej energii całkowitej po gęstości elektronowej, która jest równa potencjałowi chemicznemu. 
Ta nieciągłość jest cechą charakterystyczną funkcjonału wymienno-korelacyjnym, ale występuje również w wyrazie opisującym energię kinetyczną elektronów. 
Ponieważ nieciągłość potencjału chemicznego powiązana jest z przerwą energetyczną, to jej brak skutkuje zaniżonymi wartościami, a nawet
zerowaniem się tej wielkości w izolatorach i półprzewodnikach.
Problem przerwy energetycznej będzie dokładniej omawiany w rozdziale~(\ref{sec:insulators}).

Drugim problemem jest występowanie samoodziaływania elektronów, czyli niezerowej energii oddziaływania elektronu 
z jego własnym polem kulombowskim. Jest to efekt niefizyczny, który wynika z niedokładnego kasowania się występujących w hamiltonianie 
przyczynków do oddziaływania elektronu z jego własnym polem elektrycznym.   
W dokładnym opisie ujemny potencjał wymienno-korelacyjny wytwarzany przez każdy elektron powinien dokładnie kasować jego dodatni potencjał Hartree.
W przypadku przybliżenia Hartree-Focka samoodziaływanie nie występuje, ponieważ oddziaływanie wymienne i Hartree wzajemnie kasują się.
W przypadku DFT, potencjał Hartree wyliczany jest dokładnie, natomiast potencjał wymienno-korelacyjny jest przybliżony,
stając się źródłem samooddziaływania. 
Nawet w przypadku układu jednoelektronowego, energia Hartree nie kasuje się z energią wymienną wyznaczoną w przybliżeniu LDA lub GGA.
Na przykładzie atomu wodoru, dyskutowanym w rozdziale \ref{sec:GGA}, dobrze widać efekt jaki powodują różnice między wartościami dokładnymi i przybliżonymi.
Ponieważ wartość bezwględna energii wymiany jest mniejsza niż wartość dokładna, nie kasuje całkowicie dodatniej energii Hartree. 
Wprawdzie częściowo różnica ta redukowana jest przez niezerową i ujemną energię korelacji, ale pozostaje skończona. 
Ta dodatkowa energia oddziaływania elektronu z własnym polem elektrycznym może powodować duże błędy w obliczeniach.
W zależności od rodzaju materiału, samoodziaływanie przyjmuje bardzo różne wartości. Jest ono zaniedbywalnie małe dla rozległych stanów elektronowych,
zdelokalizowanych w dużym obszarze danego materiału. Dlatego wpływ samoodziaływania na stany elektronowe w prostych metalach, 
gdzie dominują stany walencyjne typu $s$ i $p$, może być pominięty. 
W materiałach, w których występują stany elektronowe mające tendencję do lokalizacji (np. stany {\it 3d} i {\it 4f}),
jest ono szczególnie silne i często prowadzi do wyników jakościowo niezgodnych z eksperymentem.
W stanach zlokalizowanych, oddziaływanie elektronu z jego własnym polem kulombowskim, które ma charakter odpychający,
może prowadzić do znacznego wzrostu energii. Zredukowanie tej energie jest możliwe tylko poprzez zwiększenie delokalizacji
ładunku i poszerzenie pasm elektronowych. Prowadzi to efektywnie do zmniejszenia przerwy energetycznej
otrzymywanej w obliczeniach dla półprzewodników i izolatorów. W skrajnym przypadku błędy samoodziaływania powodują delokalizację ładunków w stanach $3d$ 
i zamknięcie przerwy energetycznej, co skutkuje niewłaściwym stanem metalicznym w materiałach, które w rzeczywistości są izolatorami Motta. 

Problemy te mogą być wyeliminowane lub częściowo zredukowane przy zastosowaniu funkcjonałów zależnych od orbitali elektronowych~\cite{kummel}.
W kolejnych rozdziałach omawiane są metody, które uwzlędniają zależność orbitalną w funkcjonale wymienno-korelacyjnym
lub wprowadzają poprawki do energii całkowitej zależne od obsadzenia orbitali.  

\section{Funkcjonały OEP}
\label{sec:mgga}

Podejście nazywane metodą zoptymalizowanego potencjału efektywnego ({\it ang. optimized effective potential} - OEP)  
zostało zaproponowane prez Sharpa i Hortona w roku 1953, jeszcze zanim powstała teoria funkcjonału gęstości~\cite{Sharp}. 
Ta metoda zakłada funkcjonalną zależność energii wymienno-korelacyjnej od orbitali $\phi_{j\sigma}$, przy zachowanym lokalnym charakterze potencjału $V^{OEP}_{\sigma}$. Równanie OEP zostało rozwiązane numerycznie po raz pierwszy przez Talmana i Shadwicka~\cite{Talman}. 
Energię całkowitą Kohna-Shama możemy zapisać jako funkcjonał orbitali
%
\begin{equation}
E[\{\phi_{j\sigma}\}]=T[\{\phi_{j\sigma}\}]+\int d\bm{r} V_{ext}(\bm{r})n(\bm{r})+\frac{1}{2}\int \int d\bm{r} d\bm{r}' \frac{n(\bm{r})n(\bm{r}')}{|\bm{r}-\bm{r}'|}+E_{xc}[\{\phi_{j\sigma}\}].
\end{equation}
%
Metoda OEP ma charakter wariacyjny i opiera się na dwóch równaniach. Pierwsze określa warunek minium (stacjonarności) energii całkowitej
względem efektywnego potencjału
%
\begin{equation}
\frac{\delta E[\{\phi_{j\sigma}\}]}{\delta V^{OEP}_{\sigma}}=0,
\end{equation} 
%
który jest równoważny zasadzie wariacyjne Kohna-Shama $\delta E_\text{tot}/\delta n=0$~\cite{Sahni}.
Drugie równanie ma postać równania Kohna-Shama z efektywnym potencjałem zależnym od orbitali
%
\begin{equation}
\Big[-\frac{\hbar^2}{2m}\nabla^2+V^{OEP}_{\sigma}[\{\phi_{j\sigma}\}](\bm{r})\Big]\phi_{j\sigma}(\bm{r})=\varepsilon_{j\sigma}\phi_{j\sigma}(\bm{r}).
\end{equation}
%
Część potencjału opisującą oddziaływanie wymienno-korelacyjne można zapisać w postaci podwójnej całki wykorzystując własności pochodnych funkcjonalnych
%
\begin{equation}
V^{OEP}_{xc\sigma}(\bm{r})=\frac{\delta E_{xc}[\{\phi_{j\sigma}\}]}{\delta n_{\sigma}(\bm{r})}=\sum_{\nu\mu\i}\int\int d\bm{r}' d\bm{r}''\frac{\delta E_{xc}}{\delta \phi_{i\nu}(\bm{r}')}\frac{\delta \phi_{i\nu}(\bm{r}')}{\delta V^{OEP}_{\mu}(\bm{r}'')}\frac{\delta V^{OEP}_{\mu}(\bm{r}'')}{\delta n_{\sigma}(\bm{r})}.
\label{voep}
\end{equation}
%
Drugą z pochodnych funkcjonalnych pod całką można wyrazić w pierwszym rzędzie rozwinięcia pertubacyjnego 
%
\begin{equation}
\frac{\delta \phi_{i\nu}(\bm{r}')}{\delta V^{OEP}_{\mu}(\bm{r}'')}=\delta_{\nu\mu}\sum_{k\neq i}\frac{\phi^*_{k\mu}(\bm{r}')\phi_{k\mu}(\bm{r}'')}{\varepsilon_{i\mu}-\varepsilon_{k\mu}}\phi_{i\mu}(\bm{r}'')=\delta_{\nu\mu}G^R_{i\mu}(\bm{r}',\bm{r}'')\phi_{i\mu}(\bm{r}''),
\end{equation}
%
gdzie wprowadzono funkcję Greena dla nieodziałujących elektronów
%
\begin{equation}
G^R_{i\mu}(\bm{r}',\bm{r}'')=\sum_{k\neq i}\frac{\phi^*_{k\mu}(\bm{r}')\phi_{k\mu}(\bm{r}'')}{\varepsilon_{i\mu}-\varepsilon_{k\mu}}.
\label{fgreena}
\end{equation} 
%
Trzecia z pochodnych jest odwrotnością liniowej funkcji odpowiedzi, którą również możemy wyrazić przez funkcję Greena
%
\begin{equation}
\chi_{\sigma}(\bm{r},\bm{r}'')=\delta_{\sigma\mu}\frac{\delta n_{\sigma}(\bm{r})}{\delta V^{OEP}_{\mu}(\bm{r}'')}=\delta_{\sigma\mu}\sum_i G^R_{i\sigma}(\bm{r}'',\bm{r})\phi_{i\sigma}(\bm{r}'')\phi^*_{i\sigma}(\bm{r}).
\label{vchi}
\end{equation}
%
Mnożąc równanie (\ref{voep}) przez $\chi_{\sigma}(\bm{r}',\bm{r})$, całkując wzlędem $\bm{r}'$ i wykorzystując (\ref{vchi}) dostajemy 
równanie całkowe metody OEP
%
\begin{equation}
\sum_i\int d\bm{r}' \phi^*_{i\sigma}(\bm{r})[V^{OEP}_{xc\sigma}(\bm{r})-u^{OEP}_{xci\sigma}(\bm{r})]G^R_{i\sigma}(\bm{r}',\bm{r})\phi_{i\sigma}(\bm{r}) + c.c.=0,
\label{oepint}
\end{equation}
%
gdzie
%
\begin{equation}
u^{OEP}_{xci\sigma}(\bm{r})=\frac{1}{\phi^*_{i\sigma}(\bm{r})}\frac{\delta E_{xc}[\{\phi_{j\sigma}\}]}{\delta \phi_{i\sigma}(\bm{r})}.
\end{equation}
%
Jeżeli funkcjonał wymienno-korelacyjny zależy tylko od gęstości, wtedy $u^{OEP}_{xci\sigma}(\bm{r})=V^{OEP}_{xc\sigma}(\bm{r})$ i równanie (\ref{oepint})
jest automatycznie spełnione.
W ramach metody OEP, część wymienną można traktować dokładnie, tak jak w przybliżeniu Hartree-Focka, i wtedy potencjał zależny od orbitali
ma postać
%
\begin{equation}
u^{OEP}_{xi\sigma}=-\frac{1}{\phi^*_{i\sigma}(\bm{r})}\sum_j \phi^*_{j\sigma}\int d\bm{r}'\frac{\phi^*_{i\sigma}(\bm{r}')\phi_{j\sigma}(\bm{r}')}{|\bm{r}-\bm{r}'|}.
\end{equation} 
%
Ze względu na całkowy charakter równań OEP, wyznaczenie samozgodnych rozwiązań, nawet przy zastosowaniu dodatkowych przybliżeniach, wymaga bardzo czasochłonnych obliczeń~\cite{kummel}. Z tego powodu potencjały OEP rzadko stosowane są w praktyce.
Znacznie efektywniejsze są potencjały zależne od orbitali omówione w kolejnych rozdziałach. 

\section{Korekcja samoodziaływania (SIC)}
\label{sec:sic}

W roku 1981, Pardew i Zunger zaproponowali metodę korekcji samoodziaływania ({\it ang. self-interaction correction} - SIC) \cite{PZ}.
Metoda ta dotyczy stanów zlokalizowanych, więc rozpatrujemy zbiór orbitali $\phi_{\alpha\sigma}(\bm{r})$, gdzie $\alpha$ oznacza numer orbitalu,
a $\sigma$ kierunek spinu. Oznaczamy gęstość elektronową pojedynczego orbitalu przez 
%
\begin{equation}
n_{\alpha\sigma}(\bm{r})=f_{\alpha\sigma}|\phi_{\alpha\sigma}(\bm{r})|^2,
\end{equation}
%
gdzie liczba $f_{\alpha\sigma}$ określa obsadzenie danego orbitalu. 
Warunkiem braku samoodziaływania w każdym orbitalu jest kompletne wzajemne kasowanie się energii Hartree danej wzorem (\ref{Hartree}) 
i energii wymienno-korelacyjnej
%
\begin{equation}
E_H[n_{\alpha\sigma}]+E_{xc}[n_{\alpha\sigma}]=0,
\label{nosic}
\end{equation}
%
gdzie obydwie energie są funkcjonałami gęstości pojedynczego orbitalu.
Dla pojedynczego elektronu spełniony jest również warunek kasowania
się energii Hartree i dokładnej energii wymiany wylicznej w podejściu Hartree-Focka
%
\begin{equation}
E_H[n_{\alpha\sigma}]+E_{x}[n_{\alpha\sigma}]=0.
\end{equation}
%
Ponieważ energia korelacji $E_c=E_{xc}-E_{x}$, z tych dwóch warunków wynika, że energia korelacji dla pojedynczego orbitalu musi się zerować
%
\begin{equation}
E_c[n_{\alpha\sigma}]=0.
\end{equation}
%
Warunki te nie są spełnione dla energii wymienno-korelacyjnej danej wzorami w przybliżeniu LDA (\ref{xclda})
lub GGA (\ref{xcgga}), co prowadzi do samoodziaływania elektronów. Oznaczmy ogólnie przez $E_{xc}^{DFT}[n_{\uparrow},n_{\downarrow}]$ funkcjonał wymienno-korelacyjny wyznaczonony w jednym z tych przybliżeń. Wtedy wzór na poprawioną energię wymienno-korelacyjną można zapisać w formie
%
\begin{equation}
E_{xc}^{SIC}[n_{\alpha\sigma}]=E_{xc}^{DFT}[n_{\uparrow},n_{\downarrow}]-\sum_{\alpha\sigma}\delta_{\alpha\sigma},
\label{xcsic}
\end{equation}
%
gdzie
%
\begin{equation}
\delta_{\alpha\sigma}=E_H[n_{\alpha\sigma}]+E_{xc}^{DFT}[n_{\alpha\sigma}],
\end{equation}
%
a sumowanie jest tylko po zajętych orbitalach $\alpha$.
Dla dokładnej wartości energii wymienno-korelacyjnej mielibyśmy $\delta_{\alpha\sigma}=0$,
czyli spełniony byłby warunek (\ref{nosic}). 
Energia wymienno-korelacyjna zdefiniowana wzorem (\ref{xcsic}) nie jest funkcjonałem gęstości elektronowej,
tylko funkcjonałem orbitali atomowych. Powoduje to, że również funkcjonał energii całkowitej (\ref{EKS}) staje się zależny od orbitali
% 
\begin{equation}
E^{SIC}[n_{\alpha\sigma}]=\sum_{\alpha\sigma}\langle\phi_{\alpha\sigma}|-\frac{\hbar^2}{2m}\nabla^2|\phi_{\alpha\sigma}\rangle+E_{ext}[n]+E_H[n]+E_{xc}^{SIC}[n_{\alpha\sigma}].
\label{esic}
\end{equation}
% 
Podobnie, jak w przypadku funkcjonału Kohna-Shama, możemy zastosować metodę wariacyjną do zmninimalizowania tej energię względem
zbioru orbitali $\phi_{\alpha\sigma}$, przy założonym warunku ortonormalności 
%
\begin{equation}
\langle\phi_{\alpha\sigma}|\phi_{\alpha'\sigma'}\rangle=\delta_{\alpha\alpha'}\delta_{\sigma\sigma'}.
\end{equation}
%
Dostajemy wtedy równania jednocząstkowe analogiczne do równań Kohna-Shama
%
\begin{equation}
[-\frac{\hbar^2}{2m}\nabla^2+V_{eff}(n_{\alpha\sigma},\bm{r})]\phi_{\alpha\sigma}(\bm{r})=\varepsilon_{\alpha\sigma}^{SIC}\phi_{\alpha\sigma}(\bm{r}),
\label{eqsic}
\end{equation} 
%
z efektywnym potencjałem danym wzorem
%
\begin{equation}
V_{eff}(n_{\alpha\sigma},\bm{r})=V^{DFT}(\bm{r})-V_H(n_{\alpha\sigma},\bm{r})-V_{xc}(n_{\alpha\sigma},\bm{r}),
\end{equation}
%
gdzie ostatnie dwa wyrazy to korekcja samoodziaływania do efektywnego potencjału elektronowego.
Podobnie jak w przypadku stanów Kohna-Shama spełniona jest zależność
%
\begin{equation}
\varepsilon_{\alpha\sigma}^{SIC}=\frac{\partial E^{SIC}}{\partial f_{\alpha\sigma}}.
\end{equation}

Równanie (\ref{eqsic}) pozwala wyznaczyć orbitale i energie własne zlokalizowanych stanów elektronowych. 
Jednak w rzeczywistych materiałach, mamy często doczynienia z współistnieniem stanów zlokalizowanych i zdelokalizowanych, co powoduje, że rozwiązania tego równania nie muszą odpowiadać globalnemu minimum. Pełna optymalizacja funkcji falowych powinna 
uwzlędniać zarówno stany zlokalizowane (typu Wanniera) i zdelokalizowane (typu Blocha).
W praktyce polega to na wyznaczeniu różnych możliwych konfiguracji stanów zlokalizowanych i zdelokalizowanych,
a następnie wybraniu tej o najniższej energii.    

\begin{figure}[h]
\centering
\includegraphics[scale=1.2]{rare-earths.pdf}
\caption{Różnice energii całkowitej dla walencyjności II i III wyliczone dla ziemiach rzadkich (o), porównane z eksperymentem (linia przerywana)
i wartościami dla związków ziem rzadkich z siarką (+). Rysunek z pracy \cite{RE}.}
\label{fig:rare}
\end{figure}


Dobrym przykładem są metale ziem rzadkich, w których występują zarówno stany zlokalizowane ($4f$),
jak i zdelokalizowane ($5d$ i $6s$). Dodatkowo występują dwa rodzaje elektronów $4f$. Cześć z nich jest zlokalizowana,
tak jak elektrony rdzenia, mając decydujący wpływ na moment magnetyczny i walencyjność danego atomu. 
Pozostałe, w wyniku hybrydyzacji ze stanami $5d$ i $6s$, tworzą pasmo elektronowe i biorą udział w wiązaniu metalicznym.
W ramach tradycyjnych przybliżeń energii wymienno-korelacyjnej nie można opisać prawidłowo tak złożonej
struktury elektronowej ziem rzadkich. W pracy \cite{RE} zastosowano metodę SIC do wyliczenia obsadzeń
stanów $4f$ i walencyjności we wszystkich atomach ziem rzadkich (rysunek \ref{fig:rare}).
Walencyjność zdefiniowana jest jako ilość stanów zdelokalizowanych przypadająca na jeden atom i wyliczana jest ze wzoru
%
\begin{equation}
N_{val}=Z-N_{core}-N_{SIC},
\end{equation}
%
gdzie $Z$ jest liczbą atomową, $N_{core}$ ilością elektronów w rdzeniu i $N_{SIC}$ liczbą stanów zlokalizowanych $4f$,
dla których odjęto energię samoodziaływania. Dla ziem rzadkich walencyjność przyjmuje dwie możliwe wartości:
$N_{val}=2$ (atomy dwuwartościowe, II) lub $N_{val}=3$ (atomy trójwartościowe, III).
Dla tych dwóch przypadków wyliczone różnice ich energii całkowitych $E_{II}-E_{III}$ pokazane są na rysunku \ref{fig:rare}.
Otrzymane wartości bardzo dobrze zgadzają się z danymi eksperymentalnymi. 
Jak widać, oprócz Eu i Yb, które są dwuwartościowe, wszystkie pozostałe ziemie rzadkie są trójwartościowe. 
Podobnie dobrą zgodność z eksperymentem otrzymano dla promieni Wignera-Seitza i stałych sieci kryształów ziem rzadkich \cite{RE}.

Ze względu na jawną zależność od orbitali, funkcjonały SIC wykazują nieciągłość pochodnej po gęstości elektronowej.
Dzięki temu metoda ta znacznie poprawia wartości przerwy energetycznej i momentów magnetycznych
w układach silnie skorelowanych~\cite{svane1990,temmerman2001}. 
On the other hand, the SIC does not provide the systamic improvement in description of molecular
systems~\cite{kummel}.


\section{Funkcjonały hybrydowe}
\label{sec:hybrid}

Problem samoodziaływania nie występuje w przybliżeniu Hartree-Focka, gdzie dla każdego elektronu
jego własne oddziaływanie kulombowskie kasowane jest przez oddziaływanie wymienne.
Zatem częściowe uwzlędnienie dokładnej wartości energii wymiany w funkcjonale wymienno-korelacyjnym powinno
zmniejszyć efekt samoodziaływania i w ten sposób poprawić całkowitą energię układu. 
Takie funkcjonały, w których energia wymiany otrzymana w przybliżeniu LDA lub GGA jest częściowo zastąpiona
dokładną wartością otrzymaną z metody Hartree-Focka, nazywane są funkcjonałami hybrydowymi.  
Pierwszy funkcjonał hybrydowych zaproponował Becke w roku 1993 \cite{becke93}, uzasadniając jego postać
na podstawie tzw. formuły adiabatycznego połączenia ({\it ang. adiabatic connection formula}).
Rozważmy ogólny hamiltoniana w postaci
%
\begin{equation}
H_{\lambda}=T+\lambda V,
\end{equation}
%
gdzie $T$ jest operatorem energii kinetycznej, $V$ jest potencjałem oddziaływania między elektronami, a $\lambda$ jest parametrem, który
określa siłę tego ddziaływania ($0\leq\lambda\leq 1$). 
Funkcje falowe $\psi_{\lambda}$ i energie własne $E_{\lambda}$ tego hamiltonianu są zależne od parametru $\lambda$.
Dla nich spełnione jest równanie
%
\begin{equation}
\frac{dE_{\lambda}}{d\lambda}=\langle\psi_{\lambda}|V|\psi_{\lambda}\rangle,
\end{equation} 
%
z którego dostajemy
%
\begin{equation}
E_{\lambda=1}=E_{\lambda=0}+\int_0^1 d\lambda \langle\psi_{\lambda}|V|\psi_{\lambda}\rangle.
\end{equation}
%
Energia dla $\lambda=0$ odpowiada energii kinetycznej, natomiast $\lambda=1$ daje energie z pełnym oddziaływaniem elektronowym $V$.
Ta formuła określa właśnie adiabatyczne połączenie między układem nieoddziałujących i oddziałujących elektronów.
Po zastosowaniu tej formuły do funkcjonału energii (z pominięciem potencjału zewnętrznego $V_{ext}$) dostajemy 
%
\begin{equation}
E[n]=T[n]+\int_0^1 d\lambda \langle\psi_{\lambda}|V|\psi_{\lambda}\rangle.
\end{equation}
%
Biorąc pod uwagę funkcjonał Kohna-Shama (\ref{EKS}), możemy napisać wzór na energię wymienno-korelacyjną w postaci
%
\begin{equation}
E_{xc}[n]=\int_0^1 d\lambda \langle\psi_{\lambda}|V|\psi_{\lambda}\rangle-E_H[n]=\int_0^1d\lambda E_{xc,\lambda}[n].
\label{xcn}
\end{equation}
%
Powyższą całkę można przybliżyć, rozpatrując jedynie wartości końcowe. Dla $\lambda=0$, dostajemy energie wymiany
w przybliżeniu Hartree-Focka, $E_{xc,0}=E_x^{HF}$. Becke zaproponował, aby dla $\lambda=1$ zastosować jedno z przybliżeń stosowanych w DFT (LDA lub GGA),
$E_{xc,1}=E_{xc}^{DFT}$, a dla pośrednich wartości $\lambda$ zastosować interpolację liniową
%  
\begin{equation}
E_{xc,\lambda}[n]=(1-\lambda)E_x^{HF}[n]+\lambda E_{xc}^{DFT}[n].
\end{equation}
%
Po wstawieniu tej zależności do (\ref{xcn}) i wyliczeniu całki dostajemy funkcjonał hybrydowy 
%
\begin{equation}
E_{xc}^{hyb}[n]=\frac{1}{2}E_x^{HF}+\frac{1}{2}E_{xc}^{DFT}[n].
\label{becke}
\end{equation} 
%
Funkcjonał ten można ulepszyć poprawiając przybliżone całkowanie $E_{xc,\lambda}[n]$.
Dla dowolnego układu, można znaleźć parametr $b$, który najlepiej opisuje
energię wymienną
%
\begin{equation}
E_{xc}^{hyb}[n]=bE_x^{HF}+(1-b)E_{xc}^{DFT}[n].
\end{equation}
W praktyce stosuje się funkcjonały, w których liniowo miksowane są tylko dokładna i przybliżona część
energii wymiennej, a część korelacyjna jest przybliżona w ramach LDA lub GGA.
Parametr $b$ można wybrać empirycznie przez dofitowanie różnych wielkości, na przykład energii
atomizacji, do eksperymentalnych wartości dla dużej grupy materiałów.
Optymalne zgodności otrzymuje się zwykle dla wartości z przedziału $b=[0.15,30]$.

Wybór parametru $b$ można dodatkowo uzasadnić stosująć podejście, które zaproponowali Pardew, Ernzerhof i Burke w roku 1996~\cite{PBE0}.  
Zapisując zależność energii wymienno-korelacyjnej od $\lambda$ w formie rozwinięcia perturbacyjnego Mollera-Plesseta, 
dostaje się zależność
%
\begin{equation}
E_{xc,\lambda}[n]=E_{xc}^{DFT}[n]+(E_x^{HF}[n]-E_x^{DFT}[n])(1-\lambda)^{m-1},
\end{equation}
%
która prowadzi do funkcjonału hybrydowego w postaci
%
\begin{equation}
E_{xc}^{hyb}[n]=E_{xc}^{DFT}[n]+\frac{1}{m}(E_x^{HF}[n]-E_x^{DFT}[n]).
\end{equation}
%
Dla $m=1$, energia wymiany jest równa dokładnej wartości $E_x^{HF}$. W tym przypadku dodanie przybliżonej wartość energii korelacji $E_c^{DFT}$ 
poprawia wyniki w porównaniu do przybliżenia Hartree-Focka. Wyniki są jednak gorsze niż w przybliżeniach LDA lub GGA, w których błędy energii wymiany i korelacji mają przeciwne znaki, co powoduję ich częściowe kasowanie się. Przypadek $m=2$ odpowiada funkcjonałowi, który dany jest wzorem~(\ref{becke}). 
Za optymalny uznano wybór $m=4$~\cite{PBE0}, wskazując na bardzo dobrą zgodnością wyników otrzymanych w ramach teorii perturbacyjnej Mollera-Plesseta czwartego rzędu z danymi eksperymentalnymi \cite{pople89}. Co więcej, w tym przypadku zarówno wartości $E_{xc,\lambda}$ i $E_{xc}^{DFT}$, jak również ich pierwsze oraz drugie pochodne wzajemnie pokrywają się dla $\lambda=1$. Parametr $b$ związany jest z wykładnikiem $m$ prostą zależnością $b=\frac{1}{m}$, co daje $b=0.25$, bliskie wartościom empirycznym.

Funkcjonał hybrydowy jest specjalnym przypadkiem funkcjonału zależnego od orbitali, w którym 
potencjał Kohna-Shama składa się z części lokalnej (kwazilokalnej) i części
potencjału Focka opisującego dokładną energię wymiany. 
Oddziaływanie wymienne ma charakter krótkozasięgowy i można uwzględnić wkład od dokładnej wartości jedynie
w pewnym zakresie odległości. Uwzględniając ten efekt Heyd, Scuseria i Enzerhof (HSE) \cite{HSE} zaproponowali nowy funkcjonał
w postaci
%
\begin{equation}
E_{xc}^{hyb}(\mu) = (1-b)E_{x}^{GGA,SR}(\mu)+bE_x^{HF,SR}(\mu) + E_x^{GGA,LR} + E_c^{GGA},
\end{equation}
%
gdzie $E_x^{GGA,SR}$ i $E_x^{GGA,LR}$ odpowiadają krótkozasięgowemu i długozasięgowemu oddziaływaniu wymiennemu w przybliżeniu GGA,
$E_x^{HF,SR}$ jest częścią krótkozasięgowego dokladnego oddziaływania wymiennego, a parametr $\mu$
określa zasięg oddziaływania krótkozasięgowego. .


\section{Funkcjonały i potencjały meta-GGA}
\label{mgga}


Dwa podstawowe przybliżenia stosowane do opisu funkcjonału wymienno-korelacyjnego uwzlędniają tylko gęstość ładunku w danym punkcie
przestrzeni (LDA) lub dodatkowo jego gradient (GGA). 
Kolejnym krokiem pozwalającym poprawić ten funkcjonał jest uwzlędnienie jego zależności od gradientu orbitali, czyli 
części kinetycznej całkowitej energii. 

%
\begin{figure}[h]
\centering
\includegraphics[scale=1]{Jacob.jpg}
\caption{Drabina Jakuba opisująca kolejne poziomy dokładności opisu funkcjonału wymienno-korelacyjnego w DFT.}
\label{fig:jacob}
\end{figure}
%

Takie funkcjonały noszą nazwę meta-GGA i zapisywane są w ogólnej postaci
%
\begin{equation}
E_{xc}[n_{\uparrow},n_{\downarrow}]=\int d\bm{r} n(\bm{r}) \varepsilon_{xc}(n_{\uparrow},n_{\downarrow},\nabla n_{\uparrow},\nabla n_{\downarrow},\tau_{\uparrow},\tau_{\downarrow}),
\label{metaGGA}
\end{equation}
%
gdzie $\tau_{\sigma}$ jest gęstością energii kinetycznej wyznaczoną dla obsadzonych stanów Kohna-Shama 
%
\begin{equation}
\tau_{\sigma}(\bm{r})= \frac{1}{2}\sum_i |\nabla \psi_{i\sigma}(\bm{r})|^2,
\end{equation}
%
w jednostkach $\hbar=m_\text{e}=1$. Uzasadnieniem włączenia $\tau_{\sigma}$ do funkcjonału jest występowanie tej wielkości w rozwinięciu Taylora
uśrednionego sferycznie ładunku wymiennego wokół elektronu~\cite{Becke1998}. Drugą korzystną cecha jest możliwość zredukowania samoodziaływania
w części korelacyjnej funkcjonału.


\subsection{Potencjał Becke-Johnsona}

Becke and Johnson (BJ) zastosowali rozszerzenie potencjału oddziaływania wymiennego Slatera w postaci~\cite{Becke2006}
%
\begin{equation}
V^\text{BJ}_{\text{x}\sigma}(\bm{r})=V_{\text{x}\sigma}^\text{Slater}(\bm{r})+\frac{1}{\pi}\sqrt{\frac{5}{12}}\sqrt{\frac{2\tau_{\sigma}(\bm{r})}{n_{\sigma},(\bm{r})}}.
\label{BJ}
\end{equation} 
%
gdzie omawiany w rozdziale (\ref{sec:HF}) potencjał Slatera dany jest wzorem
%
\begin{equation}
V_{\text{x}\sigma}^\text{Slater}(\bm{r})=-\frac{e}{2}\int d\mb{r}' \frac{n(\mb{r},\mb{r}')}{|\mb{r}-\mb{r}'|}, 
\label{SlatPot}
\end{equation}
% 
w którym występuje ładunek wymienny $n(\mb{r},\mb{r}')$ opisany wzorem (\ref{EC}). Potencjał BJ znacznie poprawia energie odziaływania wymiennego
w porównaniu do przybliżenia LDA i daje wyniki porównywalne z potencjałem OEP~\cite{Becke2006}.  
Tran i Blaha zaproponowali modyfikację potencjału wymiennego BJ w postaci~\cite{Tran2009}
%
\begin{equation}
V^\text{mBJ}_{\text{x}\sigma}(\bm{r})=cV^\text{BR}_{\text{x}\sigma}(\bm{r})+(3c-2)\frac{1}{\pi}\sqrt{\frac{5}{12}}\sqrt{\frac{2\tau_{\sigma}(\bm{r})}{n_{\sigma}(\bm{r})}},
\label{mBJ}
\end{equation}
%
gdzie zamiast potencjału Slatera występuje potencjał Becke-Roussel'a
%
\begin{equation}
V^\text{BR}_{\text{x}\sigma}(\bm{r})=-\frac{1}{b_{\sigma}(\bm{r})}(1-e^{-x_{\sigma}(\bm{r})}-\frac{1}{2}x_{\sigma}(\bm{r})e^{-x_{\sigma}(\bm{r})}).
\end{equation}
%  
Wielkość $x_{\sigma}(\bm{r})$ zależy od gęstości elektronowej i jej gradientu oraz od $\tau_{\sigma}(\bm{r})$, natomiast
%
\begin{equation}
b_{\sigma}(\bm{r})=[\frac{x^3_{\sigma}(\bm{r})e^{-x_{\sigma}(\bm{r})}}{8\pi n_{\sigma}(\bm{r})}]^{\frac{1}{3}}.       
\end{equation}
%
Parametr $c$ wyraża się następującym wzorem
%
\begin{equation}
c=\alpha+\beta[\frac{1}{V_\text{cell}}\int d\bm{r}'\frac{|\nabla n(\bm{r}')|}{n(\bm{r}')}]^{\frac{1}{2}},
\end{equation}
%
gdzie $\alpha$ i $\beta$ wyznaczone są tak, aby zminimalizować odstępstwa wyliczanych przerw energetycznych
od wartości eksperymentalnych dla dużej grupy materiałów.
Potencjał mBJ (\ref{mBJ}) przechodzi w potencjał BJ (\ref{BJ}) dla $c=1$ oraz dobrze
przybliża potencjał wymienny LDA dla stałej gęstości elektronowej dany wzorem (\ref{Vex}),
niezależnie od wartości parametru $c$.
W porównaniu do przybliżeń LDA lub GGA, obliczenia z potencjałem mBJ znacznie poprawiają wartości 
przerwy energetycznej w gazach szlachetnych, półprzewodnikach i izolatorach~\cite{Tran2009}.
Wartości te są porównywalne z wynikami otrzymanymi przy zastosowaniu funkcjonałów hybrydowych lub metody GW,
które wymagają bardzo czasochłonnych obliczeń.
Ponieważ w tym podejściu poprawiany jest potencjał wymienny, a nie funkcjonał wymienny, nie można go zastosować
do optymalizacji położeń atomowych i stałych sieci krystalicznej.  
  

\subsection{Funkcja lokalizacji elektronów (ELF)}

Funkcjonały meta-GGA używają bezwymiarowego parametru 
%
\begin{equation}
\alpha=\frac{\tau-\tau^w}{\tau^{u}},
\end{equation}
%
gdzie $\tau^w$ jest gęstością energii kinetycznej Weizsäckera
%
\begin{equation}
\tau^w=\frac{|\nabla n|^2}{8n},
\end{equation}
%
która określa dokładną wartość $\tau$ dla pojedynczego orbitalu, a $\tau^u$ jest jego wartością dla jednorodnego gazu
%
\begin{equation}
\tau^u=\frac{3}{10}\Big{(}\frac{3}{\pi^2}\Big{)}^{\frac{2}{3}}n^{\frac{5}{3}}.
\end{equation}
%
Parametr $\alpha$ wykorzystywany jest do charakteryzacji rozkładu gęstości elektronowej i pozwala odróżnić
układy z wolnozmienną gestością elektronową, typową dla metali ($\alpha\approx 1$), 
materiały z wiązaniami kowalencyjnymi między pojedynczymi orbitalami ($\alpha=0$) oraz słabymi wiązaniami niekowalencyjnymi między
zamkniętymi powłokami atomowymi ($\alpha\rightarrow\infty$).
Jest on bezpośrednio powiązany z funkcją lokalizacji elektronów ({\it ang. electron localization function} - ELF)~\cite{ELF1}
%
\begin{equation}
\textrm{ELF} = \frac{1}{1+\alpha^2},
\end{equation}     
%
która przyjmuje wartości z przedziału $[0,1]$ i jest używana do opisu wiązań chemicznych~\cite{ELF2,ELF3}.
$\text{ELF}=1$ odpowiada idealnej lokalizacji elektronu, a $\text{ELF}=\frac{1}{2}$ jest wartością dla gazu elektronowego.  


\subsection{Funkcjonał SCAN}

W omówionych potencjałach BJ i mBJ zmiania się tylko część opisująca oddziaływanie wymienne, natomiast
część korelacyjna pozostaje bez zmiany i może być wyznaczona w ramach przybliżeń LDA lub GGA.
Funkcjonały meta-GGA, których ogólna postać wyraża się wzorem (\ref{metaGGA}), uwzględniają poprawki w części
wymiennej i korelacyjnej. Zaproponowano kilka wersji funkcjonałów meta-GGA, których nazwy określane są skrótami
PKZB~\cite{metaGGA}, TPSS~\cite{TPSS}, revTPSS~\cite{revTPSS} lub SCAN~\cite{SCAN}.
Tutaj omówimy krótko funkcjonał SCAN ({\it ang. strongly contrained and appropriately normed}), który 
spełnia 17 znanych warunków i norm dla funkcjonału wymienno-korelacyjnego.
Jednym z nich jest warunek dla współczynnika określającego stosunek energii wymiany do odpowiedniej wartości w LDA, $F_x=E_x/E^{LDA}_x$, który
nie może przekraczać wartości 1.174. Poprawnie opisuje również układy, dla których dokładne wartości są znane, np. jednorodny gaz elektronowy.  

Część wymienną funkcjonału można zapisać w postaci
%
\begin{equation}
E_x[n]=\int d\bm{r} n(\bm{r}) \varepsilon_x(n)F_x(s,\alpha),
\end{equation}
%
gdzie $\varepsilon_x(n)$ jest energią wymiany dla jednorodnego gazu przypadającą na pojedynczy elektron,
$\alpha$ jest parametrem zdefiniowanym w poprzednim rozdziale, a $s$ jest bezwymiarowym gradientem gęstości
%
\begin{equation}
s=\frac{|\nabla n|}{2(3\pi^2)^{1/3}n^{4/3}}.
\end{equation}
%
Rysunek~\ref{fig:Fxs} pokazuje zależność współczynnika $F_x$ od $s$ dla trzech charakterystycznych wartości $\alpha$.
$F_x$ spełnia odpowiednie warunki zarówno dla małych $s\rightarrow0$, jak i dużych wartości $s\rightarrow\infty$.  
Dla $\alpha=1$ i małych $s$, $F_x$ pokrywa się z wartościami dla funkcjonału PBE.
Wartości $F_x$ dla innych wartości $\alpha$ uzyskuje się przez odpowiednią interpolację między $\alpha=0$ i $\alpha=1$ oraz ekstrapolację
do $\alpha\rightarrow\infty$.
%
\begin{figure}[h]
\centering
\includegraphics[scale=0.5]{Fxs.png}
\caption{Współczynnik $F_x$ w funkcji $s$ dla różnych wartości $\alpha$. Rysunek z pracy \cite{SCAN}.}
\label{fig:Fxs}
\end{figure}
%

Podobnie skonstruowany jest całkowity fukcjonał wymienno-korelacyjny $F_{xc}=F_{x}+F_{c}$, który w granicy
dużych gęstości niespolaryzowanego gazu przyjmuje wartość energii wymiennej $F_x$.
Dla małych gęstości funkcjonał spełnia warunek ograniczenia Lieba-Oxforda, $F_{xc}\leq 2.215$.
 

\section{Metoda LDA+U}
\label{sec:ldau}

Metoda LDA+U zaproponowana została przez Anisimova, Zaanena i Andersena w roku 1991~\cite{anisimov}.
Jednym z pierwszych jej zastosowań było wyliczenie struktury elektronowej dla nadprzewodnika wysokotemperaturowego La$_2$CuO$_4$~\cite{czyzyk}.
Metoda ta polega na połączeniu podejścia DFT z modelem Hubbarda, który stosowany jest od lat sześćdziesiątych ubiegłego wieku do opisu układów silnie skorelowanych elektronów~\cite{hubbard}. Zacznę od omówienia tego modelu. 

W najbardziej tradycyjnym podejściu, stany elektronowe w krysztale dzielimy na dwie grupy: stany zlokalizowane, które znajdują się w obrębie rdzenia atomowego i posiadają cechy orbitali atomowych oraz stany zdelokalizowane, które rozciągają się w całej przestrzeni kryształu.
O własnościach transportowych danego materiału decydują głównie elektrony należące do drugiej grupy.
Natomiast jeżeli chodzi o własności magnetyczne, to zarówno elektrony zlokalizowane, jak i wędrowne mogą oddziaływać wymiennie
i decydować o rodzaju uporządkowania magnetycznego. W pierwszym przypadku mówimy o zlokalizowanych momentach magnetycznych, 
których oddziaływanie można opisać w ramach modelu Heisenberga. W drugim mamy doczynienia z magnetyzmem pasmowym, który wynika
z oddziaływania między mobilnymi elektronami walencyjnymi, które można opisać przy pomocy funkcji Blocha.

Występują również stany elektronowe, które posiadają cechy obydwu tych grup. Stany te tworzą pasma, zachowując jednocześnie
cechy orbitali atomowych. Ruch elektronów w takiej sytuacji polega na przeskokach między lokalnymi orbitali i odpowiadajace im pasma są znacznie węższe od
typowych pasm metalicznych. Żródłem takiego zachowania są efekty korelacyjne, które uniemożliwiają swobodny przepływ elektronów w krysztale.
Najlepszym przykładem są stany należące do niezapełnionej powłoki $3d$ w metalach przejściowych.
Zgodnie z zakazem Pauliego, w każdym orbitalu mogą znajdować się tylko dwa elektrony z przeciwnymi kierunkami spinu. 
Jeżeli występuje silne, lokalne oddziaływanie kulombowskie między ładunkami, to stany kwantowe, w których dwa elektrony   
znajdują się w tym samym orbitalu są niekorzystne energetycznie. Zatem ruch elektronu jest bardzo utrudniony i dla odpowiednio dużego oddziaływania kulombowskiego
elektrony zostają zlokalizowane. Materiały, w których występuje taki mechanizm lokalizacji to izolatory Motta.

Hubbard zaproponował model do opisu stanów elektronowych, które posiadają cechy zlokalizowanych orbitali atomowych,
wynikające z korelacji elektronowych~\cite{hubbard}.
Jest to przykład  modelu ciasnego wiązania z orbitalami Wanniera, który opisany jest w rozdziale~\ref{sec:wannier}. 
W najprostszym przypadku rozważamy pojedyncze orbitale typu $s$, zlokalizowane w węzłach atomowych. Silne korelacje elektronów znajdujących
się w tym samym orbitalu można uwzględnić dodają do hamiltonianu lokalne oddziaływanie kulombowskie o ustalonej energii $U$. 
Hamiltonian modelu Hubbarda zapisujemy stosując formalizm drugiego kwantowania
%
\begin{equation}
H=\sum_{i,j,\sigma}t_{ij}c^{\dagger}_{i\sigma}c_{j\sigma}+U\sum_i n_{i\uparrow}n_{i\downarrow},
\label{hubbard}
\end{equation}
%
gdzie $c^{\dagger}_{i\sigma}$ i $c_{i\sigma}$ to operatory kreacji i anihilacji elektronu ze spinem $\sigma$ w zlokalizowanym orbitalu Wanniera $w(\bm{r}-\bm{R}_i)$,
a całki przeskoku dane są wzorem
%
\begin{equation}
t_{ij}=\int d\bm{r} w^*(\bm{r}-\bm{R}_i)H_0w(\bm{r}-\bm{R}_j).
\end{equation}
%
$H_0$ jest jednocząstkowym hamiltonianem złożonym z energii kinetycznej i efektywnego potencjału atomowego. 
Najczęście uwzlędnia się tylko całki przeskoku między najbliższymi sąsiadami. 
Operator liczby cząstek ze spinem $\sigma$ na węźle $i$ dany jest wyrażeniem $n_{i\sigma}=c^{\dagger}_{i\sigma}c_{i\sigma}$.
Energię oddziaływania kulombowskiego dwóch elektronów w tym samym orbitalu można wyliczyć ze wzoru
%
\begin{equation}
U=\int d\bm{r}d\bm{r}' |w(\bm{r}-\bm{R}_i)|^2\frac{e^2}{|\bm{r}-\bm{r}'|}|w(\bm{r}-\bm{R}_i)|^2.
\end{equation}
%
Jeżeli wykonamy transformatę Fouriera operatorów kreacji i anihilacji,
pierwszy wyraz hamiltonianu (\ref{hubbard}) przyjmie postać $\sum_{\bm{k}\sigma}\varepsilon_{\bm{k}}n_{\bm{k}\sigma}$,
gdzie $\varepsilon_{\bm{k}}$ opisuje strukturę pasmową w modelu ciasnego wiązania,
a $n_{\bm{k}\sigma}=c^{\dagger}_{\bm{k}\sigma}c_{\bm{k}\sigma}$ jest operatorem liczby obsadzeń w stanie $\bm{k}$ i spinie $\sigma$.

Zaproponowano kilka sformułowań metody LDA+U \cite{anisimov,orbital1,dudarev}, które różnią się opisem lokalnych 
oddziaływań eletronowych. Ogólny schemat używany do wyliczenia całkowitej energii ma postać
%
\begin{equation}
E_\text{tot}=E_\text{DFT}+E_U-E_\text{dc},
\label{eldau}
\end{equation}
%
gdzie $E_\text{DFT}$ jest energią układu otrzymaną w ramach metody DFT (w przybliżeniu LDA lub GGA), $E_U$ jest energią lokalnych oddziaływań elektronowych,
a $E_\text{dc}$ odpowiada energii oddziaływań kulombowskich w przybliżeniu średniego pola.
Ten ostatni wyraz jest konieczny, aby odjąć przybliżoną energię oddziaływań elektronowych, która uwzlędniona jest w $E_\text{DFT}$.
W pracy \cite{orbital1} zaproponowano uogólnioną postać energii $E_U$, która uwzglądnia orbitalną zależność oddziaływań elektronowych
%
\begin{multline}
E_U=\frac{1}{2}\sum_{i,\{m\},\sigma}[\langle m,m''|V_{ee}|m',m'''\rangle n_{i\sigma}^{mm'}n_{i-\sigma}^{m''m'''}
\\+(\langle m,m''|V_{ee}|m',m'''\rangle - \langle m,m''|V_{ee}|m''',m'\rangle  )n_{i\sigma}^{mm'}n_{i\sigma}^{m''m'''}],
\end{multline}  
%
gdzie $i$ numeruje węzły sieci,  $\{m\}=(m, m',m'', m''')$ są magnetycznymi liczbami kwantowymi,
a elementy macierzowe oddziaływania kulombowskiego dane są wzorem
%
\begin{equation}
\langle m,m''|V_{ee}|m',m'''\rangle=\int d\bm{r} \int \bm{r}' \phi_{lm}^*(\bm{r})\phi_{lm'}(\bm{r})\frac{e^2}{|\bm{r}-\bm{r}'|}\phi_{lm''}^*(\bm{r}')\phi_{lm'''}(\bm{r}').
\label{integral}
\end{equation}
%
Występujące pod całką atomowe funkcje falowe określone są dla orbitalnej liczby kwantowej $l$, która definiują zakres magnetycznych liczb kwantowych $-l\le \{m\} \le l$. Liczby obsadzeń orbitali atomowych tworzą tensor drugiego rzędu, którego elementy wyznaczane są przez rzutowanie funkcji falowych Kohna-Shama $\psi_{\bm{k}j}^{\sigma}$ na orbitale atomowe
%
\begin{equation}
n_{i\sigma}^{mm'}=\sum_{\bm{k},j} f_{\bm{k}j}^{\sigma}\langle\psi_{\bm{k}j}^{\sigma}|\phi_{lm'}\rangle\langle\phi_{lm}|\psi_{\bm{k}j}\rangle, 
\end{equation}
%
gdzie współczynniki $f_{\bm{k}j}^{\sigma}$ okreśkaja obsadzenia stanów $j$ z wektorem falowym $\bm{k}$ i spinie $\sigma$.
Całkę (\ref{integral}) można przedstawić w postaci sumy iloczynów części kątowej i radialnej 
%
\begin{equation}
\langle m,m''|V_{ee}|m',m'''\rangle=\sum_p a_p(m,m',m'',m''')F^p,
\label{vee}
\end{equation}
%
gdzie $p$ jest liczbą parzystą z przedziału $0\le p \le 2l$. Część kątowa $a_p$ wyliczana jest przy pomocy iloczynów współczynników Clebsha-Gordana 
%
\begin{equation}
a_p(m,m',m'',m''')=\frac{4\pi}{2p+1}\sum_{q=-p}^p \langle lm|Y_{pq}|lm'\rangle\langle lm''|Y_{pq}^*|lm'''\rangle.
\end{equation}
%
$F^p$ nazywane są całkami Slatera i wyznaczane są ze wzoru
%
\begin{equation}
F^p=e^2\int d\bm{r}\int d\bm{r}' r^2 r'^2 R_{nl}^2(\bm{r})\frac{r_1^p}{r_2^{p+1}} R_{nl}^2(\bm{r}'),
\end{equation}
%
gdzie $R_{nl}(\bm{r})$ są częścią radialną atomu funkcji falowej. 
Dla stanów $d$ potrzebne są trzy całki Slatera $F^0$, $F^2$ i $F^4$, a dla stanów $f$ dodatkowo $F^6$, aby wyznaczyć elementy macierzowe 
potencjału kulombowskiego (\ref{vee}).
Efektywne parametry, które określają lokalne oddziaływanie kulombowskie ($U$) i wymienne ($J$), można wyrazić przy pomocy całek Slatera
%
\begin{eqnarray}
U&=&\frac{1}{(2l+1)^2}\sum_{m.m'} \langle m,m'|V_{ee}|m,m'\rangle=F^0,\\
J&=&\frac{1}{2l(2l+1)}\sum_{m\ne m'} \langle m,m'|V_{ee}|m',m\rangle=\frac{F^2+F^4}{14}.
\label{HundJ}
\end{eqnarray} 
%
Wykorzystując te parametry, możemy zapisać ostatni wyraz we wzorze (\ref{eldau}) w następującej formie
%
\begin{equation}
E_{dc}=\frac{1}{2}\Big[\sum_i Un_i(n_i-1)-J[n_{i\uparrow}(n_{i\uparrow}-1)+n_{i\downarrow}(n_{i\downarrow}-1)]\Big],
\end{equation}
%
gdzie $n_{i\sigma}=\Tr(n_{i\sigma}^{mm'})$ i $n_i=n_{i\uparrow}+n_{i\downarrow}$ jest całkowitym obsadzeniem orbitali w węźle $i$.
Tak wyznaczone parametry $U$ i $J$ odnoszą sie do izolowanych atomów. Efektywne wartości parametru kulombowskiego $U$,
stosowane dla danego rodzaju atomu, uwzględniaja efekty ekranowania elektronowego i zależą od rodzaju materiału.
Można go wyznaczyć w ramach metody odpowiedzi liniowej poprzez wyliczenie zmiany obsadzenia stanów elektronowych na wybranym atomie
pod wpływem przyłożonego lokalnego potencjału \cite{coco1,coco2}.
Drugi z parametrów $J$, nazywany energią wymiany Hunda, znacznie słabiej zależy od rodzaju materiału i często stosowana jest jego wartość atomowa (\ref{HundJ}). 


\chapter{Izolatory i półprzewodniki}
\label{sec:insulators}

\section{Przerwa energetyczna}
\label{sec:bandgap}

Fundamentalna przerwa energetyczna zdefiniowana jest jako różnica między energią jonizacji ($I$), czyli procesu usunięcia elektronu z danego materiału, 
a powinowactwem elektronowym ($A$), czyli energią uzyskaną z dodania elektronu
%
\begin{eqnarray}
E_g=I-A&=&[E(N-1)-E(N)]-[E(N)-E(N+1)] \nonumber \\ 
       &=&E(N+1)+E(N-1)-2E(N),
\label{energygap}       
\end{eqnarray}
%
gdzie $E(N)$, $E(N-1)$ i $E(N+1)$ to energie całkowite wyznaczone kolejno dla układu neutralnego składającego się z $N$ elektronów oraz
układów z odjętym i dodanym elektronem. Każda z tych energii może być wyznaczona jako energia stanu podstawowego układu z odpowiednią ilością
elektronów. Wartość przerwy możemy również zapisać używając energii stanów Kohna-Shama 
%
\begin{equation}
E_g=E(N+1)-E(N)-[E(N)-E(N-1)]=\varepsilon_{N+1}(N+1)-\varepsilon_N(N),
\label{truegap}
\end{equation}
%
gdzie $\varepsilon_N(N)$ jest energią najwyższego poziomu pasma walencyjnego w układzie z $N$ elektronami,
a $\varepsilon_{N+1}(N+1)$ energią najniższego stanu pasma przewodnictwa w układzie z $N+1$ elektronami.
Czyli wartość przerwy energetycznej jest teoretycznie osiągalna w ramach DFT, 
pod warunkiem, że znana jest dokladna zależność $E(N)$. Stosując funkcjonały LDA i GGA możemy otrzymać ze wzoru (\ref{truegap}) jedynie przybliżone wartości przerwy fundamentalnej. 
   
Zwykle wykonujemy obliczenia dla układu z ustaloną liczbą elektronów $N$ i otrzymujemy przerwę w spektrum energetycznym Kohna-Shama, 
która wynosi
%
\begin{equation}
\Delta_{KS}=\varepsilon_{N+1}(N)-\varepsilon_N(N).
\label{ksgap}
\end{equation}
%
Porównując wzory (\ref{truegap}) i (\ref{ksgap}) dostajemy związek między tymi dwiema przerwami
%
\begin{equation}
E_g=\Delta_{KS}+\varepsilon_{N+1}(N+1)-\varepsilon_{N+1}(N)=\Delta_{KS}+\Delta_{xc},
\end{equation}
%
gdzie różnica między nimi $\Delta_{xc}$ wynika ze zmiany nachylenia liniowej funkcji $E(N)$ przy całkowitej liczbie elektronów \cite{pardew82}.
Ta nieciągłość ma swoje źródło w potencjale wymienno-korelacyjnym, którego pochodna zmienia się skokowo przy zmianie ilości elektronów
%
\begin{equation}
\Delta_{xc}=\frac{\delta E_{xc}[n]}{\delta n(\bm{r})}|_{N+\delta}-\frac{\delta E_{xc}[n]}{\delta n(\bm{r})}|_{N-\delta},
\label{deltaxc}
\end{equation}
%
gdzie pochodne funkcjonalne wyznaczone są dla gęstości elektronów $n(\bm{r})$, których całki po całej przestrzeni równe są $N+\delta$ i $N-\delta$, w granicy $\delta\rightarrow 0$. 

Występowanie tej nieciągłości jest podstawową cechą DFT, ale w stosowanych przybliżeniach LDA i GGA, które charakteryzują się analityczną zależnością
energii od gęstości elektronowej, te nieciągłości nie występują. Przybliżony charakter energii stanów jednocząstkowych,
z których wyznaczana jest wartość $\Delta_{KS}$, w połączeniu z brakiem nieciągłości powoduje, że przerwy energetyczne wyznaczone w przybliżeniu LDA są średnio zaniżone o $40\%$. W GGA te wartości są zwykle poprawione, ale również znacznie odbiegają od wartości eksperymentalnych. 

\section{Izolatory pasmowe i półprzewodniki}

W większości półprzewodników i izolatorów przerwa energetyczna wynika z istnienia całkowicie zapełnionych pasm walencyjnych
i pustych pasm przewodzących. Rozróżnienie między półprzewodnikami i izolatorami nie jest jednoznacznie określone.
Zwykle półprzewodniki mają mniejszą przerwę energetyczną i ich przewodnictwo elektryczne wykazuje znaczną zależność
od temperatury w wyniku występowania indukowanych termicznie nośników prądu w paśmie przewodnictwa.
W tych materiałach korelacje elektronowe zwykle nie są silne, ale mimo tego wyznaczana w ramach standardowych przybliżeń DFT 
przerwa energetyczna jest znacznie zaniżona. Wynika to z braku uwzglednienia efektów wielociałowych, które występują
przy wzbudzaniu elektronu z pasma walencyjnego do pasma przewodnictwa. Przykładem jest oddziaływanie elektron-dziura
wystepujace dla stanów wzbudzonych.
W przypadku tych materiałów, metoda LDA+U nie jest skuteczna i lepiej nadają się metody SIC i funkcjonały hybrydowe,
jak również metody wielociałowe typu GW. 
W metodzie SIC, energia całkowita zależy od obsadzenia orbitali atomowych, co daje możliwość odtworzenie nieciągłości
pochodnych funkcjonału wymienno-korelacyjnego (\ref{deltaxc}) i poprawienia wartości przerwy energetycznej.
SIC daje dobre wyniki dla izolatorów z dużą przerwa takich, jak kryształy gazów szlachetnych \cite{PZ}, czy kryształy jonowe. Przykładowo dla NaCl przerwa energetyczna wynosi $E_g^{SIC}=9.2$~eV \cite{norman83}, co dobrze zgadza się z wartością eksperymentalną $E_g^{exp}=8.97$~eV.  

Funkcjonały hybrydowe, które częściowo uwzlędniają dokladną wartość oddziaływania wymiennego i przez to redukują błąd samooddziaływania elektronów, 
również poprawiają wartość przerwy energetycznej. Na rysunku~\ref{fig:gaphyb} porównano eksperymentalne wartości przerwy energetycznej
z wyliczonymi w ramach czterech funkcjonałów hybrydowych dla wybranej grupy półprzewodników i tlenków metali przejściowych (FeO, CoO, NiO, MnO, and VO$_2$)~\cite{Gap-hyb}. Dwa z tych funkcjonałów zawierają 20 \% dokładnej wartości energii wymiany (B3PW91 i B3LYP), a dwa pozostałe 25 \% (PBE0 i HSE). Dla wiekszości materiałów 
z przerwą mniejszą niź 5 eV, zgodność z eksperymentem jest bardzo dobra dla wszystkich funkcjonałów. Jednocześnie widzimy, że wybrane funkcjonały znacznie zaniżają wartości przerwy dla trzech półprzewodników o dużej przerwie (NaCl, LiCl i LiF).
Dokładna analiza błędów pokazała, że najlepszą zgodność uzyskano dla funkcjonału HSE, który ma tendencję do niewielkiego zaniżania wartości przerwy (średnio o -0.24 eV). Dwa funkcjonały B3PW91 i B3LYP lekko zawyżają przerwę (średnie błedy wynoszą 0.14 eV i 0.13 eV). Największe odstępstwo obserwujemy dla funkcjonału PBE0, który zawyża wartość przerwy średnio o 0.43 eV.

\begin{figure}[t!]
\centering
\includegraphics[scale=0.5]{gap-hyb.pdf}
\caption{Porównanie eksperymentalnych wartości przerwy energetycznej z obliczonymi dla różnych funkcjonałów hybrydowych. Rysunek z pracy \cite{Gap-hyb}.}
\label{fig:gaphyb}
\end{figure}

W tabeli~\ref{gap} zebrane są wartości przerwy energetycznej dla wybranych materiałów, wyliczone różnymi metodami
i porównane z danymi eksperymentalnymi. LDA i GGA we wszystkich przypadkach znacznie zaniżają wartość przerwy.
Dotyczy to zarówno półprzewodników, szczególnie germanu, gdzie przerwa jest równa zeru, jak i izolatorów.
Duże rozbieżności w porównaniu do eksperymentu widzimy dla trzech ostatnich materiałów, zaliczanych do izolatorów Motta.
Zastosowanie funkcjonału hybrydowego HSE poprawia wartości przerwy i dla większości półprzewodników zgodność z eksperymentem jest bardzo dobra.
Dla izolatorów, ta zgodność jest lepsza lub gorsza w zależności od materiału.
W tabeli~\ref{gap} pokazano również wyniki otrzymane w ramach metody GW i jej uproszczonej wersji G$_0$W$_0$, które również dają
lepsze wartości przerwy. Metoda GW, która uwzlędnia oddziaływania wielociałowe w ramach rozwinięcia perturbacyjnego, będzie tematem jednego z
kolejnych rozdziałów. 

\begin{table}[h!]
\caption{Wartości przerwy energetycznej wyliczone różnymi metodami i porównane z danymi eksperymentalnymi.}
\label{gap}
\begin{center}
\begin{tabular}{|c|c|c|c|c|c|c|}
\hline
 Materiał & LDA & PBE & HSE & G$_0$W$_0$ & GW & Eksperyment \\ \hline
  C & 4.14 & 4.17  & 4.94 & 5.50 & 6.18 & 5.50 \\
  Si & 0.60 & 0.71  & 1.11 & 1.12 & 1.41 & 1.17 \\
  Ge & 0.00 &  & 0.83 & 0.66 & 0.95 & 0.74 \\
  MgO & 4.70 & 4.74  & 6.46 & 7.25 & 9.16 & 7.90 \\
  GaAs & 0.30 & 0.53  & 1.41 & 1.30 & 1.85 & 1.52 \\
  SiC & 1.35 &    & 2.40 & 2.27 & 2.88 & 2.40 \\ 
  GaP & 1.53 & 1.69 & 2.09 &   &   & 2.35  \\
  CdS & 0.96 & 1.23 & 2.27 &   &   & 2.48 \\
  InP & 0.50 & 0.72 & 1.52 &   &   &  1.42 \\
  BN  & 4.42 & 4.53 & 5.39 &   &   &  6.20 \\
  NaCl & 4.70 & 5.08 & 6.42 &  &   & 8.97  \\
  LiF  & 8.84 & 9.04 & 11.4 &  &   & 13.6  \\  
  MnO & 0.76 &    & 2.80 &  & 3.50 & 3.90 \\ 
  FeO & 0 &    & 2.2 &  &  & 2.4 \\ 
  NiO & 0.42 &    & 4.2 & 1.10 & 4.80 & 4.30 \\ \hline 
\end{tabular}
\end{center}
\end{table}

\section{Izolatory Motta}

Szczególną klasę materiałów stanowią izolatory Motta, których własności elektronowe wynikają z silnych oddziaływaniań kulombowskich miedzy elektronami.
W tym przypadku wielkość charakteryzująca nieciągłość wyliczana w ramach standardowych przybliżeń $\Delta_{KS}$ jest równe zeru 
lub jest bardzo małą wielkością. Przerwa energetyczna wynika zasadniczo z nieciągłości funkcjonału energii ($\Delta_{xc}>0$), powstającej przy zmianie obsadzenia stanów orbitalnych. 
Ta nieciągła zmiana energii jest proporcjonalna do energii oddziaływania kulombowskiego $U$. Dla stanów $d$ w metalach przejściowych parametr $U$ można zdefiniować
jako energię związaną z transferem elektronu między dwoma atomami z początkowym obsadzeniem $d^n$,
który prowadzi do zwiększenia ilości elektronów na jednym atomie $d^{n+1}$ i zmniejszenia na drugim $d^{n-1}$
%
\begin{equation}
U=E(d^{n+1})+E(d^{n-1})-2E(d^n),
\end{equation} 
%
gdzie $E(d^n)$ jest całkowitą energią układu, w którym pojedynczy atom posiada obsadzenie $d^n$.
Ta relacja oznacza, że sama przerwa energetyczna dana wzorem (\ref{energygap}) powinna być liniową funkcją $U$.
Potwierdzają to obliczenia przeprowadzone w ramach metody LDA+U.

Jako przykład omówię wyniki obliczeń gestości stanów elektronowych dla izolatora Motta Fe$_2$SiO$_4$, otrzymane metodą GGA+U \cite{Mariana}.
Na rysunku 6.2 pokazane sa wyniki dla stanu antyferromagnetycznego (AFM).
Dla $U=0$ dostajemy nieprawidłowy stan metaliczny ($\Delta_{KS}=0$) z energią Fermiego przecinającą pasmo $t_{2g}$. 
Pasmo to obsadzone jest przez elektrony z mniejszościowym kierunkiem spinów (kierunek przeciwny do momentu magnetycznego danego atomu). 
Obliczenia dla $U=4.5$ eV dają prawidłowy stan elektronowy z przerwą energetyczną równą około 2 eV,
która rozdziela pasmo $t_{2g}$ na dwie części: obsadzone dolne pasmo Hubbarda i nieobsadzone górne pasmo Hubbarda.

\begin{figure}[h!]
\centering
\includegraphics[scale=0.21]{Fe2SiO4-2.pdf}
\caption{Gęstość stanów elektronowych w Fe$_2$SiO$_4$ dla $U=0$ (górny panel) i dla $U=4.5$~eV (dolny panel). Rysunek z pracy \cite{Mariana}.}
\label{fig:mott}
\end{figure}

\begin{figure}[t!]
\centering
\includegraphics[scale=0.18]{Fe2SiO4-1.pdf}
\caption{Zależność przerwy energetycznej od parametru oddziaływania kulombowskiego $U$ w Fe$_2$SiO$_4$ dla stanu antyferromagnetycznego (czerwone kółka)
i ferromagnetycznego (niebieskie kwadraty). Rysunek z pracy \cite{Mariana}.}
\label{fig:gap}
\end{figure}

Wartość przerwy energetycznej zależy od dokładnej wartości parametru $U$, co pokazano na rysunku 6.3.
Dla małych wartości $U\leq 2$ eV, przerwa energetyczna równa jest zeru. Wynika to z kryterium, które mówi, że w izolatorach Motta, 
niezerowa przerwa tworzy się dla wartości energii oddziaływania kulombowskiego większych od szerokości
pasma elektronowego, $U>W$. Rzeczywiście, z rysunku 6.2 wynika, że szerokość pasma $t_{2g}$ jest równa z dobrym przybliżeniem $W=2$ eV.
Gdy energia oddziaływania kulombowskiego $U$ przewyższa $W$, przerwa energetyczna rośnie praktycznie liniowo z jej wartością. 
Jak widać z rysunku 6.3 wartości przerwy zależą od rodzaju uporządkowania magnetycznego i dla stanu AFM są większe niż dla stanu FM.

\chapter{Polaryzacja elektryczna}

\section{Teoria fazy Berry'ego}

Polaryzacja elektryczna odgrywa ważną rolę w wielu zjawiskach fizycznych występujacych w dielektrykach. 
Wyznaczenie wartości polaryzacji elektrycznej w materiałach periodycznych jakimi są kryształy stanowiło duży problem teoretyczny. 
W obliczeniach z pierwszych zasad, dzięki periodycznym warunkom brzegowym, takie układy traktowane sa jako nieskończone
i zwykle nie bierze sie pod uwagę wpływu powierzchni materiału. 
W układzie skończonym można wyznaczyć polaryzację elektryczną obliczając całkowity moment dipolowy złożony z części jonowej
i elektronowej
%
\begin{equation}
\bm{P}=\frac{e}{\Omega}\sum_jZ_j\bm{R}_j+\frac{1}{\Omega}\int d\bm{r} n(\bm{r}) \bm{r},
\label{polaryzacja}
\end{equation}
% 
gdzie zarówno sumowanie po wszystkich jonach o ładunku $Z_je$, jak i całkowanie gestości elektronowej $n(\bm{r})$ 
odbywa się po całej objetości $\Omega$. Z tego samego wzoru można wyliczyć zmianę polaryzacji $\Delta \bm{P}$
wywołaną zmianą rozkładu gęstości elektronowej $\Delta n(\bm{r})$, np. na skutek przemieszczenia się atomów. 

W układach periodycznych własności całego kryształu zależą od rozmieszczenia atomów w komórce prymitywnej.
Wydaje się, że całkowity moment dipolowy kryształu i polaryzację elektryczną można wyznaczyć stosując wzór (\ref{polaryzacja}),
do pojedynczej komórki. Jednak takie podejście możliwe jest tylko w skrajnych przepadkach kryształów jonowych
lub molekularnych i odpowiada klasycznemu opisowi typu Clausiusa-Mossottiego, w którym występują izolowane dipole elektryczne.
Gęstość elektronowa jest wielkością ciągłą i nie jest możliwe podzielenia całego obszaru kryształu w sposób jednoznaczny.  
Co więcej, wiązania kowalencyjne, ktore występuja w ferroelektrykach, mają naturę kwantową i poprawny opis polaryzacji elektrycznej
wymaga formalizmu kwantowego. 

Punktem startowym, który umożliwia wyznaczenie polaryzacji w układach periodycznych jest 
fundamentalna zależność między lokalną wartością polaryzacji elektronowej i przepływającym prądem
%
\begin{equation}
\bm{P}_e(\bm{r},t)=\int_{t_0}^t dt' j_e(\bm{r},t').
\label{delP}
\end{equation}
%
Wzór ten umożliwia również pomiar zmiany polaryzacji $\Delta \bm{P}$ przez wyznaczenie przepływającego prądu.
Przykładowo, mierząc zmiany polaryzacji w funkcji deformacji kryształu można wyznaczyć stałą piezoelektryczną.
Przełomem w teorii były prace, w których pokazano związek polaryzacji elektrycznej z geometryczną fazą Berry'ego~\cite{berry1,berry2}. 
W obliczeniach zamiast czasu wprowadza się parametr $\lambda\in [0,1]$, który  
opisuje adiabatyczną transformację układu generującą zmianę polaryzacji
%
\begin{equation}
\Delta\bm{P}_e = \int_0^1 d\lambda \frac{\partial\bm{P}_e}{\partial\lambda}. 
\end{equation}
%
Przykładowo parametr $\lambda$ może być współrzędną, która określa położenie atomu w sieci krystalicznej.  
Stosując wyrażenie z rachunku zaburzeń można powiązać pochodną polaryzacji ze strukturą elektronową 
%
\begin{equation}
\frac{\partial\bm{P}_e}{\partial\lambda}=-i\frac{e\hbar}{\Omega m_e}\sum_{\bm{k}}\sum_{n=1}^{occ}\sum_{=1}^{emp}\frac{\langle\psi^{\lambda}_{\bm{k}n}|\bm{p}
|\psi^{\lambda}_{\bm{k}m}\rangle\langle\psi^{\lambda}_{\bm{k}m}|\partial V^{\lambda}_{\text{KS}}/\partial\lambda|\psi^{\lambda}_{\bm{k}n}\rangle}{(\varepsilon^{\lambda}_{\bm{k}n}-\varepsilon^{\lambda}_{\bm{k}m})^2}+c.c.,
\label{partialP}
\end{equation}
%
gdzie sumowanie wykonywane jest po wszystkich stanach zajętych i pustych z pierwszej strefy Brillouina. 
Elementy maciorzowe operatora pędu można wyrazić zależnością
%
\begin{equation}
\langle\psi^{\lambda}_{\bm{k}n}|\bm{p}|\psi^{\lambda}_{\bm{k}m}\rangle=\frac{m_e}{\hbar}\langle u^\lambda_{\bm{k}n}|[\nabla_{\bm{k}},H^\lambda_{\bm{k}}]|u^\lambda_{\bm{k}m}\rangle,
\end{equation}
%
w której $u^\lambda_{\bm{k}m}$ jest częścią periodyczną funkcji Blocha, a hamiltonian dany jest wyrażeniem
%
\begin{equation}
H^\lambda_{\bm{k}}=\frac{1}{2m_e}(-i\hbar\nabla+\hbar\bm{k})^2+V^\lambda_\text{KS}.
\end{equation}
%
Podobnie można wyrazić elementy macierzowe pochodnej potencjału Kohna-Shama
%
\begin{equation}
\langle\psi^{\lambda}_{\bm{k}n}|\frac{\partial V^{\lambda}_{\text{KS}}}{\partial\lambda}|\psi^{\lambda}_{\bm{k}m}\rangle=\frac{m_e}{\hbar}\langle u^\lambda_{\bm{k}n}|[\frac{\partial}{\partial\lambda},H^\lambda_{\bm{k}}]|u^\lambda_{\bm{k}m}\rangle.
\end{equation}
%
Po wstawieniu elementów macierzowych do (\ref{partialP}) otrzymujemy
%
\begin{equation}
\Delta \bm{P}_e=\frac{-ie}{8\pi^3}\sum_{n=1}^{occ}\int_\text{BZ}d\bm{k}\int_0^1d\lambda[\langle\nabla u^\lambda_{\bm{k}n}|                  
\frac{\partial u^\lambda_{\bm{k}n}}{\partial\lambda}\rangle-\langle\frac{\partial u^\lambda_{\bm{k}n}}{\partial\lambda}|                  
\nabla_{\bm{k}} u^\lambda_{\bm{k}n}\rangle].
\end{equation}
Wykonując całkowanie przez części i wykorzystując periodyczność funkcji $u_{\bm{k}n}$ dostajemy
%
\begin{equation}
\Delta \bm{P}_e=\bm{P}^{\lambda=1}_e-\bm{P}^{\lambda=0}_e,
\label{pol1}
\end{equation}
%
gdzie
%
\begin{equation}
\bm{P}^\lambda_e=\frac{ie}{8\pi^2}\sum_{n=1}^{occ}\int_\text{BZ}d\bm{k}\langle u^\lambda_{\bm{k}n}|\nabla_{\bm{k}}|u^\lambda_{\bm {k}n}\rangle.
\label{pol2}
\end{equation}
%
Występująca w tym wzorze całka jest bezpośrednio związana z fazą Berry'ego stanów elektronowych~\cite{Zak}.
Wielkość wektorowa $\bm{A}(\bm{k})=i\langle u^\lambda_{\bm{k}n}|\nabla_{\bm{k}}|u^\lambda_{\bm {k}n}\rangle$
nazywa się potencjałem cechowania lub połączeniem Berry'ego, a całka tej wielkość
po zamkniętym obszarze jest fazą Berry'ego, którą dla danego pasma $n$ i kierunku $j$ można zapisać w formie
%
\begin{equation}
\phi^\lambda_{nj}=\frac{i}{\Omega_\text{BZ}}\int_\text{BZ} d\bm{k}\langle u^\lambda_{\bm{k}n}|\bm{G}_j\nabla_{\bm{k}}|u^\lambda_{\bm {k}n}\rangle,
\end{equation}
%
gdzie $\bm{G}_j$ jest wektorem sieci odwrotnej. 
Polaryzacja związana z pasmem $n$ wyraża się wzorem
%
\begin{equation}
\bm{P}^\lambda_n=\frac{e}{2\pi\Omega}\sum_j\phi^\lambda_{nj}\bm{R}_j.
\end{equation}
%
Otrzymany wynik, który pokazuje związek polaryzacji elektrycznej z fazą funkcji falowej nie jest zaskoczeniem.
W mechanice kwantowej, faza funkcji falowej jest bezpośrednio związana z prądem elektrycznym
i zgodnie ze wzorem (\ref{delP}) określa również polaryzację elektryczną. 
Tak jak każda faza, również polaryzacja elektryczna nie jest wartością zdefiniowaną jednoznaczne, a jedynie określoną {\it modulo}
stała wartość. Żeby określić tę wartość zakładamy, że stan początkowy ($\lambda=0$) i końcowy ($\lambda=1$) opisane
są tym samym hamiltonianem. Wtedy funkcje falowe dla tych stanów mogą różnić się tylko wartością fazy
%
\begin{equation}
u^{\lambda=1}_{\bm{k}n}(\bm{r})=e^{i\theta_{\bm{k}n}}u^{\lambda=0}_{\bm{\bm{k}n}}(\bm{r}).
\end{equation}
%
Zmianę fazy można zapisać w ogólnej formie
%
\begin{equation}
\theta_{\bm{k}n}=\beta_{\bm{k}n}+\bm{k}\bm{R}_n,
\end{equation}
%
gdzie $\beta_{\bm{k}n}$ jest funkcją periodyczną w $\bm{k}$. Zgodnie ze wzorem (\ref{pol2}), zmiana polaryzacji w tym przypadku wynosi
%
\begin{equation}
\Delta\bm{P}_e=-\frac{e}{8\pi^3}\sum_{n=1}^{occ}\int_\text{BZ}d\bm{k}\nabla_{\bm{k}}\theta_{\bm{k}n}=\frac{e}{\Omega}\sum_{n=1}^{occ}\bm{R}_n=\frac{e}{\Omega}\bm{R}.
\end{equation}
%
Dla każdej sieci krystalicznej, można określić najmniejszą wartość $e\bm{R}_1/\Omega$, 
która definiuje kwant polaryzacji, związany z symetrią translacyjną potencjału elektronowego, $V(\bm{r})=V(\bm{r}-\bm{R}_1)$.
Ze względu na periodyczność kryształu wartość bezwględna polaryzacji nie ma większego sensu i istotna jest zmiana polaryzacji między dwoma stanami.
Przesunięcia atomów, które zwykle są brane pod uwagę, są znacznie mniejsze niż odległości między atomami i wyliczana zmiana polaryzacji
jest dużo mniejsza niż kwant polaryzacji, $\Delta\bm{P}_e\ll e\bm{R}_1/\Omega$. Pozwala to wyeliminować niejednoznaczność wartości polaryzacji
w praktycznych zastosowaniach.  

Polaryzację elektryczna można wyrazić również w reprezentacji funkcji Wanniera,
które sa transformatami Fouriera funkcji Blocha
%
\begin{equation}
\psi^{\lambda}_{\bm{k}j}=e^{i\bm{k}{r}}u^\lambda_{\bm{k}n}(\bm{r})=\frac{1}{\sqrt{N}}\sum_je^{i\bm{k}\bm{R}_j}w^\lambda_n(\bm{r}-\bm{R}_j),
\end{equation}
%
gdzie $\bm{R}_j$ są wektorami sieci krystalicznej. Wyliczają z tej zależności $u^\lambda_{\bm{k}n}$ i wstawiając do wzoru (\ref{pol2}) otrzymujemy 
%
\begin{equation}
\Delta\bm{P}_e=-\frac{e}{\Omega}\sum_{n=1}^{occ} [\int d\bm{r} \bm{r}|w^{\lambda=1}_n(\bm{r})|^2-\int d\bm{r} \bm{r}|w^{\lambda=0}_n(\bm{r})|^2].
\end{equation}
%
Ten wzór pokazuje, że zmiana polaryzacji jest proporcjonalna do przesunięcia środka ciężkości rozkładu gestości ładunku,
zlokalizowanego na orbitalach Wanniera, wywołanego adiabatyczną zmianą potencjału elektronowego.


\section{Ładunki efektywne}

Zgodnie ze wzorem (\ref{polaryzacja}), całkowita zmiana polaryzacji wzdłuż kierunku $\alpha$
składa się z części jonowej i elektronowej
%
\begin{equation}
(\Delta\bm{P})_\alpha=(\Delta\bm{P}_\text{ion})_\alpha+(\Delta\bm{P}_e)_\alpha.
\label{pol3}
\end{equation}
% 
Część elektronowa polaryzacji dana jest wyrażeniami (\ref{pol1}) i (\ref{pol2}), natomiast część jonowa wyliczana jest ze wzoru
%
\begin{equation}
(\Delta\bm{P}_\text{ion})_\alpha=\frac{e}{\Omega}Z_\text{ion}u_\alpha,
\label{pol4}
\end{equation}
%
gdzie $Z_{ion}$ jest ładunkiem rdzenia atomowego, a $u_\alpha$ wartością przesunięcia atomów.
Zmianę całkowitej polaryzacji można zapisać w formie
%
\begin{equation}
(\Delta\bm{P})_\alpha = \frac{\partial P_\alpha}{\partial u_\beta}u_\beta = \frac{e}{\Omega}Z^*_{\alpha\beta} u_\beta,
\label{pol5}
\end{equation}
%
gdzie $Z^*_{\alpha\beta}$ jest tensorem ładunków efektywnych Borna.
Liczba niezależnych i niezerowych składowych tensora zależy od symetrii kryształu i ilości nierównowaznych położeń atomowych. 
Wszystkie składowe tensora $Z^*_{\alpha\beta}$ można wyznaczyć korzystając ze wzoru (\ref{pol5}), wyliczając
zmianę polaryzacji dla wychyleń atomów w różnych kierunkach. 

Ładunki efektywne przyjmują niezerowe wartości we wszystkich izolatorach i odgrywają ważną rolę przy opisie
dynamiki sieci. Przesunięcia jonów związane z podłużnymi modami optycznymi (LO) w centrum strefy Brillouina ($\bm{k}=0$), które aktywne są w podczerwieni ({\it ang. infrared modes}), generują makroskopową polaryzację elektryczną i w ten sposób modyfikują siły międzyatomowe.
Te dodatkowe siły powodują, że energia modów LO blisko punktu $\Gamma$ jest większa niż poprzecznych modów optycznych (TO).
Powoduje to rozszczepienie LO-TO, które może być wyliczone jeżeli znamy ładunki efektywne jonów i stałą dielektryczną materiału~\cite{PCM}.

 
\section{Ferroelektryki}

Ferroelektryki należą do osobnej grupy izolatorów, w których spontanicznie pojawia się polaryzacja elektryczna w określonych
warunkach termodynamicznych. Praktyczne wykorzystanie ferroelektryków wynika z możliwości przełączania kierunku polaryzacji
po przyłożeniu pola elektrycznego. W metalach indukowane są prądy elektronowe, które ekranują pole elektryczne 
i dlatego lokalizacja elektronów jest warunkiem koniecznym występowania makroskopowej polaryzacji elektrycznej.  
W izolatorach elektrony nie mogą poruszać się swobodnie w krysztale, natomiast biorą udział w prądach polaryzacji, 
powiązanych z lokalną wartością polaryzacji elektrycznej.  
W ferroelektrykach ważną role odgrywają wiązania kowalencyjne i hybrydyzacja między różnymi stanami elektronowymi 
umożliwiająca transfer ładunku między jonami.
Najbardziej podstawowy mechanizm tworzenia się fazy ferroelektrycznej związany jest ze strukturalnym przejściem fazowym. 
W fazie paraelektrycznej, występującej zwykle w wyższych temperaturach, sieć krystaliczna posiada symetrię centrosymetryczną,
w której średnia polaryzacja $\bm{P}=0$. Aby powstała spontaniczna polaryzacja, musi zajść transformacja układu
do struktury bez centrum symetrii, czyli takiej, która nie jest niezmiennicza wzlędem operacji ($\bm{r}\rightarrow -\bm{r}$).
To jest warunek konieczny występowania fazy ferroelektrycznej, gdzie $\bm{P}\neq 0$.

\begin{figure}[h!]
\centering
\includegraphics[scale=1]{BaTiO3.pdf}
\caption{Deformacja sieci krystalicznej ferroelektryka o strukturze perowskitu ABO$_3$.}
\label{fig:batio3}
\end{figure}

Przykładem ferroelektryków są materiały o strukturze perowskitu ABO$_3$, pokazanej na rysunku (\ref{fig:batio3}).
Do tej grupy należy tytanian baru BaTiO$_3$ i wiele innych tlenków zawierajacych metale przejściowe.
W narożach komórki znajdują się kationy A, natomiast znajdujący się w środku kation B otoczony jest oktaedrem anionów O.
W wysokich temperaturach kryształ posiada centrosymetryczną symetrię regularną. Przy obniżaniu temperatury w materiale pojawia się
niestabilność sieci, związana z wystepowaniem miękkiego modu fononowego, którego energia dąży do zera przy zbliżaniu się do przejścia fazowego.
Wychylenia atomów w miękkim modzie, pokazane schematycznie na rysunku (\ref{fig:batio3}), prowadzą do statycznej 
deformacji sieci poniżej temperatury krytycznej. W strukturze tetragonalnej złamana jest symetria centrosymetryczna 
i pojawia się niezerowa polaryzacja elektryczna, której żródłem jest względne przesunięcie anionów (O),
względem kationów (A i B). Zmiana polaryzacji spowodowana jest nie tylko sztywnym przesunięciem ładunków zlokalizowanych na jonach,
ale również, jak to zostało wyjaśnione w poprzednich rozdziałach, przepływem prądów elektronowych wzdłuż wiązań kowalencyjnych. 
 
Wartość polaryzacji elektrycznej w ferroelektrykach można wyliczyć startując ze stanu początkowego ($\lambda=0$), który odpowiada położeniom jonów w strukturze 
centrosymetrycznej ($\bm{P}=0$).
Wtedy zmiana polaryzacji (\ref{pol3}), indukowana przysunięciem jonów do położeń w strukturze niecentrosymetrycznej,
odpowiada wartości spontanicznej polaryzacji, która pojawia się w wyniku przejścia fazowego.
Wartość polaryzacji wyliczona dla ferroelektryka KNbO$_3$ wynosi 0.35 C/m$^2$~\cite{resta1993}
i bardzo dobrze zgadza się z wartością wyznaczoną eksperymentalnie 0.37 C/m$^2$. 
 
W ferroelektrykach występują szczególnie duże wartości ładunków efektywnych, często znacznie przewyższające ładunki statyczne jonów. 
Przykładowo w strukturze regularnej BaTiO$_3$ ładunki efektywne jonów znajdujacych się w czterech nierównoważnych położeniach przyjmują wartości: 
$Z^*_\text{Ba}=2.75$, $Z^*_\text{Ti}=7.06$, $Z^*_\text{O1}=-5.83$ i $Z^*_\text{O2}=-2.11$~\cite{zhong1994}.
Znacznie podwyższone wartości ładunków efektywnych jonów Ti i O1 wynikają z silnej hybrydyzacji między stanami $3d$ i $2p$ 
w wiązaniu kowalencyjnym Ti-O1. Zmiana odległości między tymi jonami generuje transfer ładunku (prąd polaryzacji)
i zmianę polaryzacji zgodnie ze wzorem (\ref{delP}). Całkowita zmiana polaryzacji składa się z części jonowej, zależnej od ładunków statycznych~(\ref{pol4}),
i części elektronowej, która powoduje zwiększenie wartości ładunków efektywnych.
Duże ładunki efektywne powodują, że rozszczepienie LO-TO przyjmuje bardzo duże wartości w ferroelektrykach~\cite{zhong1994}.

 
\chapter{Oddziaływanie van der Waalsa}
\label{sec:vdw}

\section{Podstawowe własności}

Oddziaływanie van der Waalsa (vdW) nazywane również dyspersyjnym jest efektem korelacji elektronowych, które nie są uwzględnione
w standardowych funkcjonałach wymienno-korelacyjnych.
Jest to oddziaływanie przyciągające odgrywające ważną role w kryształach molekularnych, takich jak
kryształy gazów szlachetnych, w których atomy mają zamknięte powłoki elektronowe,
oraz materiały składające się z neutralnych cząsteczek. 
Ważne jest też w układach biologicznych, np. ma duży wpływ na prawidłowy kształt i działanie białek.
Żródłem tego oddziaływania są fluktacje kwantowe dipola elektrycznego na jednym atomie (molekule), które indukują dipol elektryczny na innym atomie (molekule).
Wzajemne oddziaływanie tych dipoli prowadzi do słabego przyciągania atomów lub molekuł. 
W klasycznym opisie, pole elektryczne wytworze przez moment dipolowy $p_1$ w odległości $R$ można wyrazić wzorem 
%
\begin{equation}
E=\frac{p_1}{R^3}.
\end{equation}
% 
Jeżeli pole elektryczne oddziałuje na cząsteczkę o polaryzowalności $\alpha$ to indukuje moment dipolowy
%
\begin{equation}
p_2=\frac{\alpha p_1}{R^3}.
\end{equation}
%
Średnia energia oddziaływania tych dwóch dipoli wynosi
%
\begin{equation}
U=-\frac{\alpha p_1^2}{R^6}.
\end{equation}
%
Uwzglednienie korelacji elektronowych w ramach teorii perturbacyjneje prowadzi do takiej samej zależność oddziaływania 
od odległości $\sim R^{-6}$, ale poprawne wyznaczenie polaryzowalności i siły oddziaływania wymaga wyjścia poza najprostsze 
przybliżenie faz przypadkowych ({\it ang. random-phase approximation} - RPA).   
 
\section{Poprawki do funkcjonału energii}
 
\subsection{Metody D2 i D3}  
 
W obliczeniach DFT można uwzlędnić przybliżoną wartość energii oddziaływania vdW jako poprawkę do całkowitej energii układu
%
\begin{equation}
E_\text{tot}=E_\text{KS}+E_\text{vdW}
\end{equation}
%
gdzie $E_\text{KS}$ jest wyznaczonym w procedurze samozgodnej funkcjonałem Kohna-Shama.
W takim podejściu oddziaływanie vdW nie wpływa bezpośrednio na strukturę elektronową, 
jedynie poprzez modyfikację odległości między atomami.
 
W wersji, którą zaproponował Grimme~\cite{grimme2004,grimme2006}, zależność energii vdW od odległości między atomami $R_{ij}$
ma postać
%
\begin{equation}
E_\text{vdW}=-\frac{1}{2}\sum_{i,j}f_\text{d}(R_{ij})\frac{C_{6ij}}{R_{ij}^6},
\end{equation}
% 
gdzie $f_\text{d}$ jest funkcją tłumiącą, która określa asymptotyczne zachowanie dla $R_{ij}\rightarrow 0$  
i wyklucza podwójne uzwględnianie efektów korelacyjnych przy pośrednich odległościach.
Jej rolą jest zapewnienie, że oddziaływanie vdW ograniczone jest do odległości większych niż 
długości typowych wiązań, poprawnie opisywanych w ramach standardowych funkcjonałów.
Współczynniki dyspersyjne $C_{6ij}$ zdefiniowane są dla każdej pary atomów
jako średnie geometryczne parametrów atomowych
%
\begin{equation}
C_{6ij}=\sqrt{C_{6i}C_{6j}}.
\end{equation}
%
Dla każdego atomu określa się parametr $C_6=0.05NI_\text{p}\alpha$, gdzie $I_\text{p}$ jest potencjałem jonizującym, $\alpha$ jest
statyczną polaryzowalnościa dipolową, a $N=2, 10, 18, 36, 54$ dla kolejnych rzędów układu okresowego pierwiastków.  
Współczynnik skalujący wyznaczonony jest tak, aby zreprodukować odległości międzyatomowe i energie wiązania
w odpowiedniej grupie pierwiastków. Funkcję tłumiącą można zapisać w formie \cite{grimme2006}
%
\begin{equation}
f_\text{d}(R_{ij})=\frac{s_6}{1+e^{-d(R_{ij}/R_{0ij}-1)}},
\label{damping}
\end{equation}
%
gdzie $s_6$ jest współczynnikiem określonym dla danego funkcjonału DFT, np. $s_6=0.75$ dla PBE,
$d$ jest współczynnikim tłumienia, którego typowa wartość wynosi 20, a $R_{0ij}=R_{0i}+R_{0j}$ jest promieniem odcięcia
i wyraża się sumą promieni vdW dla danych atomów. 
Parametry $C_{6i}$ i $R_{0i}$ są zdefiniowane dla wszystkch atomów i nie zależą od rodzaju materiału. 
Dodatkowo w obliczeniach określa się maksymalny zasięg oddziaływania vdW. 
Opisany schemat (DFT-D2) umożliwia poprawienie wyliczanych sił i odległości międzyatomowych w kryształach gazów szlachetnych oraz
układach molekularnych~\cite{bucko2010}. Oddziaływanie vdW odgrywa również ważną rolę w materiałach warstwowych,
dla których standardowe obliczenia DFT dają zwykle zaniżoną wartość siły wiążącej między warstwami. 
Przykładowo, stała sieci $c$ w graficie, której wartość w przybliżeniu PBE ($c=8.84$~\AA)
jest znacznie zawyżona w porównaniu do wartości eksperymentalnej ($6.71$~\AA), wyliczona z poprawką D2 wynosi $6.45$~\AA~\cite{bucko2010}.     

W kolejnym podejściu (D3), Grimme uwzględnił dodatkowo drugi wyraz w wzorze na energię oddziaływania proporcjonalny do $R^{-8}$~\cite{grimme2010}
%
\begin{equation}
E_\text{vdW}=-\frac{1}{2}\sum_{i,j}[f_{\text{d},6}(R_{ij})\frac{C_{6ij}}{R_{ij}^6}+f_{\text{d},8}(R_{ij})\frac{C_{8ij}}{R_{ij}^8}].
\end{equation}
%
Współczynniki dyspersyjne $C_{6ij}$ wyznaczane są ze wzoru Casimira-Poldera 
%
\begin{equation}
C_{6ij}=\frac{3}{\pi} \int d\omega \alpha_i(\omega)\alpha_j(\omega), 
\label{CP}
\end{equation}
%
gdzie $\alpha_i(\omega)$ jest średnią polaryzowalnością dipolową dla atomu $i$.
Współczynniki $C_{8ij}$ wyliczane są z zależności $C_{8ij}=3C_{6ij}\sqrt{Q_iQ_j}$, gdzie $Q_i=s\sqrt{Z_i}\langle r^4\rangle_i/\langle r^2\rangle_i$.
Funkcje tłumiące mają postać
%
\begin{equation}
f_{\text{d},n}(R_{ij})=\frac{s_n}{1+6(R_{ij}/(s_{R,n}R_{0ij}))^{-\alpha_n}}.
\end{equation}
%
W tym wzorze optymalizowane są parametry $s_6$, $s_8$ i $s_{R,6}$, a pozostałe mają ustalone wartości ($s_{R,8}=1$, $\alpha_6=14$, $\alpha_8=16$). 
Alternatywnie można zastosować funkcję tłumiącą Becke-Johnsona (BJ) w postaci
%
\begin{equation}
f_{\text{d},n}(R_{ij})=\frac{s_nR_{ij}^n}{R_{ij}^n+(a_1R_{0ij}+a_2)^n},
\end{equation}
%
gdzie $s_n$, $a_1$ i $a_2$ sa parametrami dopasowania. Promień odcięcia można wyznaczyć znając współczynniki dyspersyjne
$R_{0ij}=\sqrt{C_{8ij}/C_{6ij}}$. Analiza porównawcza przeprowadzona dla dużej grupy materiałów
pokazała niewielki wpływ rodzaju funkcji tłumiącej na wyliczone odległości między atomami i trochę lepszą dokładność 
przy zastosowaniu funkcji BJ do własności termodynamicznych~\cite{grimme2011}. 

\subsection{Metoda TS}

W metodzie zaproponowanej przez Tkatchenko i Schefflera (TS)~\cite{tkatchenko2009} bierze się pod uwagę tylko wyraz proporcjonalny do $R^{-6}$, ale
uwzględnia się zależność współczynników dyspersyjnych i promienia odcięcią od gęstości ładunku.
Wprowdzamy przybliżoną zależność polaryzowalności od częstości
%
\begin{equation}
\alpha_i(\omega)=\frac{\alpha^0_i}{1-(\omega/\eta_i)^2},
\label{alpha}
\end{equation}
%
gdzie $\alpha^0_i$ jest statyczną polaryzowalnością i $\eta_i$ jest efektywną częstością dla atomu $i$.
Po wstawieniu (\ref{alpha}) do (\ref{CP}) otrzymujemy formułę Londona
%
\begin{equation}
C_{6ij}=\frac{3\eta_i\eta_j}{2(\eta_i+\eta_j)}\alpha_i^0\alpha_j^0.
\end{equation}
%
Dla jednakowych atomów ($i=j$) otrzymujemy wzór
%
\begin{equation}
\eta_i=\frac{4C_{6i}}{3(\alpha_i^0)^2},
\end{equation}
%
który prowadzi do zależności
%
\begin{equation}
C_{6ij}=\frac{2C_{6i}C_{6j}}{\frac{\alpha^0_j}{\alpha^0_i}C_{6i}+\frac{\alpha^0_i}{\alpha^0_j}C_{6j}}.
\end{equation}
%
Współczynniki dyspersyjne dla poszczególnych atomów znajdujących się wewnątrz cząsteczek lub ciał stałych
wyznacza się z wartości dla izolowanych atomów używając następującej formuły $C_{6i}=\nu_i^2C^{free}_{6i}$, gdzie $\nu_i$
określa efektywną objętość atomu (podział Hirshfelda)
%
\begin{equation}
\nu_i=\frac{V_i^{eff}}{V_i^{free}}=\frac{\int d\bm{r} r^3 w_i(\bm{r}) n(\bm{r})}{\int d\bm{r} r^3 n_i^{free}(\bm{r})},
\end{equation}
%
z wagami Hirshfelda, które wyliczane są dla uśrednionych sferycznie gęstości elektronowych w izolowanych atomach
%
\begin{equation}
w_i(\bm{r})=\frac{n_i^{free}(\bm{r})}{\sum_j n_i^{free}(\bm{r})}.
\end{equation}
%
Podobnie znając wartości polaryzowalności i promieni vdW dla izolowanych atomów, można wyznaczyć te wartości
dla atomów w cząsteczkach i kryształach: $\alpha_i=\nu_i\alpha^{free}$ i $R_{0i}=\nu_i^{\frac{1}{3}}R_{0i}^{free}$.
Promienie vdW służą do wyznaczenia promienia odcięcia $R_{0ij}$ i funkcji tłumiacej $f_{\text{d},n}$ (\ref{damping}).

W omówionych podejściach zakłada się jedynie oddziaływania dwuciałowe, nie biorąc pod uwagę modyfikacji oddziaływania vdW przez otaczające atomy lub molekuły. 
W udoskonalonej wersji metody TS uwzglednia się dalekozasięgowe ekranowanie elektryczne w sposób samozgodny ({\it ang. self-consistent screening} - SCS)~\cite{tkatchenko2012}. Odpowiednie równanie samozgodne pozwalające wyznaczyć zmodyfikowaną polaryzowalność ma postać
%
\begin{equation}
\alpha_i^\text{SCS}(\omega)=\alpha_i(\omega)-\alpha_i(\omega)\sum_{i\neq j}\tau_{ij}\alpha_i^\text{SCS}(\omega),
\end{equation}
%
gdzie $\tau_{ij}$ jest tensorem oddziaływania dipol-dipol. Energię dyspersyjną wyznacza się z tych samych równań co w metodzie TS
z odpowiednio zmodyfikowanymi współczynnikami dyspersyjnymi i promieniami odcięcia
%
\begin{equation}
C_{6i}=\frac{3}{\pi}\int_0^\infty d\omega[\alpha^\text{SCS}_i(\omega)]^2,
\end{equation}
%
\begin{equation}
R_{0i}^\text{SCS}=\Big{(}\frac{\alpha_i^\text{SCS}}{\alpha_i}\Big{)}^{\frac{1}{3}}R_{0i}.
\end{equation}

%\subsection{Metoda MBD}

\chapter{Efekty wielociałowe i stany wzbudzone}

\section{Funkcje Greena i energia własna}

Dokładny opis stanów elektronowych wymaga wyjścia poza przybliżenie niezależnych cząstek.   
Stosując formalizm drugiego kwantowania, zapiszemy hamiltonian układu oddziałujących elektronów 
w ogólnej formie
%
\begin{equation}
H=\int d^3r \psi^\dagger(\bm{r},t) h_0(\bm{r}) \psi(\bm{r},t) +\frac{1}{2}\int d^3rd^3r' \psi^\dagger(\bm{r},t)\psi^\dagger(\bm{r}',t)v(\bm{r}-\bm{r}') \psi^\dagger(\bm{r}',t)\psi^\dagger(\bm{r},t),
\end{equation}
%  
gdzie $h_0(x)$ opisuje energię kinetyczną oraz oddziaływanie elektronów z zewnętrznym polem, $v(\bm{r}-\bm{r}')$ jest
potencjałem oddziaływania kulombowskiego między elektronami, a $\psi(\bm{r},t)$ jest operatorem pola w obrazie Heisenberga 
%
\begin{equation}
\psi(\bm{r},t)=e^{\frac{i}{\hbar}Ht}\psi(\bm{r})e^{-\frac{i}{\hbar}Ht}.
\label{field}
\end{equation}
%
Operatory pola $\psi^\dagger(\bm{r})$ i $\psi(\bm{r})$ opisują kreacje i anihilację elektronu w punkcie $\bm{r}$.

Stan podstawowy $|\Phi_N\rangle$ układu $N$ oddziałujących elektronów spełnia równanie 
%
\begin{equation}
H|\Phi_N\rangle=E|\Phi_N\rangle.
\end{equation}
%
Dalej zakładamy, że stan podstawowy jest unormowany $\langle\Phi_N|\Phi_N\rangle=1$.

W kwantowej teorii układów wielu czastek można powiązać spektrum wzbudzeń z jednocząstkowymi funkcjami Greena, 
zwanymi również propagatorami, które definiujemy
%
\begin{equation} 
iG(x,x')=\langle \Phi_N|T\psi(x)\psi^\dagger(x')|\Phi_N\rangle=\begin{cases}
        \langle \Phi_N|\psi(x)\psi^\dagger(x')|\Phi_N\rangle & t>t',\\
        -\langle \Phi_N|T\psi^\dagger(x')\psi(x)|\Phi_N\rangle & t'>t,
             \end{cases}    
\label{greenf}                      
\end{equation}
% 
gdzie wprowadzono oznaczenie $x=(\bm{r},t)$, a $T$ jest operatorem uporządkowania w czasie. 
Funkcja Greena jest amplitudą prawdopodobieństwa propagacji elektronu z punktu $x'$ do punktu $x$ dla $t>t'$ 
lub dziury z punktu $x$ do $x'$ dla $t'>t$. Te dwa przypadki opisują odpowiednio proces fotoemisji i odwrotnej fotoemisji.
Funkcje Greena dla elektronów są macierzami ze względu na spinowe stopnie swobody. Dla uproszczenia zapisu pomijamy
tutaj wskaźniki numerujące stany spinowe.
Zajomość funkcji Green pozwala wyznaczyć energię stanu podstawowego, wartości oczekiwanego jednocząstkowych operatorów
w stanie podstawowym oraz jednoelektronowe spektrum wzbudzeń.
Funkcje Greena spełniaja równanie ruchu
%
\begin{equation}
[i\frac{\partial}{\partial t}-H_0(x))]G(x,x')-\int dx''\Sigma(x,x'')G(x'',x')=\delta(x-x'),
\end{equation}
%
gdzie $H_0 =h_0+V_H$ ($V_H$ jest potencjałem Hartree), a $\Sigma$ nosi nazwę energii własnej i opisuje oddziaływania wymienno-korelacyjne.

Dla układu jednorodnego o objętości $V$, funkcję Greena możemy zapisać w formie transformaty Fouriera
%
\begin{equation}
G(x,x')=\sum_{\bm{k}}\int\frac{d\omega}{2\pi V}e^{i\bm{k}(\bm{r}-\bm{r}')}e^{-i\omega(t-t')}G(\bm{k},\omega).
\label{fourier}
\end{equation}
%
$G(\bm{k},\omega)$ jest funkcją Greena, która opisuje propagację elektronu o zdefiniowanym wektorze falowym $\bm{k}$.
Rozważmy na początku układ $N$ nieodziałujących elektronów. Operatory pola można zapisać
%
\begin{equation}
\psi(\bm{r})=\frac{1}{V}\sum_{\bm{k}}e^{-i\bm{k}\bm{r}}c_{\bm{k}},
\label{free}
\end{equation}
%
gdzie $c_{\bm{k}}$ jest operatorem anihilacji electronu, dla którego $c_{\bm{k}}|\Phi_N\rangle=0$.
Odpowiednio $c^\dagger_{k}$ jest operatorem kreacji elektronu o wektorze falowym $\bm{k}$.
Można przedefiniować operatory kreacji i anihilacji wprowadzając osobne oznaczenia dla cząstek i dziur
%
\begin{equation}
c_{\bm{k}}=\begin{cases}
a_{\bm{k}} & k>k_F,\\
b^\dagger_{-\bm{k}} & k<k_F,\end{cases}
\end{equation}
%
gdzie $k_F$ jest wektorem falowym Fermiego. Hamiltonian w tej reprezentacji przyjmuje postać
%
\begin{equation}
H_0=\sum_{\bm{k}}\hbar\omega_{\bm{k}} c^\dagger_{\bm{k}}c_{\bm{k}}=\sum_{\bm{k}>\bm{k}_F}\hbar\omega_{\bm{k}} a^\dagger_{\bm{k}}a_{\bm{k}}-\sum_{\bm{k}<\bm{k}_F}\hbar\omega_{\bm{k}} b^\dagger_{\bm{k}}b_{\bm{k}}+\sum_{\bm{k}<\bm{k}_F}\hbar\omega_{\bm{k}}.
\end{equation}
%
Wykorzystując wzory (\ref{free}) i (\ref{field}) otrzymujemy zależność dla funkcji Greena swobodnych elektronów
%
\begin{equation}
iG_0(x,x')=\frac{1}{(2\pi)^3}\int d^3k e^{i\bm{k}(\bm{r}-\bm{r}')}e^{-i\omega_{\bm{k}}(t-t')}[\theta(t-t')\theta(k-k_F)-\theta(t'-t)\theta(k_F-k)].
\end{equation}
%
gdzie sumowanie po stanach $\bm{k}$ zostało zamienione na całkę.
Funkcję schodkową $\theta(t-t')$  można zapisać przy pomocy wzoru
%
\begin{equation}
\theta(t-t')=-\int_{-\infty}^{\infty}\frac{d\omega}{2\pi i}\frac{e^{-i\omega(t-t')}}{\omega+i\eta}=\begin{cases}1 & t>t',\\0 & t<t',\end{cases}
\end{equation}
% 
gdzie $\eta$ jest infinitezymalnie małą liczbą. Zastosowanie tego wyrażenie po prostych przekształceniach
prowadzi do wzoru
%
\begin{equation}
G_0(x,x')=\frac{1}{(2\pi)^4}\int d^3k e^{i\bm{k}(\bm{r}-\bm{r}')}e^{-i\omega(t-t')}\Big{[}\frac{\theta(k-k_F)}{\omega-\omega_{\bm{k}}+i\eta}+\frac{\theta(k_F-k)}{\omega-\omega_{\bm{k}}-i\eta}\Big{]}.
\end{equation}
%
Porównujące ten wzór z (\ref{fourier}) dostajemy wyrażenie na funkcję Greena swobodnych elektronów
w przestrzeni pędów
%
\begin{equation}
G_0(\bm{k},\omega)=\frac{\theta(k-k_F)}{\omega-\omega_{\bm{k}}+i\eta}+\frac{\theta(k_F-k)}{\omega-\omega_{\bm{k}}-i\eta},
\label{G0}
\end{equation}
%
Występujące w tym wzorze dwa wyrazy odpowiadają odpowiednio propagacji elektronu dla wektora falowego $k$
większego do wektor falowego Fermiego $k_F$ i dziury dla $k<k_F$. 

Funckję Greena dla oddziałujących elektronów można zapisać w postaci szeregu perturbacyjnego
%
\begin{equation}
G(\bm{k},\omega)=G_0(\bm{k},\omega)+G_0(\bm{k},\omega)\Sigma_{\bm{k}}(\omega)G_0(\bm{k},\omega)+G_0(\bm{k},\omega)\Sigma_{\bm{k}}(\omega) G_0(\bm{k},\omega)\Sigma_{\bm{k}}(\omega) G_0(\bm{k},\omega)+...
\end{equation}
%
gdzie $\Sigma_{\bm{k}}(\omega)$ jest transformatą Fouriera energii własnej. 
Szereg ten można zapisać w formie równania Dysona
%
\begin{equation}
G(\bm{k},\omega)=G_0(\bm{k},\omega)+G_0(\bm{k},\omega)\Sigma_{\bm{k}}(\omega)G(\bm{k},\omega),
\end{equation}
%
lub po przekształceniu
%
\begin{equation}
\Sigma_{\bm{k}}(\omega)=G_0^{-1}(\bm{k},\omega)-G^{-1}(\bm{k},\omega).
\end{equation}
%
Łącząc ten wzór z wyrażeniem (\ref{G0}) dla swobodnych elektronów ($k>k_F$)
dostajemu wzór na funkcję Greena
%
\begin{equation}
G(\bm{k},\omega)=\frac{1}{\omega-\omega_{\bm{k}}-\Sigma_{\bm{k}}(\omega)}.
\end{equation}
%
Z tego wzoru wynika, że jednocząstkowe stany o energii $\omega_{\bm{k}}$ są modyfikowane zarówno
przez rzeczywistą część energii własnej, która przesuwa wartości energii, jak również przez
jej część urojoną. 

Znając funkcję Greena można wyznaczyć funkcję spektralną, która opisuje widmo energetyczne elektronów w funkcji wektora falowego,
wyznaczane w pomiarach metodą kątowo-rozdzielczej spektroskopii fotoemisyjnej (ARPES).
Funkcja spektralna wyznaczana jest części urojonej funkcji Greena
%
\begin{equation}
A_{\bm{k}}(\omega)=\frac{1}{\pi}|\text{Im} G_{\bm{k}}(\omega)|=\frac{1}{\pi}\frac{|\text{Im} \Sigma_{\bm{k}}(\omega)|}{[\omega-\omega_{\bm{k}}-\text{Re} \Sigma_{\bm{k}}(\omega)]^2+[\text{Im} \Sigma_{\bm{k}}(\omega)]^2}.
\end{equation}
%
Dla układu nieoddziałujących elektronów ($\Sigma=0$) funkcja spektralna składa się z delt Diraca w położeniach,
które odpowiadają energiom $\omega_{\bm{k}}$.
Część rzeczywista energii własnej powoduje przesunięcie energii elektronów
%
\begin{equation}
\varepsilon_{\bm{k}}=\omega_{\bm{k}}+\Sigma_{\bm{k}}(\varepsilon_{\bm{k}}),
\label{e_k}
\end{equation}
%
natomiast część urojona powoduje poszerzenie pików, co związane jest z redukcją czasu życia stanów elektronowych.
Jeżeli część urojona powoduje tylko niewielkie zaburzenie, istnieje jednoznaczna relacja między stanami
elektronów swobodnych i elektronów oddziałujących. Takie stany elektronowe lub dziurowe, które występują blisko energii Fermiego nazywamy {\it kwasicząstkami}.
Przy zbliżaniu się do poziomu Fermiego czas życia stanów elektronów rośnie ze względu na zmniejszajacą się ilość dostępnych stanów do których elektron
może się rozproszyć. 

Blisko energii Fermiego część rzeczywista energii wałsnej zależy liniowo od energii i można ją rozwinąć do liniowego wyrazu
%
\begin{equation}
\text{Re} \Sigma_{\bm{k}}(\omega)=\text{Re} \Sigma_{\bm{k}}(\omega_{\bm{k}})+(\omega-\omega_{\bm{k}})\frac{\partial\text{Re}\Sigma_{\bm{k}}(\omega)}{\partial\omega}\Big{|}_{\omega=\omega_{\bm{k}}}.
\end{equation}
%
Wprowadzając czynnik renormalizacyjny, który jest miarą charakteryzującą spektrum kwasicząstek
%
\begin{equation}
Z_{\bm{k}}=\frac{1}{1-\frac{\partial\text{Re}\Sigma_{\bm{k}}(\omega)}{\partial\omega}\Big{|}_{\omega=\omega_{\bm{k}}}}
\end{equation}
%
możemy zapisać wzór na energię (\ref{e_k}) w postaci
%
\begin{equation}
\varepsilon_{\bm{k}}=\omega_{\bm{k}}+Z_{\bm{k}}\text{Re}\Sigma_{\bm{k}}(\omega_{\bm{k}}).
\end{equation}
%
Również gęstość spektralną kwasicząstek można przybliżyć wzorem
%
\begin{equation}
A_{\bm{k}}(\omega)=Z_{\bm{k}}\frac{Z_{\bm{k}}\text{Im}\Sigma_{\bm{k}}(\varepsilon_{\bm{k}})}{(\omega-\varepsilon_{\bm{k}})^2+[Z_{\bm{k}}\text{Im}\Sigma_{\bm{k}}(\varepsilon_{\bm{k}})]^2},
\end{equation}
% 
która ma kształt linii Lorentza w położeniu $\varepsilon_{\bm{k}}$, szerokości $Z_{\bm{k}}\text{Im}\Sigma_{\bm{k}}(\varepsilon_{\bm{k}})$
i amplitudzie $Z_{\bm{k}}$, nazywanej również wagą sprektralna kwasiczastek. $Z_{\bm{k}}$ określa m.in. natężenie (amplitudę) rozpraszania fotoemisyjnego,
zmianę masy efektywnej kwazicząstek, jak również skok w obsadzeniu stanów na powierzchni Fermiego. 
Dla układu, w którym część rzeczywista energii własnej nie zależy od energii $Z_{\bm{k}}=1$, ta nieciągłość pokrywa się z rozkładem Fermiego-Diraca dla $T=0$.
Energie kwasiczastek, które są funkcjami wektorów falowych, określają strukturę pasmową materiału.
W układach silnie skorelowanych część gęstości spektralnej przekazywana jest do stanów leżących dalej od poziomu Fermiego
i pojawiają się dodatkowe struktury (satelity), których nie da się zinterpretować jako wzbudzenia kwasicząstek.

\section{Teoria perturbacyjna}

Podstawowe podejście stosowane w teorii pola opiera się na rozwinięciu perturbacyjnym funkcji Greena~\cite{AGD}.
Najwygodniejszym jest tutaj obraz oddziaływania, w którym hamiltonian zapisany jest w formie
%
\begin{equation}
H(t)=H_0+V_\text{I}(t),
\end{equation}
%
gdzie pierwszy wyraz opisuje układ swobodnych elektronów, a drugi jest zależnym od czasu operatorem oddziaływania
%
\begin{equation}
V_\text{I}(t)=\frac{1}{2}\int d\bm{r}\int d\bm{r}'\psi_\text{I}^\dagger(\bm{r},t)\psi_\text{I}^\dagger(\bm{r}',t)v(\bm{r},\bm{r}')
\psi_\text{I}(\bm{r}',t)\psi_\text{I}(\bm{r},t).
\end{equation}
%
Występujace w tym wyrażeniu operatory pola otrzymujemy przez transformację
%
\begin{equation}
\psi_\text{I}(\bm{r},t)=e^{iH_0(t-t_0)}\psi(\bm{r})e^{-iH_0(t-t_0)}.
\end{equation}
%
W obrazie oddziaływania, ewolucja stanów opisana jest zależnością
%
\begin{equation}
|t\rangle=U_\text{I}(t,t_0)|t_0\rangle,
\end{equation}
gdzie operator ewolucji może być zapisany w postaci szeregu
%
\begin{equation}
U_\text{I}=1+\sum_{n=1}^{\infty}\frac{(-1)^n}{n!}\int_{t_0}^tdt_1...\int_{t_0}^tdt_nT\Big{[}V_\text{I}(t_1)...V_\text{I}(t_n)\Big{]}.
\end{equation}
%
Wykorzystując ten operator i wprowadzając oznaczenie $S=U_\text{I}(\infty,-\infty)$
możemy zapisać funkcję Greena w postaci rozwinięcia
%
\begin{equation}
G(x,x')=-i\frac{\langle\Phi_0|T[\psi(x)\psi^{\dagger}(x')S]|\Phi_0\rangle}{\langle\Phi_0|S|\Phi_0\rangle},
\end{equation}
%
gdzie $|\Phi_0\rangle$ jest stanem podstawowym dla układu nieoddziałujących elektronów opisanego hamiltonianem $H_0$. 
Występujące w tym szeregu wyrazy można wyrazić przy pomocy funkcji Greena i wyrazów opisujących oddziaływania kulombowskie.
Prowadzi to do wzoru, który można zapisać w zwartej formie przy pomocy sumy wyznaczników
%
\begin{equation}
G(x,x')=\sum_{n=0}^\infty i^n\int v(x_1,x_1')...v(x_n,x'_n)\begin{vmatrix}
G_0(x,x') &  G_0(x,x_1)  & \dots & G_0(x,x'_n) \\ 
G_0(x_1,x') & G_0(x_1,x^+_1) & \dots & G_0(x,x'_n) \\
\dots                &  \dots                 &       &\dots \\
G_0(x'_n,x') & G_0(x'_n,x')  & \dots & G_0(x'_n,x'^+_n) \\
\end{vmatrix}.
\end{equation}
%  
Każdy z wyrazów rozwinięcia można otrzymać stosując metodę diagramów Feynmana opisaną szczegółowo w wielu podręcznikach \cite{AGD,FW,martin2}.





\section{Metoda GW}
\label{sec:gw}

Energię własną można powiązać z funkcją Greena
%
\begin{equation}
\Sigma(x_1,x_2)=i\int dx_3dx_4 G(x_1,x_3)W(x_1,x_4)\Lambda(x_3,x_2,x_4).
\end{equation}
%
$W$ jest ekranowanym potencjałem kulombowskim
%
\begin{equation}
W(x_1,x_2)=\int dx_3\epsilon^{-1}(x_1,x_3)V(\bm{r}_3-\bm{r}_2),
\end{equation}
%
\begin{equation}
\epsilon^{-1}(x_1,x_2)=\frac{\delta V(x_1)}{\delta \phi(x_2)},
\end{equation}
%
gdzie $V$ jest sumą potencjału Hartree $V_H$ i potencjału zewnętrznego $\phi$.
$\Lambda$ jest funkcją wierzchołkową
%
\begin{equation}
\Lambda(x_1,x_2,x_3) = -\frac{\delta G^{-1}(x_1)}{\delta V(x_3)}.
\end{equation}
%



%\section{Teoria dynamicznego średniego pola (DMFT)}
%\label{sec:dmft}

%\section{Kwantowe Monte Carlo (QMC)}
%\label{sec:qmc}

%\bibliographystyle{acm}
%\bibliography{biblio.bib}



\thebibliography{500}


\bibitem{Schrodinger} E. Schr\"{o}dinger, {\it An undulatory theory of the mechanics of atoms and molecules}, Phys. Rev. {\bf 28}, 1049 (1926).

\bibitem{HL} W. Heitler and F. London, {\it Wechselwirkung neutraler Atome und homöopolare Bindung nach der Quantenmechanik},
Zeitschrift für Physik {\bf 44}, 455 (1927). 

\bibitem{hartree28} D. Hartree, {\it The wave mechanics of an atom with a non-coulomb central field. Part I. Theory and methods}, Proc. Cambridge Philos. Soc. {\bf 24}, 89 (1928).

\bibitem{mulliken} R. S. Mulliken, {\it The assignment of quantum numbers for electrons in molecules}, 
Phys. Rev. {\bf 32}, 186 (1928).

\bibitem{JC} H. M. James and A. S. Coolidge, {\it The ground state of the hydrogen molecule}, J. Chem. Phys. {\bf 1}, 825 (1933).

\bibitem{fermi26} E. Fermi, {\it Zur Quantelung des idealen einatomigen Gases}, Z. Physik {\bf 36}, 902 (1926).

\bibitem{Dirac26} P. A. M. Dirac, {\it On the theory of quantum mechanics}, Proc. Roy. Soc. Lond. A {\bf 112}, 661 (1926).

\bibitem{Pauli25} W. Pauli, Z. Physik {\bf 31}, 765 (1925).
 
\bibitem{Bloch} F. Bloch, {\it Über die Quantenmechanik der Elektronen in Kristallgittern}, Z. Phys. {\bf 52}, 555 (1928).

\bibitem{peierls} R. Peierls, {\it Zur Theorie der galvanomagnetischen Effekte}, Z. Phys. {\bf 53}, 255 (1929). 

\bibitem{Wilson} A. H. Wilson, {\it The theory of electronic semi-conductors}, Proc. R. Soc. Lond. A {\bf 133}, 458 (1931). 

\bibitem{fock30} V. Fock, {\it Naherungsmethode zur Losung des quantenmechanischen Mehrkorperproblems}, Z. Physik {\bf 61}, 126 (1930).
  
\bibitem{slater30} J. C. Slater, {\it Note on Hartree's method}, Phys. Rev. {\bf 35}, 210 (1930).

\bibitem{slater51} J. C. Slater, {\it A simplification of the Hartree-Fock method}, Phys. Rev. {\bf 81}, 385 (1951).

\bibitem{wigner34} E. P. Wigner, {\it On the interactions of electrons in metals}, Phys. Rev. {\bf 46}, 1002 (1934).

\bibitem{GB} M. Gellman and K. A. Brueckner, {\it Correlation energy of an electron gas at high-density},
Phys. Rev. {\bf 106}, 364 (1957).

\bibitem{wigner33} E. Wigner and F. Seitz, {\it On the constitution of metallic sodium}, Phys. Rev. {\bf 43}, 804 (1933).

\bibitem{slater37} J. C. Slater, {\it Wave functions in a periodic potential}, Phys. Rev. {\bf 51}, 846 (1937).

\bibitem{K47} J. Korringa, {\it On the calculation of the energy of a Bloch wave in a metal}, Physica {\bf 13}, 392 (1947).

\bibitem{KR54} W. Kohn and N. Rostoker, {\it Solution of the Schr\"{o}dinger equation in periodic
lattices with an application to metallic lithium}, Phys. Rev. {\bf 94}, 1111 (1954).

\bibitem{herring40} C. Herring, {\it A new method for calculating wave functions in crystals}, Phys. Rev. {\bf 57}, 250 (1940).

\bibitem{Antoncik} E. Antoncik, {\it A new formulation of the method of nearly free electrons}, Czech. J. Phys. {\bf 4}, 439 (1954).

\bibitem{KP} J. C. Philips and L. Kleiman, {\it New method for calculating wave functions in crystals and molecules}, Phys. Rev. B {\bf 116}, 287 (1959).

\bibitem{kohn64} P. Hohenberg and W. Kohn, {\it Inhomogeneous electron gas}, Phys. Rev. {\bf 136}, 864 (1964).

\bibitem{kohn65} W. Kohn and L. J. Sham, {\it Self-consistent equations including exchange and correlation effects}, Phys. Rev. {\bf 140}, 1133 (1965).

\bibitem{CeperleyAlder80} D. M. Ceperley and B. J. Alder, {\it Ground state of the electron gas by a stochastic method},
Phys. Rev. Lett. {\bf 45}, 566 (1980).

\bibitem{PZ} J. P. Pardew and A. Zunger, {\it Self-interaction correction to density-functional approximations for many-electron systems},
Phys. Rev. B {\bf 23}, 5048 (1981).

\bibitem{VWN} S. Vosko, L. Wilk, and M. Nusair, {\it Accurate spin-dependent electron liquid correlation energies for local spin density calculations: a critical analysis}, Can. J. Phys. {\bf 58}, 1200 (1983).

\bibitem{Langreth83} D. C. Langreth and M. J. Mehl, {\it Beyond the local-density approximation in calculations of ground-state electronic properties}, Phys. Rev. B {\bf 28}, 1809 (1983).

\bibitem{Pardew86} J. P. Perdew and Y. Wang, {\it Accurate and simple density functional for the electronic exchange energy: generalized gradient approximation}, Phys. Rev. B {\bf 33}, 8800 (1986).

\bibitem{Becke88} A. D. Becke, {\it Density-functional exchange-energy approximation with correct asymptotic behavior},
Phys. Rev. A {\bf 38}, 3098 (1988).

\bibitem{Andersen75} O. K. Andersen, {\it Linear methods in band theory}, Phys. Rev. B {\bf 12}, 3060 (1975).

\bibitem{HSC} D. R. Hamann, M. Schl\"{u}ter, and C. Chiang, {\it Norm-conserving pseudopotentials}, Phys. Rev. Lett. {\bf 43}, 1494 (1979).

\bibitem{BHS} G. B. Bachelet, D. R. Hamann, and M. Schl\"{u}ter, {\it Pseudopotentials that work: from H to Pu}, Phys. Rev. B {\bf 26}, 4199 (1982). 

\bibitem{Vanderbilt90} D. Vanderbilt, {\it Soft self-consistent pseudopotentials in a generalized eigenvalue formalism}, Phys. Rev. B {\bf 41}, 7892 (1990).

\bibitem{jones} R. O. Jones and O. Gunnarsson, {\it The density functional formalism, its applications and prospects}, Rev. Mod. Phys. {\bf 61}, 689 (1989).

\bibitem{payne} M. C. Payne, M. P. Teter, D. C. Allan, T. A. Arias, 
                and J. D. Joannopoulos, {\it Iterative minimization techniques for ab initio total-energy calculations: molecular dynamics and conjugate gradients},   
                Rev. Mod. Phys. {\bf 64}, 1045 (1992).
                
\bibitem{parlinski} K. Parlinski, Z. Q. Li, and Y. Kawazoe, {\it First-principles determination of the soft mode in cubic ZrO$_2$}, 
                    Phys. Rev. Lett. {\bf 78}, 4063 (1997).                
                
\bibitem{baroni}  S. Baroni, S. Gironcoli, A. Dal Corso, and P. Giannozzi, {\it Phonons and related crystal properties from density-functional perturbation theory}, 
                  Rev. Mod. Phys. {\bf 73}, 515 (2001).

\bibitem{martin} R. M. Martin, {\it Electronic structure. Basic theory and practical methods}, Cambridge University Press, 2010.

\bibitem{peierls} N. F. Mott and R. Peierls, {\it Discussion of the paper by de Boer and Verwey}, Proc. Phys. Soc. (London) {\bf A49}, 72 (1937).
               
\bibitem{mott} N. F. Mott, {\it The basis of the electron theory of metals, with special reference to the transition metals}, Proc. Phys. Soc. (London) {\bf A62}, 416 (1949).

\bibitem{hubbard} J. Hubbard, {\it Electron correlations in narrow energy bands}, Proc. Roy. Soc. (London) {\bf 276}, 238 (1963).

\bibitem{terakura} K. Terakura, T. Oguchi, A. R. Williams, and J. K\"{u}bler, {\it Band theory of insulating transition-metal monoxides: Band-structure calculations},
                  Phys. Rev. B {\bf 30}, 4734 (1984).

\bibitem{becke93} A. D. Becke, {\it A new mixing of Hartree-Fock and local density-functional theories}, J. Chem. Phys. {\bf 98}, 1372 (1993).                   
    
              
\bibitem{anisimov} V. I. Anisimov, J. Zaanen, and O. K. Andersen, {\it Band theory and Mott insulators: Hubbard U instead of Stoner I},
                   Phys. Rev. B {\bf 44}, 943 (1991).

\bibitem{czyzyk} M. T. Czy\.{z}yk and G. A. Sawatzky, {\it Local-density functional and on-site correlations: The electronic structure of La$_2$CuO$_4$ and LaCuO$_3$}, Phys. Rev. B {\bf 49}, 14211 (1994).

\bibitem{orbital1} A. I. Liechtenstein, V. I. Anisimov, and J. Zaanen, {\it Density-functional theory and strong interactions: orbital ordering in Mott-Hubbard insulators}, Phys. Rev. B {\bf 52}, R5467 (1995).
                  
\bibitem{orbital2} T. Mizokawa and A. Fujimori, {\it Electronic structure and orbital ordering in perovskite-type 3d transition-metal oxides studied by Hartree-Fock band-structure calculations}, Phys. Rev. B {\bf 54}, 5368 (1996).

\bibitem{orbital3} V. I. Anisimov, F. Aryasetiawan and A. I. Lichtenstein, {\it First-principles calculations of the electronic structure and spectra of strongly correlated systems: the LDA+U method}, J. Phys.: Cond. Mat. {\bf 9}, 767 (1997).

\bibitem{dudarev} S. L. Dudarev, G. A. Botton, S. Y. Savrasov, C. J. Humphreys, A. P. Sutton, {\it Electron-energy-loss spectra and the structural stability of nickel oxide: an LSDA+U study}, Phys. Rev. B {\bf 57}, 1505 (1998).

%\bibitem{Pu-LDAU} S. Y. Savrasov and G. Kotliar, {\it Ground state theory of $\delta$−Pu}, Phys. Rev. Lett. {\bf 84}, 3670 %(2000).                                     

\bibitem{Georges} A. Georges, G. Kotliar, W. Krauth, and M. J. Rozenberg, {\it Dynamical mean-field theory of strongly correlated fermion systems and the limit of infinite dimensions}, Rev. Mod. Phys. {\bf 68}, 13 (1996).

\bibitem{BO} M. Born and J. R. Oppenheimer, {\it On the quantum theory of molecules}, Ann. Physik {\bf 84}, 457 (1927).    

\bibitem{Koopman} T. Koopmans, Physica {\bf 1}, 104 (1934).

\bibitem{GTO} S. F. Boys, {\it Electron wave functions I. A general method for calculation for the stationary states of any molecular system}, Proc. Roy. Soc. London {\bf 200}, 542 (1950).

\bibitem{MP} H. J. Monkhorst and J. D. Pack, {\it Special points for Brillouin-zone integrations}, Phys. Rev. B {\bf 13}, 5188 (1976).

\bibitem{thomas} L. H. Thomas, {\it The calculations of atomic fields}, Proc. Cambridge Phil. Roy. Soc. {\bf 23}, 542 (1927). 

\bibitem{fermi1927} E. Fermi, {\it Un metodo statistico per la determinazione di alcune prioprietà dell'atomo}, Rend. Accad. Naz. Lincei. {\bf 6}, 602 (1927).

\bibitem{dirac} P. A. M. Dirac, {\it Note on exchange phenomena in the Thomas-Fermi atom}, Proc. Cambridge Phil. Roy. Soc. {\bf 26}, 376 (1930). 

\bibitem{janak} J. F. Janak, {\it Proof that $\partial E/\partial n_i=\varepsilon_i$ in density-functional theory}, Phys. Rev. B {\bf 18}, 7165 (1977).

\bibitem{PW91} J. P. Perdew and Y. Wang, {\it Accurate and simple representation of the electron-gas
correlation energy}, Phys. Rev. B {\bf 45}, 13244 (1992). 

\bibitem{Pardew92} J. P. Perdew, J. A. Chevary, S. H. Vosko, K. A. Jackson, M. R. Pederson, D. J. Singh, and C. Fiolhais,
{\it Atoms, molecules, solids, and surfaces: Applications of the generalized gradient approximation for exchange and correlation},
Phys. Rev. B {\bf 46}, 6671 (1992).

\bibitem{PBE}   J. P. Perdew, K. Burke, and M. Ernzerhof, {\it Generalized gradient approximation made simple},
                   Phys. Rev. Lett.  {\bf 77}, 3865 (1996). 
                   
\bibitem{Hellmann} H. Hellmann, {\it Einführung in die Quantenchemie}, Leipzig, Franz Deuticke, 1937.

\bibitem{Feynman} R. Feynman, {\it Forces in molecules}, Phys. Rev. {\bf 56}, 340 (1939).


\bibitem{Blochl} P. E. Bl\"{o}chl, {\it Projector augmented-wave method}, Phys. Rev. B {\bf 50}, 17953 (1994). 

\bibitem{Vasp}  G. Kresse and J. Furthm\"{u}ller, {\it Efficiency of ab-initio total energy calculations for metals and semiconductors using a plane-wave basis set}, Comput. Mater. Sci. {\bf 6}, 15 (1996).


\bibitem{PawVasp}  G. Kresse and J. Joubert, {\it From ultrasoft pseudopotentials to the projector augmented-wave method}, Phys. Rev. B {\bf 59}, 1758 (1999).
                               
\bibitem{FeSe} A. Ptok, K. J. Kapcia, P. Piekarz, and A. M. Oleś, {\it The ab initio study of unconventional
superconductivity in CeCoIn$_5$ and FeSe}, New, J. Phys. {\bf 19}, 063039 (2017).

\bibitem{QE} P. Giannozzi {\it et al.}, J. Phys.: Condens. Matter {\bf 21}, 395502 (2009).

\bibitem{STO1} J. C. Slater, {\it Atomic shielding constants}, Phys. Rev. {\bf 36}, 57 (1930).

\bibitem{STO2} J. C. Slater, {\it Analytic atomic wave function}. Phys. Rev. {\bf 42}, 33 (1932).

\bibitem{Moruzzi77} V. L. Moruzzi, A. R. Williams, and J. F. Janak, {\it Local density theory of metallic cohesion}, Phys. Rev. B {\bf 15}, 2854 (1977).

\bibitem{CPA1} G. M. Stocks, W. M. Temmerman, and B. L. Gyorffy, {\it Complete solution of the Korringa-Kohn-Rostocker coherent-potential-approximation equations: Cu-Ni alloys}, Phys. Rev. Lett. {\bf 41}, 339 (1978).

\bibitem{CPA2} J. S. Faulkner and G. M. Stocks, {\it Calculating properties with the coherent-potential approximation}, Phys. Rev. B {\bf 21}, 3222 (1980).

\bibitem{singh91} D. Singh, {\it Ground-state properties of lanthanum: Treatment of extended-core states}, Phys. Rev. B 43, 6388 (1991).

\bibitem{kummel} S. K\"{u}mmel and L. Kronik, Rev. Mod. Phys. {\bf 80}, 3 (2008).

\bibitem{Sharp} R. T. Sharp and G. K. Horton, {\it A variational approach to the unipotential many-electron problem}, Phys. Rev. {\bf 90}, 317 (1953).

\bibitem{Talman} J. D. Talman and W. F. Shadwick, {\it Optimized effective atomic central potential}, Phys. Rev. A {\bf 14}, 36 (1976). 

\bibitem{Sahni} V. Sahni, J. Gruenebaum, and J. P. Perdew, Phys. Rev. B {\bf 26}, 4371 (1982).

\bibitem{RE} P. Strange, A. Svane, W. M. Temmerman, Z. Szotek, and H. Winter, {\it Understanding the valency of rare earths from first-principles theory}, Nature {\bf 399}, 756 (1999).

\bibitem{svane1990} A. Svane and O. Gunnarsson, Phys. Rev. Lett. {\bf 65}, 1148 (1990).

\bibitem{temmerman2001} W. M. Temmerman, H. Winter, Z. Szotek, and A. Svane, Phys. Rev. Lett. {\bf 86}, 2435 (2001).

\bibitem{PBE0}  J. P. Perdew, M. Ernzerhof, and K. Burke, {\it Rationale for mixing exact exchange with density functional approximations},  
                   J. Chem. Phys {\bf 105}, 9982 (1996).         
  
\bibitem{pople89} J. A Pople, M. Head-Gordon, D. J. Fox, K. Raqhavachari, and L. A. Curtiss, {\it Gaussian-1 theory: a general procedure for prediction of molecular energies}, J. Chem. Phys. {\bf 90}, 5622 (1989).  
                   
\bibitem{HSE} J. Heyd, G. E. Scuseria, and M. Ernzerhof, {\it Hybrid functionals based on a screened Coulomb potential}, J. Chem. Phys. {\bf 118}, 8207 (2003).

\bibitem{Becke1998} A. D. Becke, {\it A new inhomogeneity parameter in density-functional theory}, J. Chem. Phys. {\bf 109}, 2092 (1998).

\bibitem{Becke2006} A. D. Becke and E. R. Johnson, {\it A simple effective potential for exchange}, J. Chem. Phys. {\bf 124}, 221101 (2006). 

\bibitem{Tran2009} F. Tran and P. Blaha, {\it Accurate band gaps of semiconductors and insulators with a semilocal exchange-correlation potential}, 
Phys. Rev. Lett. {\bf 102}, 226401 (2009). 

\bibitem{ELF1} A. D. Becke and K. E. Edgecombe, {\it A simple measure of electron localization in atomic and molecular systems}, J. Chem. Phys. {\bf 92}, 5397 (1990).

\bibitem{ELF2} B. Silvi and A. Savin, {\it Classification of chemical bonds based on topological analysis of electron localization functions}, Nature {\bf 371}, 683 (1994). 

\bibitem{ELF3} J. Sun, B. Xiao, Y. Fang, R. Haunschild, P. Hao, A. Ruzsinszky, G. I. Csonka, G. E. Scuseria, and J. P. Perdew, {\it Density Functionals that Recognize Covalent, Metallic, and Weak Bonds}, Phys. Rev. Lett. {\bf 111}, 106401 (2013). 

\bibitem{metaGGA} J. P. Perdew, S. Kurth, A. Zupan, and P. Blaha, {\it Accurate Density Functional with Correct Formal Properties: A Step Beyond the Generalized Gradient Approximation}, Phys. Rev. Lett. 82, 2544 (1999). 

\bibitem{TPSS} J. Tao, J. P. Perdew, V. N. Staroverov, and G. E. Scuseria, {\it Climbing the density functional ladder: nonempirical meta–generalized gradient
approximation designed for molecules and solids}, Phys. Rev. Lett. {\bf 91}, 146401 (2003).


\bibitem{revTPSS} J. P. Perdew, A. Ruzsinszky, G. I. Csonka, L. A. Constantin, and J. Sun, {\it Workhorse semilocal density functional for condensed matter physics and quantum chemistry}, Phys. Rev. Lett. {\bf 103}, 026403 (2009).

\bibitem{SCAN} J. Sun, A. Ruzsinszky, and J. P. Perdew, {\it Strongly constrained and appropriately normed semilocal density functional}, Phys. Rev. Lett. {\bf 115}, 036402 (2015).

\bibitem{Becke1989} A. D. Becke and M. R. Roussel, Phys. Rev. A {\bf 39}, 3761 (1989).

\bibitem{coco1} M. Cococcioni and S. de Gironcoli, {\it Linear response approach to the calculation of the effective interaction parameters
in the LDA+U method}, Phys. Rev. B {\bf 71}, 035105 (2005).

\bibitem{coco2} B. Himmetoglu, A. Floris, S. de Gironcoli, and M. Cococcioni, {\it Hubbard-corrected DFT energy functionals: the LDA+U
description of correlated systems}, Int. J. Quant. Chem. {\bf 114}, 14 (2014).

\bibitem{pardew82} J. P. Perdew, R. G. Parr, M. Levy, and J. L. Balduz, {Density-functional theory for fractional particle number: derivative discontinuities of the energy}, Phys. Rev. Lett. {\bf 49}, 1691 (1982).
                   
\bibitem{norman83} M. R. Norman and J. P. Pardew, {\it Simplified self-interaction correction applied to the energy bands of neon and sodium chloride},
Phys. Rev. B {\bf 28}, 2135 (1983).                   

\bibitem{Gap-hyb} A. J. Garza and G. E. Scuseria, {\it Predicting band gaps with hybrid density functionals}, J. Phys. Chem. Lett. {\bf  7}, 4165 (2016).

\bibitem{Mariana} M. Derzsi, P. Piekarz, P. T. Jochym, J. Łażewski, M. Sternik, A. M. Oleś, and K. Parlinski, {\it Effects of Coulomb interaction on the electronic structure and lattice dynamics of the Mott insulator Fe2SiO4 spinel}, Phys. Rev. B {\bf 79}, 205105 (2009).

\bibitem{berry1} R. D. King-Smith and D. Vanderbilt, {\it Theory of polarization of crystalline solids}, Phys. Rev. B {\bf 47}, 1651 (1993).

\bibitem{berry2} R. Resta, {\it Macroscopic polarization in crastalline dielectrics: the geometric phase approach}, Rev. Mod. Phys. {\bf 66}, 899 (1994).

\bibitem{Zak} J. Zak, {\it Berry’s phase for energy bands in solids}, Phys. Rev. Lett. {\bf 62}, 2747 (1989).

\bibitem{PCM} R. M. Pick, M. H. Cohen, and R. M. Martin, {\it Microscopic theory of force constants in the adiabatic approximation},
Phys. Rev. B {\bf 1}, 910 (1970).

\bibitem{resta1993} R. Resta, M. Posternak, A. Baldereschi, {\it Towards a quantum theory of polarization in ferroelectrics: 
The case of KNbO$_3$}, Phys. Rev. Lett. {\bf 70}, 1010 (1993).

\bibitem{zhong1994} W. Zhong, R. D. King-Smith, and D. Vanderbilt, {\it Giant LO-TO splittings in perovskite ferroelectrics},
Phys. Rev. Lett. {\bf 72}, 3618 (1994).

\bibitem{grimme2004} S. Grimme, {\it Accurate description of van der Waals complexes by density functional theory including empirical corrections}, 
J. Comp. Chem. {\bf 25}, 1463 (2004). 

\bibitem{grimme2006} S. Grimme, {\it Semiempirical gga-type density functional constructed with a long-range dispersion correction}, J. Comp. Chem. {\bf 27}, 1787 (2006). 

\bibitem{bucko2010} T. Bucko, J. Hafner, S. Lebegue, and J. G. Angyan, {\it Improved description of the structure of molecular and layered crystals: ab initio DFT
calculations with van der Waals corrections}, J. Phys. Chem. A {\bf 114}, 11814 (2010).

\bibitem{grimme2010} S. Grimme, J. Antony, S. Ehrlich, and S. Krieg, {\it A consistent and accurate ab initio parametrization of density functional dispersion correction (DFT-D) for the 94 elements H-Pu}, J. Chem. Phys. {\bf 132}, 154104 (2010). 

\bibitem{grimme2011} S. Grimme, S. Ehrlich, and L. Goerigk, {\it Effect of the damping function in dispersion corrected density functional theory}, J. Comp. Chem. {\bf 32}, 1456 (2011). 

\bibitem{tkatchenko2009} A. Tkatchenko and M. Scheffler, {\it Accurate molecular van der Waals interactions from ground-state electron density and free-atom reference data}, Phys. Rev. Lett. {\bf 102}, 073005 (2009).

\bibitem{tkatchenko2012} A. Tkatchenko, R. A. Di Stasio, R. Car, and M. Scheffler, {\it Accurate and efficient method for many-body van der waals interactions}, Phys. Rev. Lett. {\bf 108}, 236402 (2012).
                       
\bibitem{AGD} A. A. Abrikosov, L. P. Gor'kov, and I. Y. Dzyaloshinskii, {\it Quantum Field Theoretical Methods in Statistical Physics}, Pergamon Press, 1965.                       

\bibitem{FW} A. L. Fetter and J. D. Walecka, {\it Quantum Theory of Many-Prticle Systems}, McGraf-Hill Book Company.
                       
\bibitem{martin2} R. M. Martin, L. Reining, and D. M. Ceperley, {\it Interacting Electrons. Theory and Computational Approaches},  Cambridge University Press, 2016.                        
                                          
\end{document}
		     





